\newcommand{\nom}{Porte conteneur}
\newcommand{\sequence}{03}
\newcommand{\num}{04}
\newcommand{\type}{TD}
\newcommand{\descrip}{Résolution d'un problème en utilisant des méthodes algorithmiques}
\newcommand{\competences}{Alt-C3: Concevoir un algorithme répondant à un problème précisément posé}
\documentclass[10pt,a4paper]{article}
  \usepackage[french]{babel}
  \usepackage[utf8]{inputenc}
  \usepackage[T1]{fontenc}
  \usepackage{xcolor}
  \usepackage[]{graphicx}
  \usepackage{makeidx}
  \usepackage{textcomp}
  \usepackage{amsmath}
  \usepackage{amssymb}
  \usepackage{stmaryrd}
  \usepackage{fancyhdr}
  \usepackage{lettrine}
  \usepackage{calc}
  \usepackage{boxedminipage}
  \usepackage[french,onelanguage, boxruled,linesnumbered]{algorithm2e}
  \usepackage[colorlinks=false,pdftex]{hyperref}
  \usepackage{minted}
  \usepackage{url}
  \usepackage[locale=FR]{siunitx}
  \usepackage{multicol}
  \usepackage{tikz}
  \makeindex

  %\graphicspath{{../Images/}}

%  \renewcommand\listingscaption{Programme}

  %\renewcommand{\thechapter}{\Alph{chapter}}
  \renewcommand{\thesection}{\Roman{section}}
  %\newcommand{\inter}{\vspace{0.5cm}%
  %\noindent }
  %\newcommand{\unite}{\ \textrm}
  \newcommand{\ud}{\mathrm{d}}
  \newcommand{\vect}{\overrightarrow}
  %\newcommand{\ch}{\mathrm{ch}} % cosinus hyperbolique
  %\newcommand{\sh}{\mathrm{sh}} % sinus hyperbolique

  \textwidth 160mm
  \textheight 250mm
  \hoffset=-1.70cm
  \voffset=-1.5cm
  \parindent=0cm

  \pagestyle{fancy}
  \fancyhead[L]{\bfseries {\large PTSI -- Dorian}}
  \fancyhead[C]{\bfseries{{\type} \no \numero}}
  \fancyhead[R]{\bfseries{\large Informatique}}
  \fancyfoot[C]{\thepage}
  \fancyfoot[L]{\footnotesize R. Costadoat, C. Darreye}
  \fancyfoot[R]{\small \today}
  
  \definecolor{bg}{rgb}{0.9,0.9,0.9}
  
  
  % macro Juliette
  
\usepackage{comment}   
\usepackage{amsthm}  
\theoremstyle{definition}
\newtheorem{exercice}{Exercice}
\newtheorem*{rappel}{Rappel}
\newtheorem*{remark}{Remarque}
\newtheorem*{defn}{Définition}
\newtheorem*{ppe}{Propriété}
\newtheorem{solution}{Solution}

\newcounter{num_quest} \setcounter{num_quest}{0}
\newcounter{num_rep} \setcounter{num_rep}{0}
\newcounter{num_cor} \setcounter{num_cor}{0}

\newcommand{\question}[1]{\refstepcounter{num_quest}\par
~\ \\ \parbox[t][][t]{0.15\linewidth}{\textbf{Question \arabic{num_quest}}}\parbox[t][][t]{0.85\linewidth}{#1\label{q\the\value{num_quest}}}\par
~\ \\}

\newcommand{\reponse}[4][1]
{\noindent
\rule{\linewidth}{.5pt}\\
\textbf{Question\ifthenelse{#1>1}{s}{} \multido{}{#1}{%
\refstepcounter{num_rep}\ref{q\the\value{num_rep}} }:} ~\ \\
\ifdef{\public}{#3 ~\ \\ \feuilleDR{#2}}{#4}
}

\newcommand{\cor}
{\refstepcounter{num_cor}
\noindent
\rule{\linewidth}{.5pt}
\textbf{Question \arabic{num_cor}:} \\
}

%%\usepackage[a4paper]{geometry}
%\geometry{margin={1cm,1.2cm}}
%\usepackage[francais]{babel}
%\usepackage{nopageno} %pas de numérotation de page
%\pagestyle{plain} %numérotation en bas de page, pas d'entête
%\usepackage{hyperref}
%\usepackage[latin1]{inputenc}

%%%%%%%%%%%%%%%%%%%%%%%%%%%%%%%%%%%%%%%%%%%%%%%%%%%%%%%%%%%%%%%%%%%%%%%%%%%%%%%%%%%%%

\usepackage{amsthm}
\usepackage{amscd}
%\usepackage{mathrsfs}
%\usepackage{amsfonts}
%\usepackage[T1]{fontenc}
%\usepackage{theorem}
\usepackage{lscape}
\usepackage{variations}  % pour faire des tableaux de variations
\usepackage{dsfont}
\usepackage{fancyvrb} % pour mettre Verbatim dans une box

% Pour les figures
\usepackage{subfig}
%\usepackage{calc} % Pour pouvoir donner des formules dans les d�finitions de longueur
%\usepackage{graphicx} % Pour inclure des graphiques 
% Attention : pour inclure des .jpg comme dans l'exemple (ou des .png ou .pdf)
% il faut compiler directement en pdf (commande pdflatex).
% Pour inclure des .eps, il faut compiler avec latex + dvips + ps2pdf.
\usepackage{psfrag}
%\usepackage{color}

%%%%%%%%%%%%%%%%%%%%%%%%%%%%%%%%%%%%%%%%%%%%%%%%%%%%%%%%%%%%%%%%%%%%%%%%%%%%%%%%%%%%%

\theoremstyle{definition}
\newtheorem*{thm}{Théorème}
%\theorembodyfont{\rmfamily}
\newtheorem*{defn}{Définition}
\newtheorem{exercice}{Exercice}
\newtheorem*{problem}{Problème}
\newtheorem*{prop}{Proposition}
\newtheorem*{corollaire}{Corollaire}
\newtheorem*{lemme}{Lemme}
\newtheorem*{remark}{Remarque}
\newtheorem*{notation}{Notation}
\newtheorem*{ex}{Exemple}
\newtheorem*{ppe}{Propriété}
\newtheorem*{meth}{Méthode}
\newtheorem*{rappel}{Rappel}
\newtheorem*{voca}{Vocabulaire}
\setlength{\columnseprule}{0.5pt}


%%%%%%%%%%%%%%%%%%%%%%%%%%%%%%%%%%%%%%%%%%%%%%%%%%%%%%%%%%%%%%%%%%%%%%%%%%%%%%%%%%%%%

\newcommand{\bi}{\bigskip}
\newcommand{\dsp}{\displaystyle}
\newcommand{\noi}{\noindent}
\newcommand{\ov}{\overline}
\newcommand{\dsum}{\displaystyle \sum}
\newcommand{\dprod}{\displaystyle \prod}
\newcommand{\dint}{\displaystyle \int}
\newcommand{\dlim}{\displaystyle \lim}

%%%%%%%%%%%%%%%%%%%%%%%%%%%%%%%%%%%%%%%%%%%%%%%%%%%%%%%%%%%%%%%%%%%%%%%%%%%%%%%%%%%%%


%\newcommand{\pgcd}{\mathrm{pgcd}} % pgcd
%\providecommand{\norm}[1]{\lVert#1\rVert} % norme
%\DeclareMathOperator{\Tan}{Tan}  % espace tangent


\newcommand{\N}{\mathbb{N}}
\newcommand{\Z}{\mathbb{Z}}
\newcommand{\Q}{\mathbb{Q}}
\newcommand{\R}{\mathbb{R}}
\newcommand{\C}{\mathbb{C}}
\newcommand{\K}{\mathbb{K}}
\newcommand{\U}{\mathbb{U}}
\newcommand{\Tr}{\text{Tr}\,}
\newcommand{\pg}{\geqslant}
\newcommand{\pp}{\leqslant}
\newcommand{\bul}{\item[$\bullet$]}
\newcommand{\card}{\text{Card}}
\newcommand{\re}{\text{Re}\;}
\newcommand{\im}{\text{Im}\;}
\newcommand{\Ker}{\text{Ker}\;}
\newcommand{\Vect}{\text{Vect}\;}
\newcommand{\rg}{\text{rg}\;}
\newcommand{\TT}{{}^t\!}
%\newcommand{\sh}{\text{sh}}
%\newcommand{\ch}{\text{ch}}
\newcommand{\Mat}{\text{Mat}}
\usepackage{textcomp}



%%%%%%%%%%%%%%%%%%%%%%%%%%%%%%%%%%%%%%%%%%%%%%%%%%%%%%%%%%%%%%%%%%%%%%%%%%%%%%%%%%%%%%%%%%%%%%%%%%%%%%%%%%%%%%%%%%%%%%%%%%%




\ifdef{\public}{\excludecomment{solution}}

\begin{document}

\begin{center}
{\Large\bf TP \no {\numero} -- \descrip}
\end{center}

\SetKw{KwFrom}{de} 


%Version récursive d’algorithmes dichotomiques. Fonctions produisant à l’aide de print successifs des figures alphanumériques. Dessins de fractales. Énumération des sous-listes ou des permutations d’une liste.
%On évite de se cantonner à des fonctions mathématiques (factorielle, suites récurrentes). On peut montrer le phénomène de dépassement de la taille de la pile.


\begin{defn}
Une fonction récursive est une fonction dont le calcul nécessite d'invoquer la fonction elle-même.
\end{defn}

\begin{exercice}\label{exponentiation naive}
On considère le programme suivant :
\begin{minted}[linenos,frame=lines]{python}
def f(x,p):
	if p==0 :
		return(1)
	else:
		return(x*f(x,p-1))		
\end{minted}
\begin{enumerate}
\item Recopiez le programme et exécutez-le. En général, que renvoie la fonction $f$ ?
\begin{solution}
$f(x,p)$ renvoie la valeur de $x^p$.
\end{solution}
\item Modifiez le programme pour qu'il affiche les valeurs successives de $x$ et de $p$.
\begin{solution}~\\
\vspace{-0.7cm}
\begin{minted}{python}
def f(x,p):
	if p==0
		return(1)
	else:
		print(x,p)
		return(x*f(x,p-1))	
\end{minted}
\end{solution}
\item Que se passe-t-il si on efface les lignes 2, 3 et 4 ? 
\begin{solution}
Le message d'erreur suivant s'affiche : "maximum recursion depth exceeded". On parle de dépassement de la pile. Le programme ne s'arrête jamais car $p$ décroit sans limite.
\end{solution}
\end{enumerate}
\end{exercice}



\begin{exercice}\label{exp rapide}Exponentiation rapide.\\
Dans cet exercice, nous allons optimiser le programme précédent grâce au principe suivant :
\[\text{pour }x\in\mathbb{R},\text{ et }p\in \mathbb{N},\qquad x^p=\left\lbrace
\begin{array}{ll}
1&\text{ si }p=0\\
\left(x^2\right)^{\frac{p}{2}}&\text{ si }p\text{ est pair}\\
x\times \left(x^2\right)^{\frac{p-1}{2}}&\text{ si }p\text{ est impair}\\
\end{array}\right.\]
La fonction \verb?expon? prend comme entrée deux valeurs $x$ et $p$.\\
\verb?expon(x,n) : ?
\begin{itemize}
\item Si $p$ est nul, renvoyer 1.
\item Si $p$ est pair, la fonction s'appelle elle-même, avec comme argument $x^2$ et $\dfrac{p}{2}$.
\item Si $p$ est impair, renvoyer $x\times \texttt{expon}(x^2,\frac{p-1}{2})$.
\end{itemize}
\begin{enumerate}
\item Ecrivez une fonction récursive \verb?expon? qui suit l'algorithme ci-dessus.
\begin{solution}~\\
\vspace{-0.7cm}
\begin{minted}{python}
def expon(x,p):
    if p==0:
        return(1)
    else:
        if p%2==0:
            return(expon(x**2,p//2))
        else:
            return(x*expon(x**2,(p-1)//2))
\end{minted}
\end{solution}
\item On a donc deux fonctions, $f$ de l'exercice \ref{exponentiation naive} et la fonction \verb?expon? qui renvoient le même résultat. Proposez une façon de comparer la complexité des deux fonctions. \\
Deux méthodes :
\begin{itemize}
\item on utilise la fonction \verb?time? de la bibliothèque \verb?time?, présentée au TP n°7 ;
\item (facultatif) on ajoute un compteur au programme à l'aide d'une variable globale.
\end{itemize} 
\begin{solution}
\newpage
Méthode 1 :\\
\begin{minted}{python}
start=time.time()
(expon(x,p))
end=time.time()
print(end-start)
        
start=time.time()
(f(x,p))
end=time.time()
print(end-start)  
\end{minted}

Méthode 2 :\\
\begin{minted}{python}
compteur=0

def expon_compteur(x,p):
    if p==0:
        return(1)
    else:
        global compteur
        compteur=compteur+1
        if p%2==0:
            return(expon_compteur(x**2,p//2))
        else:
            return(x*expon_compteur(x**2,(p-1)//2))

print(expon_compteur(x,p),compteur)

compteur=0

def f_compteur(x,p):
    if p==0 :
        return(1)
    else:
        global compteur
        compteur=compteur+1
        return(x*f_compteur(x,p-1))
\end{minted}
\end{solution}
\end{enumerate}
\end{exercice}



\begin{exercice}Dichotomie récursive.\\
Soit $f$ une fonction qui s'annule sur l'intervalle $[a,b]$. L'objectif est de calculer une valeur approchée de $c$ tel que $f(c)=0$. On procède par dichotomie, en divisant à chaque étape l'intervalle de travail en deux. On note $n$ le nombre d'étapes effectuées (ce nombre $n$ décidera donc de la précision de la réponse).\\
La fonction \verb?dicho? prend comme argument $f,a,b$ et $n$ et renvoie une valeur approchée de $c$.
\begin{enumerate}
\item Donnez un critère simple pour vérifier si deux nombres $x$ et $y$ sont de signes opposés.
\begin{solution}
$xy<0$.
\end{solution}
\item \label{Q1} Prenez une feuille et un stylo. A la manière de l'exercice \ref{exp rapide}, écrivez un algorithme qui décrit les étapes de la dichotomie en version récursive. Il démarre par : \verb? dicho(f,a,b,n)?.
\begin{solution}
\verb?dicho(f,a,b,n)?\\
On pose : $c=\dfrac{a+b}{2}$.
\begin{itemize}
\item Si $n=0$, renvoyer $c$.
\item Si $f(a)f(c)$ est négatif, renvoyer \verb?dicho(f,a,c,n-1)?.
\item Sinon, renvoyer \verb?dicho(f,c,b,n-1)?.
\end{itemize}
\end{solution}
\item Tant que le programme sur feuille n'est pas correct, revenez à l'étape \ref{Q1}.  \\
Implémentez l'algorithme précédent en langage python.
\end{enumerate}
\end{exercice}
\begin{solution}~\\
\vspace{-0.7cm}
\begin{minted}{python}
def dicho(f,a,b,n):
    c=(a+b)/2.
    if n==0:
        return(c)
    else:
        if f(a)*f(c)<0:
            return(dicho(f,a,c,n-1))
        else:
            return(dicho(f,c,b,n-1))
\end{minted}
\end{solution}






\begin{exercice}Enumeration des sous-listes.\\
Soit $L$ une liste. L'objectif de l'exercice est d'écrire un programme qui renvoie toutes les sous-listes de $L$.\\
Par exemple, pour $L=[1,2,3]$, ses sous-listes sont : 
\[[1, 2, 3], [2, 3], [1, 3], [3], [1, 2], [2], [1], []\]
\begin{enumerate}
\item Ecrivez une fonction \verb?SousListe? qui prend comme argument une liste et renvoie la liste de ses sous-listes.
\begin{solution}~\\
\vspace{-0.7cm}
\begin{minted}{python}        
def SousListe(L):
    if len(L)==0:
        return([[]])
    DernierTerme=L[len(L)-1]
    AncienneSousListe=SousListe(L[0:len(L)-1])
    NouvelleSousListe=[]
    for i in AncienneSousListe:
        NouvelleSousListe.append(i+[DernierTerme])      
    return(NouvelleSousListe+AncienneSousListe) 
\end{minted}
\end{solution}
\item En maths, la fonction construit l'ensemble $\mathcal{P}(L)$. Vous savez qu'il est de cardinal $2^\text{Card(L)}$. Vérifiez la longueur de \verb?SousListe(L)?.
\begin{solution}
\verb?len(SousListe(L))==2**len(L)?
\end{solution}
\end{enumerate}
\end{exercice}
      
        






%\begin{exercice}
%Fonctions récursives : Enumeration des permutations
%\end{exercice}
%
%\begin{solution}~\\
%\vspace{-0.7cm}
%\begin{minted}{python}
%def Permute(string):
%    if len(string) == 0:
%        return ['']
%    prevList = Permute(string[1:len(string)])
%    nextList = []
%    for i in range(0,len(prevList)):
%        for j in range(0,len(string)):
%            newString = prevList[i][0:j]+string[0]+prevList[i][j:len(string)-1]
%            nextList.append(newString)
%    return nextList
%\end{minted}
%\end{solution}



\begin{exercice}Rendu de monnaie récursif.\\
La liste \verb?monnaie_euro=[50,20,10,5,2,1]? contient l'ensemble des billets/pièces (dont le montant est un entier) du système monétaire européen dans l'ordre décroissant. Etant donné une valeur \verb?restant_du?, on veut déterminer la liste des pièces nécessaires pour atteindre ce montant, obtenue par un algorithme glouton.\\
Proposez une fonction récursive \verb?rendu? qui prend comme argument \verb?restant_du? et \verb?monnaie_a_rendre? et renvoie la liste demandée.\\
Par exemple : \verb?rendu(216,[0,0,0,0,0,0])? doit renvoyer $[4,0,1,1,0,1]$.
\end{exercice}

\begin{solution}~\\
\vspace{-0.7cm}
\begin{minted}{python}
monnaie_euro=[50,20,10,5,2,1]

def rendu(m,L):
    if m==0:
        return(L)
    else:
        i=0
        while i<len(monnaie_euro) and m<monnaie_euro[i]:
            i=i+1   
        L[i]=L[i]+1
        m=m-monnaie_euro[i]
        return(rendu(m,L))
\end{minted}
\end{solution}


\end{document}


















