\newcommand{\nom}{Porte conteneur}
\newcommand{\sequence}{03}
\newcommand{\num}{04}
\newcommand{\type}{TD}
\newcommand{\descrip}{Résolution d'un problème en utilisant des méthodes algorithmiques}
\newcommand{\competences}{Alt-C3: Concevoir un algorithme répondant à un problème précisément posé}
\documentclass[10pt,a4paper]{article}
  \usepackage[french]{babel}
  \usepackage[utf8]{inputenc}
  \usepackage[T1]{fontenc}
  \usepackage{xcolor}
  \usepackage[]{graphicx}
  \usepackage{makeidx}
  \usepackage{textcomp}
  \usepackage{amsmath}
  \usepackage{amssymb}
  \usepackage{stmaryrd}
  \usepackage{fancyhdr}
  \usepackage{lettrine}
  \usepackage{calc}
  \usepackage{boxedminipage}
  \usepackage[french,onelanguage, boxruled,linesnumbered]{algorithm2e}
  \usepackage[colorlinks=false,pdftex]{hyperref}
  \usepackage{minted}
  \usepackage{url}
  \usepackage[locale=FR]{siunitx}
  \usepackage{multicol}
  \usepackage{tikz}
  \makeindex

  %\graphicspath{{../Images/}}

%  \renewcommand\listingscaption{Programme}

  %\renewcommand{\thechapter}{\Alph{chapter}}
  \renewcommand{\thesection}{\Roman{section}}
  %\newcommand{\inter}{\vspace{0.5cm}%
  %\noindent }
  %\newcommand{\unite}{\ \textrm}
  \newcommand{\ud}{\mathrm{d}}
  \newcommand{\vect}{\overrightarrow}
  %\newcommand{\ch}{\mathrm{ch}} % cosinus hyperbolique
  %\newcommand{\sh}{\mathrm{sh}} % sinus hyperbolique

  \textwidth 160mm
  \textheight 250mm
  \hoffset=-1.70cm
  \voffset=-1.5cm
  \parindent=0cm

  \pagestyle{fancy}
  \fancyhead[L]{\bfseries {\large PTSI -- Dorian}}
  \fancyhead[C]{\bfseries{{\type} \no \numero}}
  \fancyhead[R]{\bfseries{\large Informatique}}
  \fancyfoot[C]{\thepage}
  \fancyfoot[L]{\footnotesize R. Costadoat, C. Darreye}
  \fancyfoot[R]{\small \today}
  
  \definecolor{bg}{rgb}{0.9,0.9,0.9}
  
  
  % macro Juliette
  
\usepackage{comment}   
\usepackage{amsthm}  
\theoremstyle{definition}
\newtheorem{exercice}{Exercice}
\newtheorem*{rappel}{Rappel}
\newtheorem*{remark}{Remarque}
\newtheorem*{defn}{Définition}
\newtheorem*{ppe}{Propriété}
\newtheorem{solution}{Solution}

\newcounter{num_quest} \setcounter{num_quest}{0}
\newcounter{num_rep} \setcounter{num_rep}{0}
\newcounter{num_cor} \setcounter{num_cor}{0}

\newcommand{\question}[1]{\refstepcounter{num_quest}\par
~\ \\ \parbox[t][][t]{0.15\linewidth}{\textbf{Question \arabic{num_quest}}}\parbox[t][][t]{0.85\linewidth}{#1\label{q\the\value{num_quest}}}\par
~\ \\}

\newcommand{\reponse}[4][1]
{\noindent
\rule{\linewidth}{.5pt}\\
\textbf{Question\ifthenelse{#1>1}{s}{} \multido{}{#1}{%
\refstepcounter{num_rep}\ref{q\the\value{num_rep}} }:} ~\ \\
\ifdef{\public}{#3 ~\ \\ \feuilleDR{#2}}{#4}
}

\newcommand{\cor}
{\refstepcounter{num_cor}
\noindent
\rule{\linewidth}{.5pt}
\textbf{Question \arabic{num_cor}:} \\
}

%%\usepackage[a4paper]{geometry}
%\geometry{margin={1cm,1.2cm}}
%\usepackage[francais]{babel}
%\usepackage{nopageno} %pas de numérotation de page
%\pagestyle{plain} %numérotation en bas de page, pas d'entête
%\usepackage{hyperref}
%\usepackage[latin1]{inputenc}

%%%%%%%%%%%%%%%%%%%%%%%%%%%%%%%%%%%%%%%%%%%%%%%%%%%%%%%%%%%%%%%%%%%%%%%%%%%%%%%%%%%%%

\usepackage{amsthm}
\usepackage{amscd}
%\usepackage{mathrsfs}
%\usepackage{amsfonts}
%\usepackage[T1]{fontenc}
%\usepackage{theorem}
\usepackage{lscape}
\usepackage{variations}  % pour faire des tableaux de variations
\usepackage{dsfont}
\usepackage{fancyvrb} % pour mettre Verbatim dans une box

% Pour les figures
\usepackage{subfig}
%\usepackage{calc} % Pour pouvoir donner des formules dans les d�finitions de longueur
%\usepackage{graphicx} % Pour inclure des graphiques 
% Attention : pour inclure des .jpg comme dans l'exemple (ou des .png ou .pdf)
% il faut compiler directement en pdf (commande pdflatex).
% Pour inclure des .eps, il faut compiler avec latex + dvips + ps2pdf.
\usepackage{psfrag}
%\usepackage{color}

%%%%%%%%%%%%%%%%%%%%%%%%%%%%%%%%%%%%%%%%%%%%%%%%%%%%%%%%%%%%%%%%%%%%%%%%%%%%%%%%%%%%%

\theoremstyle{definition}
\newtheorem*{thm}{Théorème}
%\theorembodyfont{\rmfamily}
\newtheorem*{defn}{Définition}
\newtheorem{exercice}{Exercice}
\newtheorem*{problem}{Problème}
\newtheorem*{prop}{Proposition}
\newtheorem*{corollaire}{Corollaire}
\newtheorem*{lemme}{Lemme}
\newtheorem*{remark}{Remarque}
\newtheorem*{notation}{Notation}
\newtheorem*{ex}{Exemple}
\newtheorem*{ppe}{Propriété}
\newtheorem*{meth}{Méthode}
\newtheorem*{rappel}{Rappel}
\newtheorem*{voca}{Vocabulaire}
\setlength{\columnseprule}{0.5pt}


%%%%%%%%%%%%%%%%%%%%%%%%%%%%%%%%%%%%%%%%%%%%%%%%%%%%%%%%%%%%%%%%%%%%%%%%%%%%%%%%%%%%%

\newcommand{\bi}{\bigskip}
\newcommand{\dsp}{\displaystyle}
\newcommand{\noi}{\noindent}
\newcommand{\ov}{\overline}
\newcommand{\dsum}{\displaystyle \sum}
\newcommand{\dprod}{\displaystyle \prod}
\newcommand{\dint}{\displaystyle \int}
\newcommand{\dlim}{\displaystyle \lim}

%%%%%%%%%%%%%%%%%%%%%%%%%%%%%%%%%%%%%%%%%%%%%%%%%%%%%%%%%%%%%%%%%%%%%%%%%%%%%%%%%%%%%


%\newcommand{\pgcd}{\mathrm{pgcd}} % pgcd
%\providecommand{\norm}[1]{\lVert#1\rVert} % norme
%\DeclareMathOperator{\Tan}{Tan}  % espace tangent


\newcommand{\N}{\mathbb{N}}
\newcommand{\Z}{\mathbb{Z}}
\newcommand{\Q}{\mathbb{Q}}
\newcommand{\R}{\mathbb{R}}
\newcommand{\C}{\mathbb{C}}
\newcommand{\K}{\mathbb{K}}
\newcommand{\U}{\mathbb{U}}
\newcommand{\Tr}{\text{Tr}\,}
\newcommand{\pg}{\geqslant}
\newcommand{\pp}{\leqslant}
\newcommand{\bul}{\item[$\bullet$]}
\newcommand{\card}{\text{Card}}
\newcommand{\re}{\text{Re}\;}
\newcommand{\im}{\text{Im}\;}
\newcommand{\Ker}{\text{Ker}\;}
\newcommand{\Vect}{\text{Vect}\;}
\newcommand{\rg}{\text{rg}\;}
\newcommand{\TT}{{}^t\!}
%\newcommand{\sh}{\text{sh}}
%\newcommand{\ch}{\text{ch}}
\newcommand{\Mat}{\text{Mat}}
\usepackage{textcomp}



%%%%%%%%%%%%%%%%%%%%%%%%%%%%%%%%%%%%%%%%%%%%%%%%%%%%%%%%%%%%%%%%%%%%%%%%%%%%%%%%%%%%%%%%%%%%%%%%%%%%%%%%%%%%%%%%%%%%%%%%%%%




\begin{document}

\begin{center}
{\Large\bf TP \no {\numero} -- \descrip}
\end{center}
 
Tous les fichiers utiles sont placés sur le serveur de base de données. On se connecte au serveur de base de données comme expliqué au TP précédent.

% \section{Alimentation et exploitation d'une base de données}
% 
% On fournit dans le fichier \verb|cafes.csv| la liste\footnote{Récupérée sur http://opendata.paris.fr} des noms, des adresses et des coordonnées géographiques des endroits de Paris où le café était servi à 1 euro au 15 mai 2014.
% 
% \begin{enumerate}
%  \item Créer une nouvelle base de données dans un fichier \verb|cafes.sqlite|.
%  
%  \item Créer la table dont le schéma relationnel est :\\
%  \verb|cafes (nom:text, adresse:text, codepostal:int, latitude:float, longitude:float)|.
%  
%  \item En utilisant la fonction « Import » du menu déroulant « Base de données », importez les données du fichier \verb|cafes.csv|.
%  
%  \item Combien de cafés sont situés dans le XI\up{e} arrondissement ?
%  
%  \item Les coordonnées géographiques du lycée Dorian sont en degrés décimaux : (latitude) 48,854563° et (longitude) 2,392678°. Pour des lieux tous situés dans une zone géographique aussi petite que Paris, on peut approximer que la distance entre deux points est directement proportionnelle à la racine carrée de la somme de la différence des longitudes au carré et de la différence des latitudes au carré : $\sqrt{(lat1-lat2)^2  + (long1-long2)^2}$. Quel est le café à un euro le plus proche du lycée ?
% \end{enumerate}

\section{Prénoms}

La base de données \verb|prenoms| contient une table \verb|prenoms| contenant les prénoms enregistrés à l'état civil de Paris depuis 2004 jusqu'en 2013.

\begin{enumerate}
\item Écrire sur papier le schéma relationnel de la table de cette base.
\item Pour chacune des requêtes SQL suivantes, donner la traduction « en français » et vérifier la vraisemblance du résultat depuis MySQL Workbench\footnote{Voir Annexe pour les fonctions d'agrégation \texttt{COUNT} et \texttt{SUM} et la commandes \texttt{GROUP BY}.}.

\begin{enumerate}
\item  \verb|SELECT DISTINCT prenom FROM prenoms ;|
\item  \verb|SELECT SUM(nombre) FROM prenoms ;|
\item  \verb|SELECT COUNT(DISTINCT prenom) FROM prenoms ;|
\item  \verb|SELECT annee, SUM(nombre) FROM prenoms GROUP BY annee ;|
\end{enumerate}

\item Combien de fois votre prénom a-t-il été donné à Paris pendant les années concernées ? On donnera le résultat trié par années croissantes.

\item Même question avec le prénom des professeurs présents. 

\item Donner, pour chaque année, le nombre de prénoms différents qui ont été enregistrés.

\item 
    \begin{enumerate}
    
    \item Quels prénoms ont été donnés exactement 100 fois sur au moins une année ?
    
    \item Quels prénoms ont été donnés exactement 100 fois sur la période complète ?
                                                                                                                                                                                      \end{enumerate}

\item 
    \begin{enumerate}
    \item Que fait la requête suivante ?
    
    \verb|SELECT MAX(nombre) as maxi, annee FROM prenoms GROUP BY annee|
    \end{enumerate}
    
    On peut donner un nom (ou alias) à la table qui est créée par la requête précédente. Ainsi :
    
    \verb|(SELECT MAX(nombre) as maxi, annee FROM prenoms GROUP BY annee) liste_max|
    
    permet de nommer \verb|liste_max| la table issue de la requête entre parenthèses. En procédant ainsi on peut l'utiliser dans une requête plus complexe.    
        
    Par exemple, à l'aide d'une requête en jointure utilisant les tables \verb|prenoms| et \verb|liste_max|, on peut déterminer le prénom qui a été le plus donné pour chaque année (en affichant le prénom, l'année et le nombre) :
    
    \verb|SELECT P.annee, P.prenom, P.nombre FROM|\\
    \verb|prenoms P JOIN (SELECT MAX(nombre) as max, annee FROM prenoms GROUP BY annee) liste_max|\\
    \verb|ON liste_max.annee=P.annee|\\
    \verb|WHERE liste_max.max=P.nombre|\\
    \verb|ORDER BY P.annee;|\\
    
    \begin{enumerate}
    \setcounter{enumii}{1}
    \item En vous inspirant de l'exemple précédent, déterminer le prénom féminin qui a été le plus donné pour chaque année (donner le prénom, l'année et le nombre) ?
    
    \item Analyser et comprendre ce que fait la requête suivante :
    
%       \verb|SELECT prenom, MAX(somme)|\\
%       \verb|FROM (|\\
%       \verb|SELECT prenom, SUM(nombre) AS somme|\\
%       \verb|FROM prenoms GROUP BY prenom|\\
%       \verb|)|
    
    \verb|SELECT prenom, somme|\\
    \verb|FROM (SELECT prenom, SUM(nombre) AS somme FROM prenoms GROUP BY prenom) liste_somme|\\
    \verb|WHERE liste_somme.somme=(|\\
    \verb|SELECT MAX(somme) FROM (SELECT prenom, SUM(nombre) AS somme|\\
    \verb|FROM prenoms GROUP BY prenom) liste_somme|\\
    \verb|)|
 
    Traduire le but de cette requête en français.
                                                                                                                                                       
    \end{enumerate}
    

\end{enumerate}

\newpage
\section{Notes de colles}

La base de données \verb|colles| contient trois tables (\verb|colles|, \verb|eleves|, \verb|profs|) décrivant les colles virtuelles données par des agrégés de la promotion 1930 à des agrégés de la promotion 1950.
\begin{enumerate}

\subsection{Structure de la base de données}

\item Écrire le schéma relationnel des différentes tables. Comprendre le lien entre les différentes tables.

\subsection{Requêtes élémentaires}

\item Déterminer la liste des professeurs.
\item Déterminer celle des élèves.
\item Déterminer le nombre de « 20 » qui ont été attribués, ainsi que le nombre de notes majorant 6.

\subsection{Requêtes avec jointures}
\item Déterminer les notes de Jacques-Louis Lions (triées selon les semaines croissantes).
\item Refaire la même chose avec cette fois le nom des colleurs associés.
\item Déterminer les quadruplets (élève, prof, note, semaine) pour toutes les colles où la note était supérieure
ou égale à 19.

\subsection{Utilisation d'agrégats}

\item Déterminer la moyenne des notes de colle de Jacques-Louis Lions.

\item Déterminer la liste des couples (élève, moyenne).

\item Analyser et essayer de comprendre ce que fait la requête suivante :

\verb|SELECT eleves.nom, COUNT(*) AS nombre_de_plantages|\\
\verb|FROM eleves JOIN colles ON ide=eleve|\\
\verb|WHERE note < 10|\\
\verb|GROUP BY eleves.nom|\\
\verb|HAVING nombre_de_plantages > 10|

Traduire le but de cette requête en français.

\item Parmi tous les élèves, afficher ceux ayant eu au moins 6 notes strictement supérieures à 18.

\item Déterminer le nom du colleur qui tend à donner les meilleurs notes en moyenne.

\item 
    \begin{enumerate}
    \item Déterminer pour chaque colleur la moyenne des notes données et l'écart-type. On utilisera cette expression de la variance : $\displaystyle \sigma =\sqrt{\frac{1}{n}\,\left(\sum_{i=1}^n x_i^2\right) - \bar{m}^2}$ où $\bar{m}$ est la moyenne.
    
    \item Quel est le colleur qui donne le plus souvent les mêmes notes ?
    \end{enumerate}

\item Écrire (sans tricher) une requête permettant de calculer la moyenne des moyennes des élèves.

\end{enumerate}

 \newpage
 \section{Annexe}
\subsection{Fonction d'agrégation}

Dans tous les exemples on utilise la table nommée \verb|donnees|, ci -dessous :
\begin{center}
\begin{tabular}{|c|c|c|}\hline
attribut1 & attribut2 & attribut3\\\hline
0 & 0 & 0\\\hline
1 & 1 & 20\\\hline
2 & 2 & 40\\\hline
3 & 2 & 60\\\hline
\end{tabular}
\end{center}

\subsubsection{Définitions}
On appelle \textit{fonction d'agrégation}, une fonction qui s'applique sur un groupe de valeurs (appelé \textit{agrégat}). Les différentes fonctions d'agrégation sont :
\begin{itemize}
  \item \verb|COUNT(agregat)| : permet de compter le nombre d'éléments de l'agrégat ;
  \item \verb|MAX(agregat)|et \verb|MIN(agregat)| : permettent de déterminer le maximum, le minimum de l'agrégat ;
  \item \verb|AVG(agregat)| : permet de déterminer la moyenne de l'agrégat ;
  \item \verb|SUM(agregat)| : permet de déterminer la somme de l'agrégat ;
\end{itemize}

Les agrégats sont la plupart du temps les valeurs d'un attribut de la table (c'est-à-dire une colonne), sauf pour \verb|COUNT()| qui peut s'appliquer à la totalité de la table (\verb|COUNT(*)|).
 
Exemple : \verb|SELECT AVG(attribut1) FROM donnees| renvoie la moyenne de toutes les valeurs de l'attribut 1.
 
 \subsubsection{GROUP BY}
 
Il est possible d'appliquer les fonctions sur des sous-agrégats plus petits qu'un attribut en groupant certaines valeurs sur un critère donné grâce à \verb|GROUP BY critere|. Si le critère est un attribut (une colonne), alors les sous-agrégats sont formés par valeurs distinctes de cet attribut. Si le critère est une condition, alors les sous-agrégats sont (dans l'ordre booléen) les lignes qui ne respectent pas la condition et celles qui la respectent.

\verb|SELECT AVG(attribut3) FROM donnees GROUP BY attribut2=0| renvoie : 

\begin{center}
\begin{tabular}{|c|}\hline
AVG(attribut3)\\\hline
40 \\\hline
0 \\\hline
\end{tabular}
\end{center}

\verb|SELECT AVG(attribut3) FROM donnees GROUP BY attribut2| renvoie : 

\begin{center}
\begin{tabular}{|c|}\hline
AVG(attribut3)\\\hline
0 \\\hline
20 \\\hline
50 \\\hline
\end{tabular}
\end{center}

Pour mieux expliciter le résultat, on peut aussi demander d'afficher l'attribut2, qui sert à sélectionner les sous-agrégats.

\verb|SELECT attribut2, AVG(attribut3) FROM donnees GROUP BY attribut2| renvoie : 

\begin{center}
\begin{tabular}{|c|c|}\hline
attribut2 & AVG(attribut3)\\\hline
0 & 0 \\\hline
1 & 20 \\\hline
2 & 50 \\\hline
\end{tabular}
\end{center}

\subsubsection{Sélection en amont et en aval}

On peut restreindre les agrégats sur lesquels s'appliquent les fonctions d'agrégation, par la clause \verb|WHERE critere| (sélection en amont, c'est-à-dire avant de réaliser la fonction d'agrégation) ou bien restreindre ce qu'affiche la fonction d'agrégation par la clause \verb|HAVING critere| (sélection en aval, c'est-à-dire après avoir réalisé la fonction d'agrégation). \verb|HAVING| doit toujours être situé après un \verb|GROUP BY|.

\paragraph{Sélection en amont}

La sélection s'effectue avec la clause \verb|WHERE| sur les lignes initiales de la table avant de réaliser l'agrégation.

\verb|SELECT AVG(attribut3) FROM donnees WHERE attribut2=1| renvoie 

\begin{center}
\begin{tabular}{|c|c|}\hline
AVG(attribut3)\\\hline
20 \\\hline
\end{tabular}
\end{center}

\paragraph{Sélection en aval}

La sélection s'effectue avec la clause \verb|HAVING| sur les lignes qui restent après la réalisation de l'agrégation, ce qui n'a de sens que si on a effectué au préalable un regroupement grâce à \verb|GROUP BY|.

\verb|SELECT AVG(attribut3) FROM donnees GROUP BY attribut2 HAVING attribut2=2| renvoie 

\begin{center}
\begin{tabular}{|c|c|}\hline
AVG(attribut3)\\\hline
50 \\\hline
\end{tabular}
\end{center}

\subsection{ORDER BY}

La clause \verb|ORDER BY attribut| permet de trier les résultats de la requête selon un attribut donné. Par défaut le tri est ascendant. L'option \verb|DESC| permet de trier par ordre descendant.

\verb|ORDER BY nom| trie la requête par ordre ascendant des valeurs de l'attribut \verb|nom|

\verb|ORDER BY annee DESC| trie la requête par ordre descendant des valeurs de l'attribut \verb|annee|

Elle est toujours située à la fin de la requête.



\end{document}
