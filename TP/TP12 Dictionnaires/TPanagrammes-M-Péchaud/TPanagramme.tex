\documentclass[10pt]{article}
\usepackage[OT1]{fontenc}
\usepackage{mickaelpechaud_cours}
\usepackage{fancyvrb}
\usepackage{caption}
\usepackage{subfig}

\usepackage[a4paper,pdftex,margin=.7in]{geometry}
\usepackage[
    type={CC},
    modifier={by-nc},
    version={4.0},
    ]{doclicense}
    \usepackage{fancyhdr}

\makeatletter
\fancypagestyle{logo}{%
  \fancyhf{}%
  \renewcommand{\headrulewidth}{0pt}%
  \fancyfoot[LE,RO]{%
    \begin{minipage}{0.5\textwidth}%
      \def\doclicense@imagewidth{2cm}%
      \tiny
      \doclicenseThis
    \end{minipage}%
  }%
}
\fancypagestyle{plain}{%
  \fancyhf{}%
  \renewcommand{\headrulewidth}{0pt}%
  \fancyfoot[LE,RO]{%
    \begin{minipage}{0.5\textwidth}%
      \def\doclicense@imagewidth{2cm}%
      \tiny
      \doclicenseThis
    \end{minipage}%
  }%
}
\makeatother
\pagestyle{logo}

\VerbatimFootnotes

\title{\textbf{Autour des anagrammes}} \date{Mars 2021}
\author{Mickaël Péchaud (\href{mailto:mickaelpechaud@protonmail.com}{mickaelpechaud@protonmail.com})}
\begin{document}
%\doclicenseThis
\renewcommand{\labelitemi}{\textbullet}

\maketitle



\begin{itemize}
\item Documentez vos fonctions.
\item Testez au fur et à mesure les fonctions que vous écrivez!
\end{itemize}

Un fichier \verb+anagrammes.py+ à compléter est fourni.

\section{Introduction}

Une \emph{anagramme} d'un mot $m$ est un mot obtenu par permutation des
lettres de $m$. Dans une anagramme, on ignore la casse
(majuscule/minuscule), les symboles diactritiques (accents et
cédilles), tirets et autres apostrophes. Ainsi, les mots suivants sont des anagrammes:

\begin{itemize}
\item «nectar» et «carnet»
\item «sortie» et «rôties»
\item «niche» et «Chine»
\end{itemize}

\smallskip

L'objectif de ce TP est d'écrire différentes fonctions permettant de
trouver rapidement des anagrammes de mots (ou de listes de mots à la
fin du TP).

\smallskip

Le fichier \verb+listemotsfrancais.txt+ fourni est un fichier texte
contenant 629\;569 mots français\footnote{Fichier engendré par à
  partir du dictionnaire français d'\emph{aspell}
  (\url{http://aspell.net/})}.

\smallskip

Vous trouverez en début du fichier \verb+anagrammes.py+

\begin{itemize}
\item une fonction \verb+enleve_diacritiques_majuscules+, qui renvoie la chaîne de caractères en minuscule et sans symboles diacritiques ni apostrophes ni tiret correspondant à la chaîne passée en argument,
\item une fonction \verb+charger_liste_mots+, qui permet de charger le fichier \verb+listemotsfrancais.txt+ sous forme d'une liste de chaînes de caractères. 
\end{itemize}

Ces fonctions seront utiles à partir de la partie~\ref{tousanagrammes}.

\medskip

Dans la suite, sauf précision contraire, les mots seront considérés comme étant en minuscule et sans symbole diacritique.

\section{Conversions}

\begin{enumerate}
\item Quelle est en pratique la principale différence entre une chaîne
  de caractères (i.e. un objet de type \verb+str+) et une liste de
  caractères? Compléter les fonctions de conversions entre ces deux
  types.
\end{enumerate}

\begin{correction}
  Les caractères d'une \verb+str+ ne sont pas modifiables
  contrairement à ceux d'une liste de caractères. La liste de
  caractères est donc plus souple d'utilisation, mais ne pourra par
  contre pas être utilisée comme clé d'un dictionnaire.
\end{correction}

Dans la suite, prêter attention au type des paramètres et valeur de retour des fonctions! \textbf{Sauf mention contraire, «mot» désignera une chaîne de caratères.}

\section{Tester si deux mots sont anagrammes l'un de l'autre}

Dans toute cette partie, on supposera que les deux mots testés sont
de même longueur.

\subsection*{Une bien mauvaise méthode}

On commence par étudier une très «mauvaise» méthode pour savoir si
deux mots $m_1$ et $m_2$ sont anagrammes l'un de l'autre. On va
engendrer toutes les permutations des lettres de $m_1$, et vérifier si
l'une d'elle est égale à $m_2$.

\begin{enumerate}[resume]
\item Écrire une fonction \emph{récursive} \verb+permutation_aux+ qui
  prend en argument une liste $l$ de longueur $n$ et un indice
  $i\in\zen{n}$ et renvoie la liste des permutations de $l$ laissant
  fixes les éléments d'indices strictement inférieurs à $i$. Par
  exemple:
\begin{verbatim}
>>> permutations_aux(['a','b','c','d'], 1)

[['a', 'b', 'c', 'd'],
 ['a', 'b', 'd', 'c'],
 ['a', 'c', 'b', 'd'],
 ['a', 'c', 'd', 'b'],
 ['a', 'd', 'c', 'b'],
 ['a', 'd', 'b', 'c']]
\end{verbatim}
  On ne se souciera pas des éventuels doublons dans le résultat renvoyé. 
\item En déduire une fonction \verb+permutation+ qui prend en argument une liste $l$ et renvoie la liste de ses permutations.
\item En déduire une fonction \verb+sont_anagrammes1+, qui prend en argument deux mots et teste s'ils sont anagrammes l'un de l'autre. 
\item En quoi cette fonction est-elle «mauvaise»?

  \begin{correction}
    Dans le pire des cas, toutes les permutations du premier mot vont
    être engendrées. On effectuera donc au moins $n!$ comparaisons de
    listes, ce qui va devenir prohibitif pour des mots de plus de
    quelques lettres\dots
  \end{correction}
  
\end{enumerate}

Nous allons donc chercher d'autres façons de procéder.

\begin{enumerate}[resume]
\item On propose l'algorithme suivant pour tester si deux mots sont
  anagrammes: étant donnés deux mots $m_1$ et $m_2$, on cherche si
  chaque lettre de $m_1$ apparaît dans $m_2$, et si réciproquement
  chaque lettre de $m_2$ apparaît dans $m_1$. Cet algorithme est-il correct?

  \begin{correction}
    Non. Il renvoie par exemple vrai pour \verb+'aba'+ et
    \verb+'aab'+, qui ne sont pas anagrammes l'un de
    l'autre. L'algorithme teste en fait si les deux mots ont le même
    ensemble de lettres, alors qu'il faudrait tester s'ils ont le même
    \emph{multiensemble} de lettres (un multiensemble étant un
    ensemble dans lequel le nombre d'apparition des éléments compte).
  \end{correction}
  
\item Avant de passer à la suite, réfléchir à une autre stratégie pour
  résoudre le problème posé.
\end{enumerate}


\subsection*{Une seconde méthode}

\begin{enumerate}[resume]
\item Écrire une fonction \verb+mot_to_dico+, qui prend en argument un mot $m$ et renvoie un dictionnaire
  \begin{itemize}
  \item dont les clés sont les lettres de $m$
  \item et la valeur correspondant à une clé $k$ est le nombre d'occurrences de $k$ dans $m$. 
  \end{itemize}
\item En déduire une fonction \verb+sont_anagrammes2+ (on pourra
  utiliser \verb+==+ pour comparer l'égalité de deux dictionnaires,
  avec une complexité temporelle $O(n)$ où $n$ est la taille commune
  des dictionnaires, et $O(1)$ s'ils sont de tailles différentes). Quelle est sa complexité?

  \begin{correction}
    L'insertion ou la modification dans un dictionnaire s'effectuant
    en temps constant, \verb+mot_to_dico+ est de complexité $O(n)$,
    ainsi que \verb+sont_anagrammes2+.
  \end{correction}
  
\end{enumerate}

\subsection*{Une troisième méthode}

\begin{enumerate}[resume]
\item Écrire une fonction \verb+tri_liste+ qui trie la liste de
  caractères $l$ passée en argument, en utilisant votre algorithme de
  tri préféré. On indique que \verb+<+ permet de comparer des
  caractères.
\item En déduire la fonction \verb+tri_mot+, qui prend en argument un
  mot, et renvoie la version triée de ce mot.
\item En déduire une fonction \verb+sont_anagrammes3+. Quelle est sa complexité?

  \begin{correction}
    La comparaisons de chaînes et les conversions chaînes $\lra$ listes s'effectue en temps $O(n)$. Le facteur limitant est donc l'algorithme de tri -- qui doit être de complexité $O(n\log(n))$ ou $O(n^2)$ selon votre choix d'il y a deux questions. 
  \end{correction}
\item Effectuer une rapide comparaison empirique de temps d'exécution des trois versions de \verb+sont_anagrammes+. Vous pourrez utiliser la fonction \verb+process_time()+ du module \verb+time+.

  \begin{correction}
    Sur le corrigé du TP -- qui utilise traitreusement \verb+sort+ pour trier une liste -- les deux dernières méthodes donnent des temps d'exécution du même ordre de grandeur. 
  \end{correction}
  
\end{enumerate}

\section{Recherche de toutes les anagrammes d'un mot donné}\label{tousanagrammes}

\emph{Dans cette partie, les mots peuvent tous avoir des majuscules et des symboles diacritiques}. La notion d'anagramme s'entend cependant toujours comme dans l'introduction. 

\begin{enumerate}[resume]
\item En vous appuyant sur l'une des fonctions \verb+sont_anagrammes+
  ci-dessus, écrire une fonction \verb+liste_anagrammes1+, qui prend
  en argument un mot et une liste de mots, et renvoie la liste des
  anagrammes du mot dans la liste. Par exemple
  \begin{verbatim}
>>> liste_anagrammes1('préparation', liste_mots)
['apparieront', 'préparation', 'appaireront']
\end{verbatim}
  Quelle est sa complexité? Constater empiriquement son temps d'exécution sur quelques exemples. 

  \begin{correction}
    Dans la version du corrigé, \verb+sont_anagramme2+, de complexité
    $O(n)$ est utilisé.

    Si l'on note $N$ la taille de la liste de mots, on obtient donc
    une complexité $O(nN)$.
  \end{correction}  
\end{enumerate}

On cherche maintenant à \emph{pré-traiter} la liste de mots pour
obtenir un algorithme plus efficace.

Pour cela, on va créer un dictionnaire à partir de la liste de mots
\begin{itemize}
\item dont les clés $k$ sont des chaînes de caractères triées (sans diacritiques ou majuscules)
\item dont la valeur correspondant à $k$ est la liste des mots qui sont anagrammes de $k$.
\end{itemize}

Voici à titre d'exemple un extrait du dictionnaire que l'on souhaite obtenir:

\begin{verbatim}
{
 ...
 'acehoppr': ['achopper', 'approche', 'approché', 'choppera'],
 'acehopps': ['achoppes', 'échoppas', 'achoppés'],
 'aacehinoppt': ['achoppaient'],
 ...
}
\end{verbatim}

\begin{enumerate}[resume]
\item Écrire la fonction \verb+liste_mots_to_dict+ correspondante, et générer le dictionnaire, qui servira dans toute la fin du TP.
\item En déduire une fonction \verb+liste_anagrammes2+, qui prend en
  argument un mot et un dictionnaire, et renvoie \emph{en temps
    constant en la taille du dictionnaire} la liste des anagrammes du mot.
\item Trouver l'ensemble de mots tous anagrammes les uns des autres le plus grand possible. 
\end{enumerate}

\section{Anagrammes avec plusieurs mots}

On veut maintenant faire des anagrammes de listes de mots: étant
donnés un ou plusieurs mots, on cherche toutes les listes de deux,
trois, quatre\dots mots tels que la concaténation de ces mots forme une
anagramme de la concaténation des mots de départ.

Par exemple, parmi les ensemble de deux ou trois mots possibles pour
«informatique», on trouve «quota+infirme», «if+romantique»,
«requin+à+motif» ou encore «tif+maroquiné» (et bien d'autres).

\subsection*{Fonctions préliminaires}

On dit que $m$ est un \emph{sous-mot} de $m'$ si toute lettre de $m$
apparait également dans $m'$ avec un nombre d'occurrences au moins
égal (indépendamment de l'ordre des lettres).

\begin{enumerate}[resume]
\item En utilisant \verb+mot_to_dico+, écrire une fonction
  \verb+sous_mot1+ qui prend en argument deux mots \verb+str1+ et
  \verb+str2+, et teste si le premier est un sous-mot du second.
\item On considère maintenant le cas où \verb+str1+ et \verb+str2+
  sont des mots \emph{triés}. Notons $a$ la première lettre de
  \verb+str1+, et $p(a)$ la position de $a$ dans \verb+str2+ si elle y
  apparaît. En remarquant que \verb+str1+ est un sous-mot de
  \verb+str2+ si et seulement si la $a$ apparaît dans \verb+str2+ et que
  \verb+str1[1:]+ est un sous-mot de \verb-str2[p(a)+1:]-, écrire une
  fonction \verb+sous_mot2+ utilisant une fonction auxiliaire
  \emph{récursive} pour répondre à cette question.
\item Écrire une fonction \verb+soustrait+ qui prend en argument deux
  mots triés \verb+str1+ et \verb+str2+ tels que le premier soit un
  sous-mot du second, et renvoie le mot trié correspondant à
  \verb+str2+ auquel on a enlevé les lettres de \verb+str1+ (autant de
  fois que leur nombre d'occurrence dans \verb+str1+). On pourra
  utiliser au choix une stratégie itérative ou récursive.
\item Écrire une fonction \verb+tous_anagrammes_liste+ qui prend en
  argument un dictionnaire et une liste de mots triés
  $[m_1,\dots m_n]$, et renvoie la liste de toutes les listes de la forme
  $[m'_1,\dots m'_n]$, où pour tout $i\in\unn{n}$, $m'_i$ est une
  anagramme de $m_1$ appartenant à la liste de mots de départ. Par exemple
  \begin{verbatim}
>>> tous_anagrammes_liste(dictionnaire, ['aeiimnqstuu', 'fou', 'er'])
[['musiquaient', 'fou', 'Re'], ['musiquaient', 'fou', 'ré'], 
 ['musiquaient', 'ouf', 'Re'], ['musiquaient', 'ouf', 'ré']]
  \end{verbatim}
\end{enumerate}

\subsection*{Anagrammes avec plusieurs mots}

Nous pouvons maintenant écrire une fonction \emph{récursive}
\verb+anagramme_liste+ qui prend en argument:

\begin{itemize}
\item un dictionnaire \verb+dictionnaire+
\item une liste \verb+accumulateur+ de mots triés
\item un mot trié \verb+reste+
\item un entier \verb+n+ strictement positif
\end{itemize}

et qui \emph{affiche} (au format de votre choix) toutes les listes constituées de
\verb-len(accumulateur)+n- mots français telles que

\begin{itemize}
\item les \verb+len(accumulateur)+ premiers éléments soient chacun une anagramme du mot trié correspondant d'\verb+accumulateur+
\item la concaténation des \verb+n+ derniers mots soit une anagramme de \verb+reste+.
\end{itemize}

Par exemple:

\begin{verbatim}
>>> anagramme_liste(dictionnaire, [], tri_mot('informatiquepourtous'), 2)
['mouftai', 'pronostiqueur']
['pronostiqueur', 'mouftai']

>>> anagramme_liste(dictionnaire, [], tri_mot('anagramme'), 3)
['an', 'agréa', 'mm']
['an', 'égara', 'mm']
['an', 'ragea', 'mm']
['na', 'agréa', 'mm']
['na', 'égara', 'mm']
['na', 'ragea', 'mm']
...

>>> anagramme_liste(dictionnaire, ['aaegr'], 'anmm', 2)
['agréa', 'an', 'mm']
['agréa', 'na', 'mm']
['égara', 'an', 'mm']
['égara', 'na', 'mm']
['ragea', 'an', 'mm']
...
\end{verbatim}

\begin{enumerate}[resume]
\item Écrire la fonction \verb+anagramme_liste+. Indications:
  \begin{itemize}
  \item Que faire dans le cas de base $n=1$?
  \item Dans le cas $n\geq 2$, on parcourra les clés du dictionnaire à
    la recherche de sous-mots de \verb+reste+. Pour chaque sous-mot
    ainsi trouvé, on fera un appel récursif de la fonction avec un
    dernier paramètre \verb+n-1+, et d'autres paramètres bien choisis.
  \end{itemize}
\end{enumerate}

\medskip

L'exécution de cette fonction peut-être vue comme une exploration des
listes de mots par une méthode de \emph{backtracking} (ou \emph{retour
  sur trace}). Voir par exemple
\url{http://mpechaud.fr/scripts/parcours/index.html}.

\medskip

\subsubsection*{Pour aller plus loin}

\begin{itemize}
\item L'exécution de \verb+anagramme_liste+ affiche les $n!$
  permutations de chaque liste solution. Pour y remédier, modifier la
  fonction de façon à ce qu'elle n'affiche que les listes classées par
  ordre lexicographique croissant des mots triés (ce qui va au passage
  réduire la taille de l'espace exploré\dots).
\item Afin d'accélérer les calculs on pourrait commencer par enlever
  du dictionnaire tous les mots qui ne sont pas sous-mots du mot dont
  on cherche une anagramme au départ. Coder et expérimenter!
\end{itemize}

\appendix

\section*{Annexe}


Voici les fonctions et méthodes sur les listes, dictionnaires et deque
dont l'usage est autorisé -- ainsi que les complexités que vous
utiliserez pour les calculs de complexité (en fonction de la longueur
de la liste $n$). Tout autre fonction dont vous auriez besoin doit
être implémentée!

\subsection*{Fonctions et méthodes sur les listes}


\begin{tabular}{|c|c|c|}
  \hline
  Opération & Exemple & Complexité \\
  \hline 
  \hline 
  Création d'une liste vide & \verb+l=[]+ & $O(1)$\\
  \hline 
  Accès direct & \verb+l[0]+ & $O(1)$\\
  \hline
  Longueur & \verb+len(l)+ & $O(1)$\\
  \hline
  Concaténation & \verb-l1+l2- & $O(n1+n2)$\\
  \hline
  Ajout en fin de liste & \verb+l.append(1)+ & $O(1)$\\
  \hline
  Suppression en fin de liste & \verb+l.pop()+ & $O(1)$\\
  \hline
  Extraction de tranche & \verb+l[1:10]+ & $O(n)$, où $n$ est la longueur de la tranche.\\
  \hline
  Répétition & \verb+[0]*k+ & $O(n)$, où $n$ est la longueur de la liste créée. \\
  \hline
  Création par compréhension & \verb+[k**2 for k in range(n)]+ & $O(n)$ si l'expression est évaluée en temps constant\\
  \hline
  Copie & \verb+copy(l)+ & $O(n)$\\
  \hline
  Test d'égalité & \verb+l1 == l2+ & $O(1)$ si les longueurs sont différentes, $O(n)$ sinon\\
  \hline
\end{tabular}

\subsection*{Fonctions et méthodes sur les chaînes de caractères}


\begin{tabular}{|c|c|c|}
  \hline
  Opération & Exemple & Complexité \\
  \hline 
  \hline
  Création & \verb+s = 'Ma chaîne'+ & $O(n)$\\
  \hline
  Accès direct & \verb+s[0]+ & $O(1)$\\
  \hline
  Longueur & \verb+len(s)+ & $O(1)$\\
  \hline
  Concaténation & \verb-s1+s2- & $O(n1+n2)$\\
  \hline
  Extraction de tranche & \verb+s[1:10]+ & $O(n)$, où $n$ est la longueur de la tranche.\\
  \hline
\end{tabular}



\subsection*{Fonctions et méthodes sur les dictionnaires}


\begin{tabular}{|c|c|c|}
  \hline
  Opération & Exemple & Complexité \\
  \hline 
  \hline 
  Création & \verb+d = {cle : valeur}+ & $O(1)$\\
  \hline
  Test d'appartenance d'une clé & \verb+cle in d+ & $O(1)$\\
  \hline
  Ajout d'un couple clé/valeur & \verb+d[cle] = valeur+ & $O(1)$\\
  \hline
  Valeur correspondant à une clé & \verb+d[cle]+ & $O(1)$\\
  \hline
  Égalité de deux dictionnaires &  \verb+d1 == d2+ & $O(n)/O(1)$ si de tailles différentes\\
  \hline
  Ensemble des clés & \verb+d.keys()+ & $O(n)$\\
  \hline
\end{tabular}


\end{document}
