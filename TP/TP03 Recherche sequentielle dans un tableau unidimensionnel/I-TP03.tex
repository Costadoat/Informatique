\newcommand{\nom}{Porte conteneur}
\newcommand{\sequence}{03}
\newcommand{\num}{04}
\newcommand{\type}{TD}
\newcommand{\descrip}{Résolution d'un problème en utilisant des méthodes algorithmiques}
\newcommand{\competences}{Alt-C3: Concevoir un algorithme répondant à un problème précisément posé}
\documentclass[10pt,a4paper]{article}
  \usepackage[french]{babel}
  \usepackage[utf8]{inputenc}
  \usepackage[T1]{fontenc}
  \usepackage{xcolor}
  \usepackage[]{graphicx}
  \usepackage{makeidx}
  \usepackage{textcomp}
  \usepackage{amsmath}
  \usepackage{amssymb}
  \usepackage{stmaryrd}
  \usepackage{fancyhdr}
  \usepackage{lettrine}
  \usepackage{calc}
  \usepackage{boxedminipage}
  \usepackage[french,onelanguage, boxruled,linesnumbered]{algorithm2e}
  \usepackage[colorlinks=false,pdftex]{hyperref}
  \usepackage{minted}
  \usepackage{url}
  \usepackage[locale=FR]{siunitx}
  \usepackage{multicol}
  \usepackage{tikz}
  \makeindex

  %\graphicspath{{../Images/}}

%  \renewcommand\listingscaption{Programme}

  %\renewcommand{\thechapter}{\Alph{chapter}}
  \renewcommand{\thesection}{\Roman{section}}
  %\newcommand{\inter}{\vspace{0.5cm}%
  %\noindent }
  %\newcommand{\unite}{\ \textrm}
  \newcommand{\ud}{\mathrm{d}}
  \newcommand{\vect}{\overrightarrow}
  %\newcommand{\ch}{\mathrm{ch}} % cosinus hyperbolique
  %\newcommand{\sh}{\mathrm{sh}} % sinus hyperbolique

  \textwidth 160mm
  \textheight 250mm
  \hoffset=-1.70cm
  \voffset=-1.5cm
  \parindent=0cm

  \pagestyle{fancy}
  \fancyhead[L]{\bfseries {\large PTSI -- Dorian}}
  \fancyhead[C]{\bfseries{{\type} \no \numero}}
  \fancyhead[R]{\bfseries{\large Informatique}}
  \fancyfoot[C]{\thepage}
  \fancyfoot[L]{\footnotesize R. Costadoat, C. Darreye}
  \fancyfoot[R]{\small \today}
  
  \definecolor{bg}{rgb}{0.9,0.9,0.9}
  
  
  % macro Juliette
  
\usepackage{comment}   
\usepackage{amsthm}  
\theoremstyle{definition}
\newtheorem{exercice}{Exercice}
\newtheorem*{rappel}{Rappel}
\newtheorem*{remark}{Remarque}
\newtheorem*{defn}{Définition}
\newtheorem*{ppe}{Propriété}
\newtheorem{solution}{Solution}

\newcounter{num_quest} \setcounter{num_quest}{0}
\newcounter{num_rep} \setcounter{num_rep}{0}
\newcounter{num_cor} \setcounter{num_cor}{0}

\newcommand{\question}[1]{\refstepcounter{num_quest}\par
~\ \\ \parbox[t][][t]{0.15\linewidth}{\textbf{Question \arabic{num_quest}}}\parbox[t][][t]{0.85\linewidth}{#1\label{q\the\value{num_quest}}}\par
~\ \\}

\newcommand{\reponse}[4][1]
{\noindent
\rule{\linewidth}{.5pt}\\
\textbf{Question\ifthenelse{#1>1}{s}{} \multido{}{#1}{%
\refstepcounter{num_rep}\ref{q\the\value{num_rep}} }:} ~\ \\
\ifdef{\public}{#3 ~\ \\ \feuilleDR{#2}}{#4}
}

\newcommand{\cor}
{\refstepcounter{num_cor}
\noindent
\rule{\linewidth}{.5pt}
\textbf{Question \arabic{num_cor}:} \\
}


\ifdef{\public}{\excludecomment{solution}}


\begin{document}

\begin{center}
{\Large\bf TP \no {\numero} -- \descrip}
\end{center}

\SetKw{KwFrom}{de} 


\section{Manipulations de listes}

\begin{defn}
Une liste est une structure de données qui contient une série de valeurs. Pour en déclarer une, on utilise la syntaxe suivante :
\begin{minted}{python}
[1,2,3]
['a','x','mot']
[1,'chiffre',-2**3]
\end{minted}
\end{defn}

\begin{exercice}Commandes de base.
\begin{enumerate}
\item Créer la liste \verb? L=['a','b','c','d','e','f']?.
\item Afficher le résultat des commandes suivantes : \verb?L[0]?,  \verb?L[1]?. Comment faire pour afficher l'élément \verb?'d'? de \verb?L? ?
\item Que renvoie \verb?L[6]? ? Pourquoi ?
\item Que renvoie \verb?L[2:5]? ? Que renvoie en général \verb?L[i:j]? ? Et \verb?L[i:]? ?
\item Que renvoie \verb?len(L)? ? Comment faire pour afficher le dernier élément de \verb?L? ?
\item Opérations sur les listes : \verb?+? et \verb?*?. \\
Exécuter les commandes suivantes pour comprendre l'effet de ces opérations sur les listes (il faudra afficher le résultat des opérations pour constater leurs actions) :
\begin{minted}{python}
[1,2,3]+[7,8,9]
[1,2]+[6,7,8,9]
[0]*10
\end{minted}
\item Méthodes sur les listes : \verb?pop? et \verb?append?.\\
Exécuter les commandes suivantes pour comprendre l'effet de ces méthodes sur les listes :
\begin{minted}{python}
L=[1,2,3]
L.pop()
L.append(7)
\end{minted}
\end{enumerate}
\end{exercice}


\begin{defn}Liste en compréhension\\
Une liste en compréhension est une liste dont le contenu est défini par filtrage du contenu d'une autre liste. Sa construction se rapproche de la notation ensembliste en mathématiques.
\noindent Exemples : 
\begin{center}
\begin{tabular}{|l|l|l|}
\hline
en langage ensembliste :&$\mathcal{S}_1=\{3n+2|n\in\mathbb{N},\; n< 20\}$ \\[0.3cm]
en Python : &\verb?S1=[3*n+2 for n in range(20)]? \\
\hline
\end{tabular}
\end{center}
\end{defn}


\begin{exercice}~\\
Utilisez une liste en compréhension pour créer la liste des multiples de 5 entre 0 et 100.
\end{exercice}


\begin{exercice}
Soit \verb?L1? la liste suivante : \verb?L1=[12,-3,8,1.2,7]?.
\begin{enumerate}
\item Ajouter \verb?10? \` a la fin de la liste. Afficher la nouvelle liste.
\item Afficher l'élément \verb?1.2? de la liste.
\item Retirer l'élément \verb?12? de la liste. 
\item Modifier l'élément en position 1 de la liste par l'élément \verb?77?.\\
A ce stade, votre liste \verb?L1? doit être : \verb?[-3,77,1.2,7,10]?. 
\item Créer la liste \verb?L2=[-3,77,1.2,7,10,0,1,2,3,...,100]?.
\item Créer la liste des 30 premiers termes de \verb?L2? sans ses 3 premiers termes.
\item Créer la liste \verb?L2? suivie de 40 zéros.
\end{enumerate}
\end{exercice}

\begin{exercice}
Utilisez une boucle \verb?for? pour afficher un par un tous les termes de la liste \verb?L=[2,8,-7,3]?.\\
(il y a deux syntaxes possibles).
\end{exercice}

\begin{exercice}
Ecrire une fonction \verb?renverser? qui \` a une liste renvoie la liste renversée.\\
On reprogramme ainsi la méthode \verb?reverse? déj\` a implémentée dans Python.
\begin{minted}{python}
>>> L=[2,8,-1,7]
>>> renverser(L)
[7,-1,8,2]
\end{minted}
\end{exercice}


\section{La bibliothèque copy}

\begin{defn}
Une bibliothèque est un ensemble de fonctions. Celles-ci sont regroupées et mises à disposition afin de pouvoir être utilisées sans avoir à les réécrire. Celles-ci permettent de faire : du calcul numérique, du graphisme,... Quelques exemples : les bibliothèques \verb?math?, \verb?matplotlib.pyplot?, \verb?copy?. \\
Il existe plusieurs syntaxes pour importer une bibliothèque et pour ensuite dans le corps du programme, appeler les fonctions :

\begin{center}
\begin{tabular}{c|c|c}
\begin{minipage}{4cm}
\begin{minted}{python}
import math as m
m.cos(m.pi/3)
\end{minted} 
\end{minipage} &
\begin{minipage}{4cm}
\begin{minted}{python}
import math 
math.cos(math.pi/3)
\end{minted}        
\end{minipage}&
\begin{minipage}{4cm}
\begin{minted}{python}
from math import cos, pi
cos(pi/3)
\end{minted} 
\end{minipage} 
\end{tabular}
\end{center}

\end{defn}

\begin{exercice}
Importez la bibliothèque \verb?copy? sous l'alias \verb?copy?.\\
Testez les fonctions suivantes. Expliquez les résultats :
\begin{center}
\begin{tabular}{cc}
\begin{minipage}{7cm}
\begin{minted}[frame=lines]{python}
L=[1,2,3,4]
M=L
L[0]=100
print(L,M)
print(id(L),id(M))
\end{minted} 
\end{minipage} &
\begin{minipage}{7cm}
\begin{minted}[frame=lines]{python}
L=[1,2,3,4]
M=copy.copy(L)
L[0]=100
print(L,M)
print(id(L),id(M))
\end{minted}        
\end{minipage}\\
Programme n°1&Programme n°2
\end{tabular}
\end{center}
\end{exercice}







\section{Recherches}


\begin{exercice}Recherche dans une liste\\
Programmer une fonction \verb?position(liste,element)? qui a comme entrée une liste et un élément et qui renvoie la position de l'élément dans la liste et qui renvoie \verb?None? si l'élément n'est pas dans la liste.\\
On reprogramme ainsi la méthode \verb?index? déj\` a implémentée dans Python
\end{exercice}


\begin{exercice}Recherche du maximum dans une liste de nombres.
\begin{enumerate}
\item Ecrire une fonction \verb?maximum(liste)? qui renvoie le maximum d'une liste de nombres non triée.
\begin{minted}{python}
>>>maximum([2,8,-7,3])
8
\end{minted}
\item Dans l'algorithme précédent, combien de fois parcourt-on la liste ? 
\item Ecrire une fonction \verb?maximum2(liste)? qui renvoie la deuxième valeur maximale d'une liste de nombres non triés, tous distincts.
\begin{minted}{python}
>>>maximum2([2,8,-7,3])
3
\end{minted}
\end{enumerate}
\end{exercice}

\begin{exercice}
Pour les listes, il existe une fonction \verb?max? et une méthode \verb?index?. On en rappelle les syntaxes respectives dans l'annexe.\\
Testez \verb?max? et \verb?index? sur des exemples pour comprendre ce qu'elles renvoient.\\
A l'aide de \verb?max? et \verb?index?, déterminer la position du maximum de la liste \verb?L?. 
\end{exercice}





\section{Exercices en plus}

\begin{exercice}
Recherche d'un mot dans une cha\^ ine de caract\` eres.
\begin{enumerate}
\item Ecrire une fonction \verb?estIci(motif,texte,i)? qui a comme entrée deux listes (ou deux cha\^ ines de caract\` eres) \verb?motif? et \verb?texte? et un entier \verb?i? et qui renvoie \verb?True? si \verb?motif? est dans \verb?texte? \` a la position \verb?i? et \verb?False? sinon.
\begin{minted}{python}
>>>estIci('le','Bonjour le monde',8)
True
>>>estIci('le','Bonjour le monde',9)
False
\end{minted} 
\item Ecrire une fonction \verb?recherche(motif,texte)? qui a comme entrée deux listes (ou deux cha\^ ines de caract\` eres) et qui renvoie \verb?True? si \verb?motif? est dans \verb?texte? et \verb?False? sinon.
\begin{minted}{python}
>>>recherche('le','Bonjour le monde')
True
>>>recherche('bonjour','Bonjour le monde')
False
\end{minted}
%\item Déterminer la complexité de l'algorithme de recherche d'un motif dans une liste.\\\textit{Indication :} On se placera dans le pire des cas. La complexité dépendra de la longueur de \verb?motif? et de celle de \verb?liste?
\end{enumerate}
\end{exercice}



\begin{exercice}Calendrier grégorien.
%Vuibert IPT page 84

La manipulation des dates dans les logiciels de gestion, ou encore sur les sites web de réservation, doit s'effectuer conformément au calendrier grégorien (entré en vigueur à la fin du XVI\up{e} siècle en France) en précisant pour chaque date le jour de la semaine. Or le calendrier grégorien est un format assez éloigné des formats habituels de stockage des nombres dans un ordinateur. 

On cherche à déterminer, pour une date donnée, sa position dans l'énumération des jours depuis le 1\up{er} janvier 1600. et le jour de la semaine correspondant.

\begin{enumerate}
 \item Proposer une fonction \texttt{nombre\_de\_jours(jours,mois,annee)} qui prend en entrée une date de la forme \texttt{jours,mois,annee} postérieure au 1,1,1600 et renvoie le nombre de jours écoulés entre le 1\up{er} janvier 1600 et la date considérée, sans tenir compte des années bissextiles (c'est-à-dire en supposant que chaque année comporte 365 jours). On pourra utiliser une liste des nombres de jours de chaque mois pour une année non bissextile : \texttt{m = [31,28,31,30,31,30,31,31,30,31,30,31]}.

\item Les années bissextiles sont déterminées par la règle suivante\footnote{Wikipédia en français, « Année bissextile », consulté le 23 novembre 2015.} : 
\begin{itemize}
\item Si l'année est divisible par 4 et non-divisible par 100, c'est une année bissextile (elle a 366 jours).
\item Si l'année est divisible par 400, c'est une année bissextile (elle a 366 jours).
\item Sinon, l'année n'est pas bissextile (elle a 365 jours).
\end{itemize} 

Proposer une fonction \texttt{bissextile(annee)} renvoyant \texttt{True} si l'année est bissextile et \texttt{False} sinon.

 \item Modifier le programme de la fonction \texttt{nombre\_de\_jours(jours,mois,annee)} pour tenir compte des années bissextiles. Par exemple, \texttt{nombre\_de\_jours(1,2,1600)} doit retourner la valeur 31, \texttt{nombre\_de\_jours(1,1,1604)} la valeur 1461 (366+365+365+365) puisque l'année 1600 est bissextile.
 
 \item Le 1\up{er} janvier 2001 était un lundi. Déterminer, en utilisant la fonction de la question précédente, quel jour de la semaine tombera le 1\up{er} mai 2040. Quel jour de la semaine est tombé le 14 juillet 1789 ?
\end{enumerate}
\end{exercice}



\newpage

\section*{Annexe}


\subsubsection*{Fonctions et méthodes sur les listes}

\noindent En Python, il y a deux fa\c cons de faire des opérations sur les objets qu'on manipule : les fonctions et les méthodes.\\
La différence est, pour vous, surtout d'ordre syntaxique.\\
\begin{center}
\begin{minipage}{7cm}
Syntaxe fonction :
\begin{verbatim}
fonction(objet ,parametres)
\end{verbatim}
\end{minipage}
\begin{minipage}{7cm}
Syntaxe méthode :
\begin{verbatim}
objet.methode(parametres)
\end{verbatim}
\end{minipage}
\bigskip 
\end{center}
Pour les listes, on trouve des fonctions et des méthodes. Certaines modifient la liste et ne renvoient rien, d'autres renvoient un résultat sans modifier la liste.\bigskip \\
Exemples de méthodes pour les listes. 
\begin{itemize}
\item \textbf{append(élément)} : modifie la liste en ajoutant élément \` a la fin de la liste
\item \textbf{pop()} : modifie la liste en supprimant le dernier élément
%\item \textbf{remove(élément)} : modifie la liste en supprimant élément de la liste
\item \textbf{reverse()} : modifie la liste en inversant les valeurs de la liste
%\item \textbf{count(élément)} : renvoie le nombre d'occurrences d'élément dans la liste
\item \textbf{index(élément)} : renvoie la position de élément dans la liste
\bigskip 
\end{itemize}
Exemples de fonctions pour les listes :
\begin{itemize}
%\item \textbf{del liste[index]} : modifie la liste en éliminant l'item en position \textbf{index}
\item \textbf{len} : renvoie la longueur de la liste
\item \textbf{max} : renvoie le maximum d'une liste de nombres
\end{itemize}


\subsubsection*{Fonctions et méthodes au programme}

\begin{tabular}{|c|c|c|}
  \hline
  Opération & Exemple \\
  \hline 
  \hline 
  Création d'une liste vide & \verb+l=[]+ \\
  \hline 
   Création  & \verb+l=[1,56,13]+ \\
  \hline 
  Accès direct & \verb+l[0]+ \\
    \hline
  Extraction de tranche & \verb+l[1:10]   +   \verb+l[3:]+ \\
  \hline
  Longueur & \verb+len(l)+ \\
  \hline
  Concaténation & \verb-l1+l2- \\
    \hline
  Répétition & \verb+[0]*k+\\
  \hline
  Modification par affectation & \verb?l[0]=0 ?\\
  \hline
  Ajout en fin de liste & \verb+l.append(1)+ \\
  \hline
  Suppression en fin de liste & \verb+l.pop()+ \\
  \hline
  Création par compréhension & \verb+[k**2 for k in range(n)]+ \\
  \hline
  Copie & \verb+copy(l)+ \\
  \hline
\end{tabular}

\subsubsection*{Fonctions et méthodes sur les chaînes de caractères}
On retrouve certaines fonctions et méthodes pour les chaînes de caractères.

\begin{tabular}{|c|c|c|}
  \hline
  Opération & Exemple  \\
  \hline 
  \hline
  Création d'une chaîne vide & \verb+l=''+ \\
      \hline 
  Création & \verb+s = 'Ma chaîne'+ \\
  \hline
  Accès direct & \verb+s[0]+ \\
    \hline
  Extraction de tranche & \verb+s[1:10]+ \\
  \hline
  Longueur & \verb+len(s)+\\
  \hline
  Concaténation & \verb-s1+s2- \\
      \hline
  Répétition & \verb+'e'*k+\\
  \hline
\end{tabular}












\ifdef{\public}{\end{document}}{}

\newpage 

\begin{center}
{\Large\bf Correction TP \no {\numero} -- \descrip}
\end{center}



\begin{solution}
\begin{enumerate}
\item \verb? L=['a','b','c','d','e','f']?.
\item \verb?L[3]? renvoie \verb?'d'?.
\item \verb?L[6]? renvoie \verb? out of range? : on a dépassé l'indice maximal de la liste.
\item \verb?L[2:5]? renvoie \verb?['c','d','e']?. \\
\verb?L[i:j]? est la sous-liste de \verb?L? qui comprend les éléments d'indice $i$ à $j-1$. \\
\verb?L[i:]? renvoie la sous-liste de \verb?L? à partir de l'indice $i$.
\item \verb?len(L)? renvoie la longueur de la liste. \verb?L[len(L)]? renvoie le dernier terme.
\item \verb?L1+L2? concatène les deux listes.\\
\verb?L1*n? répète $n$ fois la liste \verb?L?.
\item L.append(élément) : modifie la liste en ajoutant élément à la fin de la liste
\item pop() : modifie la liste en supprimant le dernier élément
\end{enumerate}
\end{solution}


\begin{solution}~\\
\vspace{-0.7cm}
\begin{minted}{python}
L=[5*n for n in range(21)]
\end{minted}
\end{solution}


\begin{solution}
\begin{enumerate}
\item \verb?L1.append(10)?
\item \verb?print L1[3]?
\item \verb?L1=L1[1:]?
\item \verb?L1[1]=77?
\item \verb?L2=L+[i for i in range(101)]?
\item \verb?L3=L2[3:31]?
\item \verb?L4=L2+40*[0]?
\end{enumerate}
\end{solution}



\begin{solution}~\\
\vspace{-0.7cm}
\begin{minted}[frame=lines]{python}
for i in L:
    print(i)
\end{minted}
ou 
\begin{minted}[frame=lines]{python}
for i in range(len(L)):
	print(L[i])
\end{minted}
\end{solution}



\begin{solution}~\\
\vspace{-0.7cm}
\begin{minted}[frame=lines]{python}
def renverser(liste):
    n=len(liste)
    listeRenversee=[]
    for i in range(n):
        listeRenversee.append(liste[n-i-1])
    return(listeRenversee)  
\end{minted}
\end{solution}


\begin{solution}
Dans le programme n°1, il n'y a qu’un seul objet liste référencé par deux références : \verb?L? et \verb?M?. Si on modifie le contenu de \verb?L?, la liste \verb?M? est aussi modifiée. On le vérifie en affichant l'identité des objets référencés, qui est la même.\\
Dans le programme n°2, on crée un nouvel objet, avec une nouvelle référence.
\end{solution}


\begin{solution}~\\
\vspace{-0.7cm}
\begin{minted}[frame=lines]{python}
def f(L,a):
# on commence au début de la liste
    i=0         
    # tant que on n'est pas au bout de la liste et que le terme de la liste n'est pas a, on avance dans la liste
    while i<len(L) and L[i]!=a:
        i=i+1
    # si on est sorti de la liste, on renvoie None    
    if i==len(L):
        return(None)
    # sinon on renvoie la position    
    else:
        return(i)                    
\end{minted}
\end{solution}



\begin{solution}~\\
\vspace{-0.7cm}
\begin{enumerate}
\item 
\begin{minted}[frame=lines]{python}
def maximum(liste):
    n=len(liste)
    # le maximum est initialisé avec le premier terme de la liste
    max=liste[0]
    
    # on parcourt la liste
    for i in range(n):
    # si le i\` eme terme de la liste est supérieur au maximum, il devient la nouvelle valeur du maximum\\
        if liste[i]>max:
            max=liste[i]
    return(max)                    
\end{minted}


\item Dans cette fonction, on compte le nombre d'opérations :
\begin{itemize}
\item deux affectations avant la boucle
\item $n$ itérations de la boucle dans laquelle on effectue :
\begin{itemize}
\item un test
\item au pire des cas, une affectation.
\end{itemize}
\end{itemize}
Au total, on a comme nombre d'opérations : $2+2n=O(n)$.
%\fbox{On a donc bien une complexité linéaire.}
\item 
\begin{minted}[frame=lines]{python}
def maximum2(L):
    Max=L[0]
    DeuxiemeMax=L[1]
    if Max<DeuxiemeMax:
        (Max,DeuxiemeMax)=(DeuxiemeMax,Max)
    for i in range(2,len(L)):
        if L[i]>Max:
            DeuxiemeMax=Max
            Max=L[i]
        elif:
            L[i]>DeuxiemeMax
            DeuxiemeMax=L[i]
    return(DeuxiemeMax)            
\end{minted}
\end{enumerate}
\end{solution}



\begin{solution}~\\
\vspace{-0.7cm}
\begin{minted}{python}
>>> L.index(max(L))
\end{minted}
\end{solution}




\begin{solution}
\begin{enumerate}
\item ~\\
\vspace{-0.7cm}
\begin{minted}[frame=lines]{python}
def estIci(motif,texte,i):
    k = 0
    p = len(motif)
    # on parcourt le texte \` a partir de l'index i tant que on ne dépasse pas la longueur de motif et que texte et motif sont identiques
    while k<p and motif[k] == texte[k+i]:
        k = k+1
    # si on a parcouru p termes, alors, motif est dans texte \` a l'index i
    if k==p:
        resultat=True
    else:
        resultat=False    	
    return resultat                
\end{minted}

\item ~\\
\vspace{-0.7cm}
\begin{minted}[frame=lines]{python}
def recherche(motif,texte):
    n = len(texte)
    p = len(motif)
    # si motif est plus long que texte, on renvoie False
    if p>n:
        return False
    else:
    # par défaut, Resultat est False
        resultat=False
        i=0
        # on cherche motif dans texte \` a toutes les positions i 
        while i <= n-p and resultat==False:
            resultat=estIci(motif,texte,i)
            i = i + 1
        return resultat           
\end{minted}
%\item Le pire des cas s'obtient avec un texte type : \verb?texte='aaaaa...aaaaa'? et un motif du type : \verb?'aa..aab'?. On note $n$ la longueur du texte et $p$ celle du motif.\\
%Compté grossi\` erement : dans \verb?estIci(motif,texte,i)?, on effectue $p$ tests. On rép\` ete $n-p$ fois cette opération dans \verb?recherche(motif,texte)?. Donc on s'attend \` a une complexité en : \fbox{$O((n-p)p)$.} \\
%On obtient $O(np)$ si $n>>p$\bigskip \\
%Plus précisément, avec le pire des cas :\\
%dans \verb?estIci(motif,texte,i)?, on a :
%\begin{itemize}
%\item 2 affectations avant la boucle
%\item on rép\` ete $p$ fois :
%\begin{itemize}
%\item 2 tests
%\item 1 affectation
%\item 1 addition
%\end{itemize}
%\item 1 test
%\item 1 affectation
%\end{itemize}
%\fbox{Au total, on a comme nombres d'opérations : $2+4p+2=4p+4=O(p)$.}\\
% \verb?recherche(motif,texte)?, on a :
% \begin{itemize}
% \item 2 affectations
% \item dans la boucle else : 2 affectations
% \item on rép\` ete $n-p$ fois :
% \begin{itemize}
% \item 2 tests
% \item 1 affectation
% \item 1 calcul de \verb?estIci?
% \item 1 afffectation
% \item 1 addition
% \end{itemize}
% \end{itemize}
%Au total, on a comme nombres d'opérations : \\
% \fbox{$2+2+(n-p)\times (5+4p+4)=4+(n-p)(4p+9)=O((n-p)p)$.}
\end{enumerate}
\end{solution}


\begin{solution}Calendrier grégorien.
%Vuibert IPT page 108

\begin{enumerate}
 \item Sans tenir compte des années bissextiles :
 \begin{itemize}
  \item On compte 365 jours pour toutes les années entièrement écoulées. 
  \item On ajoute le nombre de jours des mois entièrement écoulés dans l'année en cours. 
  \item On ajoute le nombre de jours dans le mois en cours.                                                                                                                                                                                                           \end{itemize}

\begin{minted}[linenos,frame=lines]{python}
def nombre_de_jours_annees_classiques(jour,mois,annee):
  m = [31,28,31,30,31,30,31,31,30,31,30,31]
  nbjours = (annee-1600)*365 # nombre de jours ecoules pour les annees entieres
  for i in range(mois-1): # mois numerotes de 1 a 12 ; liste indexee de 0 a 11
    nbjours = nbjours + m[i]  # nombre de jours ecoules pour les mois termines
  nbjours = nbjours + jour - 1 # nombre de jours ecoules pour le mois en cours
  return nbjours
\end{minted}

 \item Détermination des années bissextiles.

\begin{minted}[linenos,frame=lines]{python}
def bissextile(annee):
  if annee % 4 == 0:
    if annee % 100 == 0:
      if annee % 400 == 0: 
        bissex = True # annee multiple de 400
      else:
        bissex = False # annee multiple de 4 et de 100 mais pas de 400
    else:
      bissex = True # annee multiple de 4 mais pas de 100
  else:
    bissex = False # annee non multiple de 4
  return bissex
\end{minted}

 \item En tenant compte des années bissextiles :
 \begin{itemize}
  \item On compte 365 ou 366 jours pour toutes les années entièrement écoulées selon qu'elles sont bissextiles ou non. 
  \item On ajoute le nombre de jours des mois classiques entièrement écoulés dans l'année en cours. On ajoute un jour si le mois de février de l'année en cours est terminé et que l'année en cours est bissextile.
  \item On ajoute le nombre de jours dans le mois en cours.
  \end{itemize}

\begin{minted}[linenos,frame=lines]{python}
def nombre_de_jours_toutes_annees(jour,mois,annee):
  m = [31,28,31,30,31,30,31,31,30,31,30,31]
  nbjours = 0
  for an in range(1600,annee): # pour tous les ans entierement ecoules
    if bissextile(an):
      nbjours = nbjours + 366 # si annee bissextile on ajoute 366
    else:
      nbjours = nbjours + 365 # sinon on ajoute 365
  for i in range(mois-1): 
    nbjours = nbjours + m[i] # nombre de jours ecoules pour les mois termines
  if (mois > 3 or mois == 3) and bissextile(annee):
    nbjours = nbjours + 1 # on ajoute 1 si fevrier est termine et
			  # que l'annee en cours bissextile
  nbjours = nbjours + jour - 1 # nombre de jours ecoules pour le mois en cours
  return nbjours  
\end{minted}
 
 \item Les semaines font toujours 7 jours. On détermine donc uniquement le nombre de jours écoulés depuis le 1\up{er} janvier 2001 modulo 7. Si la date testée est postérieure au 1\up{er} janvier 2001, le reste est négatif ce qui ne pose pas de souci à python qui gère les indices négatifs dans les listes (-1 étant l'indice du dernier élément, -2 celui de l'avant -dernier, etc.).

\begin{minted}[linenos,frame=lines]{python}
jours = ["lundi","mardi","mercredi","jeudi","vendredi","samedi","dimanche"]

print(jours[
( nombre_de_jours_toutes_annees(1,5,2040) 
- nombre_de_jours_toutes_annees(1,1,2001) ) % 7
])

print(jours[
( nombre_de_jours_toutes_annees(14,7,1789) 
- nombre_de_jours_toutes_annees(1,1,2001) ) % 7
])
\end{minted}
\end{enumerate}
\end{solution}




\end{document}
















\section{GNA : Reste}


\begin{exercice}
Ecrire une fonction \verb?supprime? qui prend comme entrée une liste \verb?L? et un élément \verb?a? et qui supprime de \verb?L? toutes les occurrences de \verb?a?.
\begin{verbatim}
>>>L=[2,3,3,2,16]
>>>supprime(L,3)
[2,2,16]
\end{verbatim}
\end{exercice}



\begin{exercice}Suite de Syracuse\\
On consid\` ere la suite $(u_n)_{n\in\N}$ définie par la relation de récurrence suivante :
\[u_0\in\N^*\qquad \text{ et }\qquad u_{n+1}=\left\lbrace\begin{array}{lll}
\dfrac{u_n}{2}&\text{ si } u_n \text{ est pair} \bigskip \\
3u_n+1&\text{ si } u_n \text{ est impair}
\end{array}\right.\]
La suite $(u_n)_{n\in\N}$ est une suite d'entiers. 
\begin{enumerate}
\item Ecrire une fonction \verb?syracuse(n,u0)? qui prend \verb?n? et \verb?u0? comme entrée et renvoie la liste $[u_0,u_1,\cdots,u_n]$.
\item Tester la fonction. (\verb?syracuse(20,15)? renvoie \verb?GNA?)
\item Tester cette fonction pour différentes valeurs de $n$ et de $u_0$. Que constatez-vous ? 
\end{enumerate}
\end{exercice}






\begin{exercice}
\begin{enumerate}
\item Une fonction \verb?max? existe. Testez-la sur nos exemples.
\item Pour trouver la position du maximum, on peut utiliser la fonction numpy \verb?where? :\\
\verb?np.where? (condition) renvoie les positions du tableau o\` u la condition est vérifiée. \\
Dans $T$, quelles sont les valeurs supérieures strictes \` a 4 ? En quelles positions sont-elles ?
\item Testez les commandes suivantes et vérifiez les résultats :\\
\verb?np.where(T>4)?, \verb?np.where(T=1)? \footnote{Pourquoi cela renvoie-t-il un message d'erreur ? Corrigez}, \verb?np.where(T<-100)?
\item Ecrire une commande qui permet d'avoir les positions du maximum de $T$.
\end{enumerate}
\end{exercice}





\section{Manipulation de tableaux}
\begin{defn}
Dans la section suivante, nous travaillerons \` a l'aide de tableaux. Importez le module numpy qui permet de nombreuses manipulations.\\
Un tableau sera vu comme une liste de listes. \\
Exemple : $T=[[1,2,3],[-1,4,2],[0,6,5],[12,1,-4]]$.
\end{defn}

\begin{exercice}
\begin{enumerate}
\item Recopier le tableau T en prenant soin d'en faire un tableau numpy. Vérifier son type.
\item Extraire la premi\` ere ligne de $T$.
\item Extraire le terme \verb?4? de $T$.
\item Ecrire un programme qui affiche tous les termes de $T$ :
\begin{verbatim}
1
2
3
-1
4
\end{verbatim}
etc...
\end{enumerate}
\end{exercice}

\begin{exercice}
\begin{enumerate}
\item Que renvoie \verb?T+T?, \verb?3*T? ? Qu'aurions nous obtenu si $T$ était une liste ?\\
Notez que l'addition et la multiplication externe sur des tableaux numpy sont les opérations matricielles vues en cours de maths.
\item Pour concaténer deux tableaux numpy \` a deux dimensions, on peut utiliser la fonction numpy \verb?concatenate? qui prend comme argument, les deux tableaux ou parties de tableaux, et une option \verb?axis?.\\
Testez la fonction sur les tableaux suivants. Que fait l'option \verb?axis? ?
\begin{verbatim}
A=np.array([[1,2,3],[4,5,6]])
B=np.array([[7,8,9],[10,11,12]])
C=np.concatenate((A,B),axis=0)
D=np.concatenate((A,B),axis=1)
\end{verbatim}
\end{enumerate}
\end{exercice}

\begin{exercice}
Ecrire une fonction \verb?taille? qui renvoie la taille d'un tableau.
\end{exercice}



\begin{exercice}GNA bof
Ecire une fonction \verb?tri? qui \` a une liste de nombres renvoie la liste triée dans l'ordre croissant.
\begin{verbatim}
>>>L=[2,8,-1,7]
>>>tri(L)
[-1,2,7,8]
\end{verbatim}

\end{exercice}



GNA ? garder ?
\begin{exercice}~\\
Utilisez une liste en compréhension pour créer la liste \verb?['a1','a2','a3','b1','b2','b3']? \` a partir des deux cha\^ ines de caract\` eres \verb?'ab'? et \verb?'123'?.
\end{exercice}

