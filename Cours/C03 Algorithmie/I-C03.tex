\input{../../headers/beamercoursHeadings}

\section{Algorithme} 

\ifdef{\public}{\begin{frame}
\frametitle{Table des matières}
\tableofcontents[currentsection]
\end{frame}}{}

{\frame{
\frametitle{Algorithme}

\begin{defi}
Un \textbf{algorithme} est une procédure permettant de résoudre un problème, écrite de façon suffisamment détaillée
pour pouvoir être suivie sans posséder de compétence particulière ni même comprendre le problème que l'on est en
train de résoudre.
\end{defi}

\begin{defi}
Un \textbf{programme} est la traduction d'un algorithme dans un langage particulier, à la fois interprétable par la machine
et compréhensible par l'homme. Il est constitué d'un assemblage d'instructions, regroupées dans un fichier texte
appelé le code source du programme.
\end{defi}

\begin{rem}
En toutes circonstances, il faudra s'assurer que :
\begin{itemize}
 \item le résultat de l'algorithme est conforme au résultat attendu ;
 \item l'algorithme soit documenté.
\end{itemize}
\end{rem}
}}

{\frame{
\frametitle{Exemple des algorithmes}

\textit{Objectif: Déterminer la valeur de $\pi$ par dichotomie.}

On sait que :
\begin{itemize}
 \item $sin(x) > 0$ pour $x < pi$,
 \item $sin(x) <= 0$ pour $x >= pi$.
\end{itemize}

Ainsi un algorithme permet de calculer Pi:
\begin{enumerate}
 \item Soient $a = 3$, $b = 4$,
 \item Soit $m$ le milieu de $a$ et $b$,
 \item Si $sin(m) > 0$, $a = m$ sinon $b = m$,
 \item Recommencer à partir de l'étape 2, tant que la précision n'est pas celle souhaitée.
\end{enumerate}
}}

{\frame{
\frametitle{Sémantique}

\textbf{Opérateurs}

Chaque type de donnée admet un jeu d'opérateurs adaptés :
\begin{itemize}
 \item opérateurs arithmétiques : $+$, $-$, $*$, $/$, $div$, $mod$, $?$, $**$,
 \item opérateurs relationnels : $==$, $\neq$, $>$, $<$, $\leq$,
 \item opérateurs logiques : \textit{négation}, \textit{ou}, \textit{et}.
\end{itemize}

\textbf{Instruction}

Il existe aussi des opérateurs d'instruction comme l'opérateur d'affectation : $\leftarrow$.


\begin{defi}
Une \textbf{expression} est un groupe de nombres, constantes, variables liées par des opérateurs. Elles sont évaluées pour donner un résultat. \\
Une séquence d'\textbf{instructions} est appelé bloc d'instructions. Lors de l'exécution du programme, les instructions s'exécutent les unes après les autres.
\end{defi}
}}

\section{Définition de fonctions} 

\ifdef{\public}{\begin{frame}
\frametitle{Table des matières}
\tableofcontents[currentsection]
\end{frame}}{}

\begin{frame}[fragile]
\frametitle{Les fonctions}

\begin{minipage}{0.55\linewidth}
\begin{defi}
Les fonctions permettent :
\begin{itemize}
 \item d'automatiser des tâches répétitives,
 \item d'ajouter de la clarté à un algorithme,
 \item d'utiliser des portions de code dans un autre algorithme.
\end{itemize}
\end{defi}
\end{minipage}\hfill
\begin{minipage}{0.35\linewidth}
Une analogie avec les mathématiques reviendrait à dire que: 
$f:x\rightarrow y$ s'écrit \\
\begin{GrayBox}[0.75\textwidth]
\begin{verbatimtab}[3]
def f(x):
    return y
\end{verbatimtab}
\end{GrayBox}
\end{minipage}

Lors de l'exécution d'un programme, il faut parfois répéter un grand nombre de fois une instruction. Ainsi, il peut être pratique de créer une fonction qui permet de limiter l'écriture.

\vspace{-0.1cm}

\begin{GrayBox}[0.75\textwidth]
\begin{verbatimtab}[3]
>>>def moyenne(a,b):
    m=(b+a)/2.
    return m
>>>a=(moyenne(40,2))
>>>print a
	21.0
\end{verbatimtab}
\end{GrayBox}
\end{frame}

\begin{frame}[fragile]
\frametitle{Variables locales/globales}

\begin{defi}
\textbf{Visibilité} : Une variable globale est définie en dehors de toute fonction, une variable locale est définie dans une
fonction et masque toute autre variable du même nom. \\
\textbf{Durée de vie} : Une variable globale existe durant l'exécution du programme, une variable locale existe durant
l'exécution de la fonction. \\
Par défaut, dans un langage interprété, les variables sont locales à un bloc.
\end{defi}

\begin{minipage}{0.3\linewidth}
\begin{GrayBox}[0.75\textwidth]
\begin{verbatimtab}[3]
a=4
def f(x):
	a=3
	b=2
	res = a*x+b
	return res
y=f(a)
\end{verbatimtab}
\end{GrayBox}
\end{minipage}\hfill
\begin{minipage}{0.65\linewidth}
\begin{GrayBox}[0.75\textwidth]
\begin{verbatimtab}[3]
>>> print a
4
>>> print y
14
>>> print b
Traceback (most recent call last):
  File "<stdin>", line 1, in <module>
NameError: name 'b' is not defined
\end{verbatimtab}
\end{GrayBox}
\end{minipage}
\end{frame}

\begin{frame}[fragile]
\frametitle{Documentation des fonctions}

Il faut absolument commenter les fonctions mises en place dans un programme afin de pouvoir revenir sur un code plus tard.

\begin{GrayBox}[0.8\textwidth]
\begin{verbatimtab}[3]
def f(x):
	# x est l'abscisse de la droite déterminée par la droite f
	# la réponse à la fonction est y=a*x+b
	a=3
	# a est le coefficient directeur de la droite
	b=2
	# b est l'ordonnée à l'origine de la droite
	res = a*x+b
	return res
\end{verbatimtab}
\end{GrayBox}
\end{frame}

\begin{frame}[fragile]
\frametitle{Import de méthodes et de fonctions}

Par défaut, Python ne permet que de réaliser des opérations élémentaires (opérations mathématiques élémentaires, comparaisons, boucles etc.).

Il existe par ailleurs un grand nombre de bibliothèques permettant d'utiliser beaucoup plus de fonctions permettant de faciliter votre travail.

\begin{GrayBox}[0.95\textwidth]
\begin{verbatimtab}[3]
import math # Import de toutes les methodes de la bibliotheque math
math.sqrt(2) # Permet d'utiliser la methode sqrt de math
from math import sqrt # Import de la methode sqrt de math
import os # Import de la bibliotheque os permettant
				 de realiser des operations systemes
\end{verbatimtab}
\end{GrayBox}
\end{frame}

\begin{frame}[fragile]
\frametitle{Exemples d'algorithmes}

Dans l'exemple de l'algorithme du calcul de $\pi$, il est nécessaire d'arrêter le calcul au bout d'un certain temps.

Exemple de boucle \textbf{tant que}, ou \textbf{while}.

\begin{GrayBox}[0.75\textwidth]
\begin{verbatimtab}[3]
while math.sin(m) > 10**-10 or math.sin(m) < -10**-10:
	# tant que le sinus est plus petit que +- 10^-10
\end{verbatimtab}
\end{GrayBox}

Exemple d'instruction fonctionnelle \textbf{si}, ou \textbf{if}.

\begin{GrayBox}[0.75\textwidth]
\begin{verbatimtab}[3]
if math.sin(m) > 0:
	# action à effectuer si sin(m)>0
else:
	# action à effectuer si sin(m)<0
\end{verbatimtab}
\end{GrayBox}
\end{frame}



\begin{frame}[fragile]
\frametitle{Exemples d'algorithmes}

Exemple de boucle d'instructions itératives, ou \textbf{for}.

\begin{GrayBox}[0.75\textwidth]
\begin{verbatimtab}[3]
a,n=1,8
for i in range(1,n+1):
	a=a*i # réaliser une action n fois
\end{verbatimtab}
\end{GrayBox}

Il est parfois nécessaire de mettre en place des systèmes de gestion des erreurs qui permettent de contrôler les données d'entrée des fonctions.

\begin{GrayBox}[0.8\textwidth]
\begin{verbatimtab}[3]
p=math.pi
if (p > a and p < b) or (p < a and p > b):
    print "Faire le calcul"
else:
    print "Impossible de trouver pi entre %s et %s" % (a,b)
\end{verbatimtab}
\end{GrayBox}

\vspace{-0.5cm}

\end{frame}


\end{document}
