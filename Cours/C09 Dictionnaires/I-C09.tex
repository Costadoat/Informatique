\input{../../headers/beamercoursHeadings}

\section{Les dictionnaires} 

\begin{frame}[fragile]
\frametitle{Présentation des dictionnaires}

Comme les \textbf{listes} ou les \textbf{tuples}, les dictionnaures font partie des \textbf{types construits} à partir de types construits (\textit{int}, \textit{float}, \textit{str},...)

Alors que les éléments d’une liste sont ordonnés et qu'on y accède grâce à un index, on accédera à un élément d'un dictionnaire grâce à une \textbf{clé}.

C'est pour cela que cela porte le même nom qu'un dictionnaire (livre) où on accède à une définition avec un mot.

Contrairement aux listes qui sont délimitées par des crochets, on utilise des accolades pour les dictionnaires.

\begin{GrayBox}[0.85\textwidth]
\begin{verbatimtab}[3]
dictionnaire = {"a": "mot1", "b": "mot2"}
print(dictionnaire["a"])
	"mot1"
\end{verbatimtab}
\end{GrayBox}

\end{frame}

\begin{frame}[fragile]
\frametitle{La manipulation des dictionnaires}

\begin{minipage}[t]{0.38\linewidth}
Ajouter un élément à un dictionnaire (on peut mixer les types)
\end{minipage}\hfill
\begin{minipage}[t]{0.58\linewidth}
\vspace{-0.5cm}
\begin{GrayBox}[0.85\textwidth]
\begin{verbatimtab}[3]
dictionnaire["c"]=3
\end{verbatimtab}
\end{GrayBox}
\end{minipage}

\begin{minipage}[t]{0.38\linewidth}
Modifier un élément d'un dictionnaire
\end{minipage}\hfill
\begin{minipage}[t]{0.58\linewidth}
\vspace{-0.5cm}
\begin{GrayBox}[0.85\textwidth]
\begin{verbatimtab}[3]
dictionnaire["b"]='mot3'
print(dictionnaire["b"])
	"mot3"
\end{verbatimtab}
\end{GrayBox}
\end{minipage}

\begin{minipage}[t]{0.38\linewidth}
Parcourir tous les éléments d'un dictionnaire (peu d'intérêt)
\end{minipage}\hfill
\begin{minipage}[t]{0.58\linewidth}
\vspace{-0.5cm}
\begin{GrayBox}[0.85\textwidth]
\begin{verbatimtab}[3]
for i in dictionnaire.items():
    print(i)
\end{verbatimtab}
\end{GrayBox}
\end{minipage}

\begin{minipage}[t]{0.38\linewidth}
Créer un dictionnaire
\end{minipage}\hfill
\begin{minipage}[t]{0.58\linewidth}
\vspace{-0.5cm}
\begin{GrayBox}[0.85\textwidth]
\begin{verbatimtab}[3]
dictionnaire={}
\end{verbatimtab}
\end{GrayBox}
\end{minipage}
\end{frame}


\begin{frame}[fragile]
\frametitle{Exemple de dictionnaire : le format JSON}

Le JavaScript Object Notation (JSON) est un format standard utilisé pour représenter des données structurées de façon semblable aux objets Javascript.

Il est usuellement mis sous la forme d'un dictionnaire.

Des informations sur les lieux de tournage à Paris sont disponibles au format JSON à l'adresse

\url{https://opendata.paris.fr/explore/dataset/lieux-de-tournage-a-paris/}

\end{frame}

\begin{frame}[fragile]
\frametitle{Exemple de dictionnaire : le format JSON}

\small
\hspace{-1cm}
\begin{minipage}{0.45\linewidth}
\begin{verbatimtab}[3]
{
    "datasetid":"lieux-de-tournage-a-paris",
    "recordid":"2e04541164cf790cc4098...",
    "fields":{
        "coord_y":48.88533162,
        "type_tournage":"Série Web",
        "nom_producteur":"MY BOX FILMS",
        "date_fin":"2021-03-22",
        "geo_point_2d":[
            48.88533162277764,
            2.343681821668969
        ]
        ,
        "nom_tournage":"PLAN COEUR - S3",
        "ardt_lieu":"75018",
        "geo_shape":{
            "coordinates":[
                2.343681821668969,
                48.88533162277764
            ]
            ,
            "type":"Point"
\end{verbatimtab}
\end{minipage}\hspace{0.cm}
\begin{minipage}{0.45\linewidth}
\begin{verbatimtab}[3]
        }
        ,
        "id_lieu":"2021-295",
        "nom_realisateur":"NOÉMIE SAGLIO",
        "adresse_lieu":"square louise michel, 75018 paris",
        "date_debut":"2021-03-22",
        "annee_tournage":"2021",
        "coord_x":2.34368182
	    }
    	,
   		"geometry":{
        "type":"Point",
    	"coordinates":[
        	2.343681821668969,
        	48.88533162277764
        	]
	    }
    	,
    	"record_timestamp":"2022-02-21T11:01:17.756Z"
		}

\end{verbatimtab}
\end{minipage}


\end{frame}
\end{document}
