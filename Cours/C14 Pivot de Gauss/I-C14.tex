\input{../../headers/beamercoursHeadings}

\section{Cas triangulaire} 

\begin{frame}[fragile]
\frametitle{Système d'équations}

Soit le système d'équations suivant :
\begin{center}
$\left\{\begin{array}{l}
y_0=a_{0,0}.x_0+a_{0,1}.x_1+a_{0,2}.x_2+...+a_{0,n-1}.x_{n-1} \\
y_1=a_{1,0}.x_1+a_{1,1}.x_2+a_{1,2}.x_3+...+a_{1,n-1}.x_{n-1} \\
...\\
y_{n-1}=a_{n-1,0}.x_0+a_{n-1,1}.x_1+a_{n-1,2}.x_2+...+a_{n-1,n-1}.x_{n-1}
\end{array}\right.$
\end{center}

Ce système d'équations peut s'écrire sous la forme suivante: $Y=A.X$, avec:\\
\begin{center}
$X=\left(\begin{array}{c}
x_0 \\ x_1 \\ ...\\ x_{n-1}
\end{array}\right)$,
$Y=\left(\begin{array}{c}
y_0 \\ y_1 \\ ...\\ y_{n-1}
\end{array}\right)$,
$A=\left(\begin{array}{c c c c c}
a_{0,0} & a_{0,1} & a_{0,2} & ... & a_{0,n-1} \\
a_{1,0} & a_{1,1} & a_{1,2} & ... & a_{1,n-1} \\
... & ... & ... & ... & ... \\
a_{n-1,0} & a_{n-1,1} & a_{n-1,2} & ... & a_{n-1,n-1}
\end{array}\right)$
\end{center}

Ces systèmes sont dit de Cramer, c'est à dire qu'ils admettent une unique solution. La matrice A est donc inversible.

\end{frame}

\begin{frame}[fragile]
\frametitle{Cas triangulaire}

Si la matrice A est triangulaire, la résolution du problème est simple.

Ce système d'équations peut s'écrire sous la forme suivante: $Y=A.X$, avec:\\
\begin{center}
$A=\left(\begin{array}{c c c c c}
a_{0,0} & a_{0,1} & a_{0,2} & ... & a_{0,n-1} \\
0 & a_{1,1} & a_{1,2} & ... & a_{1,n-1} \\
0 & 0 & a_{2,2} & ... & a_{2,n-1} \\
... & ... & ... & ... & ... \\
0 & 0 & 0 & ... & a_{n-1,n-1}
\end{array}\right)$
\end{center}

La dernière équation devient: $y_{n-1}=a_{n-1,n-1}.x_{n-1}$, ainsi $x_{n-1}=\dfrac{y_{n-1}}{a_{n-1,n-1}}$. En remontant tout le système l'ensemble des $x_i$ peuvent être déterminés.

Ainsi, $\forall i \in [0,n-2], x_i=\dfrac{y_i-\sum \limits_{j=i+1}^{n-1} a_{i,j}.x_j}{a_{i,i}}$

\end{frame}

\begin{frame}[fragile]
\frametitle{Cas triangulaire}

Soit la matrice triangulaire $A$ est le vecteur $Y$ suivant:
\begin{center}
$A=\left(\begin{array}{c c c}
2 & 5 & 3 \\
0 & 1 & 2 \\
0 & 0 & 4
\end{array}\right)$,$Y=\left(\begin{array}{c}
6 \\
4 \\
9
\end{array}\right)$
\end{center}

Cela correspond au système d'équations suivant :
\begin{center}
$\left\{\begin{array}{l}
6=2.x_0+5.x_1+3.x_2 \\
4=1.x_1+2.x_2 \\
9=4.x_2
\end{array}\right.$
\end{center}

Cela mène à la résolution suivante :
\begin{center}
$\left\{\begin{array}{l}
x_0=\ifdef{\public}{\uncover<2->{0.875\\}}{ \\}
x_1=\ifdef{\public}{\uncover<2->{-0.5\\}}{ \\}
x_2=\ifdef{\public}{\uncover<2->{2.25}}{}
\end{array}\right.$
\end{center}

\end{frame}

\begin{frame}[fragile]
\frametitle{Cas triangulaire: Code Python}

\begin{GrayBox}[0.85\textwidth]
\begin{semiverbatim}\small
\ifdef{\public}{\uncover<2->{
def triangle(A,b):
    n =len(b)
    x = [0 for i in range(n)]
    x[n-1] = b[n-1]/A[n-1][n-1]
    for i in range(n-2,-1,-1):
        s = 0
        for j in range(i+1,n):
            s = s + A[i][j]*x[j]
        x[i] = (b[i] - s)/ A[i][i]
    return x}}{\vspace{5cm}}
   \end{semiverbatim}
\end{GrayBox}
\end{frame}

\begin{frame}[fragile]
\frametitle{Etape 1: Placer le pivot sur la première ligne}

Soit la matrice $A=\left(\begin{array}{c c c c}
2 & 2 & -3 & 4 \\
-2 & -1 & -3 & 2 \\
6 & 4 & 4 & 5 \\
1 & 5 & 3 & 2
\end{array}\right)$,$Y=\left(\begin{array}{c}
2 \\
-5 \\
16 \\
11
\end{array}\right)$

\begin{enumerate}
 \item Recherche de la plus grande valeur $A[0]_max$ de la première colonne de la matrice: ici $A[2][0]=6$,
 \item Inverser cette ligne et la première de la matrice et faire de même pour le vecteur.
\end{enumerate}

$A=\left(\begin{array}{c c c c}
6 & 4 & 4 & 5 \\
-2 & -1 & -3 & 2 \\
2 & 2 & -3 & 4 \\
1 & 5 & 3 & 2
\end{array}\right)$,$Y=\left(\begin{array}{c}
16 \\
-5 \\
2 \\
11
\end{array}\right)$

\end{frame}

\section{Mise sous forme triangulaire}

\begin{frame}[fragile]
\frametitle{Etape 2: Mettre des 0 sur le reste de la première colonne}

Soit la matrice $A=\left(\begin{array}{c c c c}
6 & 4 & 4 & 5 \\
-2 & -1 & -3 & 2 \\
2 & 2 & -3 & 4 \\
1 & 5 & 3 & 2
\end{array}\right)$,$Y=\left(\begin{array}{c}
16 \\
-5 \\
2 \\
11
\end{array}\right)$

Pour chaque ligne sauf la première
\begin{enumerate}
 \item Calcul du coefficient $x=A[i][0]/A[0][0]$,
 \item Calculer les nouveaux coefficients qui permettent d'annuler le premier:  $A[i][k] = A[i][k] - x*A[j][k]$
\end{enumerate}

$A=\left(\begin{array}{c c c c}
6 & 4 & 4 & 5 \\
\ifdef{\public}{\uncover<2->{0 & 0.33 & -1.66 & 3.66\\}}{ \\}
\ifdef{\public}{\uncover<2->{0 & 0.66 & -4.33 & 2.33\\}}{ \\}
\ifdef{\public}{\uncover<2->{0 & 4.33 & 2.33 & 1.16}}{}
\end{array}\right)$,$Y=\left(\begin{array}{c}
16 \\
0.33 \\
-3.33 \\
8.33
\end{array}\right)$
\end{frame}

\begin{frame}[fragile]
\frametitle{Etape 3: Mettre des 0 sur le reste de la première colonne}

La procédure continue avec la matrice de dimension juste inférieure

$A=\left(\begin{array}{c c c}
0.33 & -1.66 & 3.66 \\
0.66 & -4.33 & 2.33 \\
4.33 & 2.33 & 1.16
\end{array}\right)$,$Y=\left(\begin{array}{c}
0.33 \\
-3.33 \\
8.33
\end{array}\right)$

Le résultat de l'étape de mise sous forme triangulaire est alors:
$A=\left(\begin{array}{c c c c}
6 & 4 & 4 & 5 \\
\ifdef{\public}{\uncover<2->{0 & 4.33 & 2.33 & 1.16\\}}{ \\}
\ifdef{\public}{\uncover<2->{0 & 0 & -4.69 & 2.15\\}}{ \\}
\ifdef{\public}{\uncover<2->{0 & 0 & 0 & 2.73}}{}
\end{array}\right)$,$Y=\left(\begin{array}{c}
16 \\
8.33 \\
-4.61 \\
1.5
\end{array}\right)$

\end{frame}


\begin{frame}[fragile]

Il ne reste plus qu'à utiliser la méthode décrite précédemment afin de déterminer le résultat.

Le résultat final pour ces valeurs est alors: $X=\left(\begin{array}{c}
0.642642642643 \\ 1.10810810811 \\ 1.23723723724 \\ 0.552552552553\end{array}\right)$

\end{frame}

\section{Codage sous Python}

\begin{frame}[fragile]
\frametitle{Etude 1: Code Python}

\textbf{Recherche du pivot et permutation des lignes:}

\begin{GrayBox}[0.85\textwidth]
\begin{semiverbatim}\small
\ifdef{\public}{\uncover<2->{
def recherche_pivot(A, j0):
    n = len(A) 
    imax = j0 
    for i in range(j0+1, n):
        if abs(A[i][j0]) > abs(A[imax][j0]):
            imax = i
    return imax}}{\vspace{5cm}}
   \end{semiverbatim}
\end{GrayBox}

\begin{GrayBox}[0.85\textwidth]
\begin{semiverbatim}\small
\ifdef{\public}{\uncover<2->{
def permutation(A,i,j):
    t = A[i]
    A[i] = A[j]
    A[j] = t}}{\vspace{5cm}}
   \end{semiverbatim}
\end{GrayBox}

\end{frame}

\begin{frame}[fragile]
\frametitle{Etude 2: Code Python}

\textbf{Transvection:}

\begin{GrayBox}[0.85\textwidth]
\begin{semiverbatim}\small
\ifdef{\public}{\uncover<2->{
def transvection(A,i,j,x):
    # si vecteur
    if type(A[0]) != list:
        A[i] = A[i] + x*A[j]
    else:
        n = len(A[0])
        for k in range(n):
            A[i][k] = A[i][k] + x*A[j][k]}}{\vspace{5cm}}
   \end{semiverbatim}
\end{GrayBox}

\end{frame}

\begin{frame}[fragile]
\frametitle{Etude 3: Code Python}

\textbf{Résolution du système:}

\begin{GrayBox}[0.85\textwidth]
\begin{semiverbatim}\small
\ifdef{\public}{\uncover<2->{
def resolution_systeme(A,b):
    n = len(A)

    for i in range(n-1):
        i0 = recherche_pivot(A,i)
        permutation(A,i,i0)
        permutation(b,i,i0) 
        
        for k in range(i+1,n):
            x = A[k][i]/A[i][i]
            transvection(A,k,i,-x)
            transvection(b,k,i,-x) 
            
    return triangle(A,b)}}{\vspace{5cm}}
   \end{semiverbatim}
\end{GrayBox}
\end{frame}


\end{document}
