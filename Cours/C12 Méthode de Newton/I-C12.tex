\input{../../headers/beamercoursHeadings}

\section{Développement de Taylor} 

\begin{frame}[fragile]
\frametitle{Développement de Taylor}

\begin{minipage}{0.45\linewidth}
Hypothèses de départ:
\begin{itemize}
	\item $f$ est continue sur l'intervalle $[a,b]$,
	\item $f$ s'annule en un unique point de $[a,b]$, que l'on notera $x_{sol}$,
	\item $f$ une fonction dérivable sur $[a,b]$,
	\item $f'$ ne s'annule pas sur $[a,b]$.
\end{itemize}

\end{minipage}\hfill
\begin{minipage}{0.5\linewidth}
 \includegraphics[width=0.9\linewidth]{img/courbe}
\end{minipage}

\begin{defi}
Théorème du développement de Taylor à l'ordre 1. \\
Soit $f$ une fonction définie sur un intervalle $I$ et $u\in I$. Le développement de Taylor à l'ordre 1 de $f$ est donné par:\\
$f(x)=f(u)+f'(u).(x-u)+o(x-u)$
\end{defi}

Géométriquement, lorsqu'on néglige le reste, le développement de Taylor donne l'équation de la tangente en $u$. L'intersection de cette tangente avec l'axe $(O,\overrightarrow{x})$ tend vers $x_{sol}$.

\end{frame}

\section{Démarche} 

\begin{frame}[fragile]
\frametitle{Démarche}

Notons $\Delta(x)$ cette équation, l'abscisse $c$ de l'intersection de la tangente avec l'axe des abscisses est donnée par la résolution de
$\Delta(c)=0() \Leftrightarrow f(u)+f'(u).(c-u) = 0 \Leftrightarrow c=u-\dfrac{f(u)}{f'(u)}$.

\textbf{Pseudo code:}
\begin{enumerate}
 \item Soit $x_0$, un réel dans $I$,
 \item La tangente à $Cf$ au point d'abscisse $x_0$, a pour équation $y=f'(x_0).(x-x_0)+f(x_0)$,
 \item $x_1$ est le réel vérifiant: $f'(x_0).(x_1-x_0)+f(x_0)=0$,
ainsi $x_1=x_0-\dfrac{f(x_0)}{f'(x_0)}$,
 \item Récurrence: le réel $x_n$ étant construit, calculer $x_{n+1}=x_n-\dfrac{f(x_n)}{f'(x_n)}$.
\end{enumerate}

\textbf{Conditions d'arrêt}
\begin{itemize}
 \item lorsque l'écart entre deux termes consécutifs est inférieur à une valeur définie,
 \item lorsque l'image du milieu de notre intervalle par $f$ est inférieur à une valeur définie,
 \item lorsque le nombre d'itérations est supérieur à un nombre défini.
\end{itemize}

Le dernier terme $x_n$ sera une approximation de $x_{sol}$.

\end{frame}
\begin{frame}[fragile]
\frametitle{Code python}

Proposer le code python de la fonction \verb? methode_Newton ?.
\begin{GrayBox}[0.85\textwidth]
\begin{semiverbatim}\small
\ifdef{\public}{\uncover<2->{
def methode_Newton(f,df,a,p) :
    x0=a
    while abs(f(x0))>p : # ou abs(e)>p
        x1=x0-f(x0)/df(x0)
        e=x1-x0
        x0=x1
    return x1}}{\vspace{5cm}}
   \end{semiverbatim}
\end{GrayBox}
\end{frame}

\begin{frame}[fragile]
\frametitle{Exercice}

Déterminer en utilisant Python le point d'intersection des deux fonctions suivantes sur l'intervalle $[0,10]$.

\begin{minipage}{0.45\linewidth}

\begin{itemize}
 \item $f_1(x)=-2.cos(\dfrac{x}{5})+2,1$,
 \item $f_2(x)=0,05.x+0,8$.
\end{itemize}

\includegraphics[width=0.9\linewidth]{img/figure_1}

\end{minipage}\hfill
\begin{minipage}{0.5\linewidth}
\begin{GrayBox}[0.95\textwidth]
\begin{semiverbatim}\small
\ifdef{\public}{\uncover<2->{
def f(x):
    return -2*np.cos(x/5)+2.1-(0.05*x+0.8)    
def df(x):
    return 2/5.*np.sin(x/5)-(0.05)
    
print(methode_Newton(f,df,a,p))
}}{\vspace{4.5cm}}
   \end{semiverbatim}
\end{GrayBox}
\end{minipage}
\end{frame}

\begin{frame}[fragile]
\frametitle{Comparaison des deux méthodes}

Deux critères permettent de déterminer l'efficacité des deux méthodes.

\begin{itemize}
 \item \textbf{Rapidité de convergence:} Dans les hypothèses où les deux méthodes convergent, la méthode de Newton converge plus « rapidement ». Exemple dans le cas précédent \verb? (résultat, nombre d'itérations) ?:
 \begin{GrayBox}[0.4\textwidth]
\begin{verbatimtab}[3]
(5.1059522149400793, 5)
(5.105952024459839, 25)
\end{verbatimtab}
\end{GrayBox}
 \item \textbf{Hypothèses et fonctions étudiées:} Sur ce critère, la méthode est moins contraignante, de plus, il n'est pas nécessaire avec la méthode de la dichotomie de déterminer la dérivée de la fonction.
\end{itemize}

\end{frame}

\end{document}
