\newcommand{\id}{30}
\newcommand{\nom}{Les bases de données}
\newcommand{\sequence}{04}
\newcommand{\num}{16}
\newcommand{\type}{C}
\newcommand{\descrip}{Présentation des bases de données}
