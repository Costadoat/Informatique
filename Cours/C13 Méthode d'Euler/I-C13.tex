\input{../../headers/beamercoursHeadings}

\section{Schéma d'Euler} 

\begin{frame}[fragile]
\frametitle{Schéma d'Euler}

Soit une équation différentielle écrite sous la forme :
\begin{center}
$\forall x \in I, u'(x)=f(x,u(x))$, avec $u(x_0) = y_0$
\end{center}

\begin{obj}
L'objectif de l'utilisation de la méthode d'Euler est de calculer une approximation de la fonction $u$ sur l'intervalle $I$. Cette approximation peut être:
\begin{itemize}
 \item \textbf{locale}: si seule la valeur en un point nous intéresse,
 \item \textbf{globale}: calcul des valeurs de $u$ à intervalles réguliers.
\end{itemize}
\end{obj}

\end{frame}

\begin{frame}[fragile]
\frametitle{Schéma d'Euler}

D'après le calcul du taux d'accroissement:
\begin{center}
$\forall x \in I, u'(x)=\lim\limits_{x\rightarrow a} \dfrac{u(x)-u(a)}{x-a}$
\end{center}

Ce qui peut s'écrire, après un changement de variable:
\begin{center}
$\forall x \in I, u'(x)=\lim\limits_{dx\rightarrow 0} \dfrac{u(x+dx)-u(x)}{dx}$
\end{center}

En reprenant l'équation différentielle du 1er ordre vu précédemment, on obtient:
\begin{center}
$\forall x \in I, f(x,u(x))=\lim\limits_{dx\rightarrow 0} \dfrac{u(x+dx)-u(x)}{dx}$
\end{center}
\end{frame}

\section{Démarche} 

\begin{frame}[fragile]
\frametitle{Démarche}

En discrétisant le problème précédent, il est alors possible d'écrire:
\begin{center}
\ifdef{\public}{\uncover<2->{$f(x,u(x_{i}))=\dfrac{u(x_{i+1})-u(x_{i})}{x_{i+1}-x_{i}}$}}{}
\end{center}

C'est la récurrence suivante qui va nous permettre de résoudre l'équation différentielle:
\begin{center}
\ifdef{\public}{\uncover<2->{$u(x_{i+1})=(x_{i+1}-x_{i})*f(x,u(x_{i}))+u(x_{i})$}}{}
\end{center}

\textbf{Pseudo code:}
\begin{enumerate}
 \item Soit $x$, un vecteur représentant la variable de dérivation,
 \item Soit $y_0$, un réel représentant la condition initiale $u(x_{0})=y_0$
 \item $u(x_{1})=(x_{1}-x_{0})*f(x,u(x_{0}))+u(x_{0})$
 \item Récurrence: le réel $x_n$ étant construit, calculer $u(x_{i+1})=(x_{i+1}-x_{i})*f(x,u(x_{i}))+u(x_{i})$
\end{enumerate}
\end{frame}

\begin{frame}[fragile]
\frametitle{Code python}

Proposer le code python de la fonction \verb? methode_Euler ?.
\begin{GrayBox}[0.85\textwidth]
\begin{semiverbatim}\small
\ifdef{\public}{\uncover<2->{
def methode_euler(F,y0,t):
    y = [0]*len(t)
    y[0] = y0
    for i in range(len(t)-1):
        y[i+1] = y[i]+(t[i+1]-t[i])*F(y[i],t[i])
    return y}}{\vspace{5cm}}
   \end{semiverbatim}
\end{GrayBox}
\end{frame}

\section{Application}

\begin{frame}[fragile]
\frametitle{Application}

On considère une équation différentielle d'ordre 1 avec condition initiale (problème de Cauchy) :
$\left\{\begin{array}{l}
y'(t) = y(t) \\
y(0) = 1
\end{array}\right.$

Résoudre cette équation à l'aide de la méthode d'Euler.
\begin{GrayBox}[0.85\textwidth]
\begin{semiverbatim}\small
\ifdef{\public}{\uncover<2->{
def F(y,t):
    return y

def methode_euler(F,y0,t):
    y = [0]*len(t)
    y[0] = y0
    for i in range(len(t)-1):
        y[i+1] = y[i]+(t[i+1]-t[i])*F(y[i],t[i])
    return y
}}{\vspace{3cm}}
\end{semiverbatim}
\end{GrayBox}
\end{frame}

\begin{frame}[fragile]
\frametitle{Application: Le circuit R.C.}

L'objet de l'étude suivante est le calcul de la tension aux bornes du condensateur dans un circuit R.C. lorsqu'il est en charge.

Donner les équations électriques qui régissent son comportement.
\ifdef{\public}{\uncover<2->{
$\left\{\begin{array}{l}
e(t)=U_R(t)+U_C(t) \\
U_R(t)=R.i(t) \\
i(t)=C.\dfrac{dU_C(t)}{dt}
\end{array}\right.$}}

~\

Donc, \ifdef{\public}{\uncover<2->{
$\left\{\begin{array}{l}
\dfrac{dU_C(t)}{dt}=\dfrac{e(t)-U_C(t)}{R.C} \\
U_C(0)=0
\end{array}\right.$}}{}

Données:
\begin{itemize}
 \item $R=300\Omega$,
 \item $e(t)=6V$,
 \item $C=10mF$.
\end{itemize}
\end{frame}



\begin{frame}[fragile]
\frametitle{Application: Le circuit R.C.}
Résoudre cette équation à l'aide de la méthode d'Euler.
\begin{GrayBox}[0.85\textwidth]
\begin{semiverbatim}\small
\ifdef{\public}{\uncover<2->{
e=6
R=300
C=10*10**(-3)

def F(y,t):
    return (e-y)/(R*C)

def methode_euler(F,y0,t):
    y = [0]*len(t)
    y[0] = y0
    for i in range(len(t)-1):
        y[i+1] = y[i]+(t[i+1]-t[i])*F(y[i],t[i])
    return y}}{\vspace{3cm}}
\end{semiverbatim}
\end{GrayBox}
\end{frame}
\end{document}
