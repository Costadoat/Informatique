\documentclass[10pt,a4paper]{article}
  \usepackage[french]{babel}
  \usepackage[utf8]{inputenc}
  \usepackage[T1]{fontenc}
  \usepackage{xcolor}
  \usepackage[]{graphicx}
  \usepackage{makeidx}
  \usepackage{textcomp}
  \usepackage{amsmath}
  \usepackage{amssymb}
  \usepackage{stmaryrd}
  \usepackage{fancyhdr}
  \usepackage{lettrine}
  \usepackage{calc}
  \usepackage{boxedminipage}
  \usepackage[french,onelanguage, boxruled,linesnumbered]{algorithm2e}
  \usepackage[colorlinks=false,pdftex]{hyperref}
  \usepackage{minted}
  \usepackage{url}
  \usepackage[locale=FR]{siunitx}
  \usepackage{multicol}
  \usepackage{tikz}
  \makeindex

  %\graphicspath{{../Images/}}

%  \renewcommand\listingscaption{Programme}

  %\renewcommand{\thechapter}{\Alph{chapter}}
  \renewcommand{\thesection}{\Roman{section}}
  %\newcommand{\inter}{\vspace{0.5cm}%
  %\noindent }
  %\newcommand{\unite}{\ \textrm}
  \newcommand{\ud}{\mathrm{d}}
  \newcommand{\vect}{\overrightarrow}
  %\newcommand{\ch}{\mathrm{ch}} % cosinus hyperbolique
  %\newcommand{\sh}{\mathrm{sh}} % sinus hyperbolique

  \textwidth 160mm
  \textheight 250mm
  \hoffset=-1.70cm
  \voffset=-1.5cm
  \parindent=0cm

  \pagestyle{fancy}
  \fancyhead[L]{\bfseries {\large PTSI -- Dorian}}
  \fancyhead[C]{\bfseries{{\type} \no \numero}}
  \fancyhead[R]{\bfseries{\large Informatique}}
  \fancyfoot[C]{\thepage}
  \fancyfoot[L]{\footnotesize R. Costadoat, C. Darreye}
  \fancyfoot[R]{\small \today}
  
  \definecolor{bg}{rgb}{0.9,0.9,0.9}
  
  
  % macro Juliette
  
\usepackage{comment}   
\usepackage{amsthm}  
\theoremstyle{definition}
\newtheorem{exercice}{Exercice}
\newtheorem*{rappel}{Rappel}
\newtheorem*{remark}{Remarque}
\newtheorem*{defn}{Définition}
\newtheorem*{ppe}{Propriété}
\newtheorem{solution}{Solution}

\newcounter{num_quest} \setcounter{num_quest}{0}
\newcounter{num_rep} \setcounter{num_rep}{0}
\newcounter{num_cor} \setcounter{num_cor}{0}

\newcommand{\question}[1]{\refstepcounter{num_quest}\par
~\ \\ \parbox[t][][t]{0.15\linewidth}{\textbf{Question \arabic{num_quest}}}\parbox[t][][t]{0.85\linewidth}{#1\label{q\the\value{num_quest}}}\par
~\ \\}

\newcommand{\reponse}[4][1]
{\noindent
\rule{\linewidth}{.5pt}\\
\textbf{Question\ifthenelse{#1>1}{s}{} \multido{}{#1}{%
\refstepcounter{num_rep}\ref{q\the\value{num_rep}} }:} ~\ \\
\ifdef{\public}{#3 ~\ \\ \feuilleDR{#2}}{#4}
}

\newcommand{\cor}
{\refstepcounter{num_cor}
\noindent
\rule{\linewidth}{.5pt}
\textbf{Question \arabic{num_cor}:} \\
}

%\usepackage[a4paper]{geometry}
\geometry{margin={1cm,1.2cm}}
\usepackage[francais]{babel}


\usepackage{multicol}
%\usepackage{nopageno} %pas de numérotation de page
\pagestyle{plain} %numérotation en bas de page, pas d'entête
\usepackage{hyperref}
%\usepackage[latin1]{inputenc}


%%%%%%%%%%%%%%%%%%%%%%%%%%%%%%%%%%%%%%%%%%%%%%%%%%%%%%%%%%%%%%%%%%%%%%%%%%%%%%%%%%%%%

\usepackage[utf8]{inputenc} 
\usepackage{amssymb,amsmath}
\usepackage{stmaryrd}
\usepackage{amsthm}
\usepackage{amscd}
%\usepackage{mathrsfs}
%\usepackage{amsfonts}
%\usepackage[T1]{fontenc}
%\usepackage{theorem}
\usepackage{lscape}
\usepackage{variations}  % pour faire des tableaux de variations
\usepackage{dsfont}
\usepackage{fancyvrb} % pour mettre Verbatim dans une box
\usepackage{moreverb} % pour mettre Verbatim dans une box 
\usepackage{comment} % pour afficher ou non les commentaires, solutions
%\usepackage{slashbox} % pour dans un tabular, couper une case en deux
\usepackage{boxedminipage} % pour cadrer du texte
\usepackage{listings}

% Pour les figures
\usepackage{subfig}
\usepackage{calc} % Pour pouvoir donner des formules dans les d�finitions de longueur
\usepackage{graphicx} % Pour inclure des graphiques 
% Attention : pour inclure des .jpg comme dans l'exemple (ou des .png ou .pdf)
% il faut compiler directement en pdf (commande pdflatex).
% Pour inclure des .eps, il faut compiler avec latex + dvips + ps2pdf.
\usepackage{psfrag}
\usepackage{color}

%%%%%%%%%%%%%%%%%%%%%%%%%%%%%%%%%%%%%%%%%%%%%%%%%%%%%%%%%%%%%%%%%%%%%%%%%%%%%%%%%%%%%

\theoremstyle{definition}
\newtheorem{thm}{Théorème}
%\theorembodyfont{\rmfamily}
\newtheorem*{defn}{Définition}
\newtheorem{exercice}{Exercice}
\newtheorem*{problem}{Problème}
\newtheorem{prop}{Proposition}
\newtheorem{corollaire}{Corollaire}
\newtheorem*{lemme}{Lemme}
\newtheorem*{remark}{Remarque}
\newtheorem*{notation}{Notation}
\newtheorem*{ex}{Exemple}
\newtheorem*{ppe}{Propriété}
\newtheorem*{meth}{Méthode}
\newtheorem*{rappel}{Rappel}
\newtheorem*{voca}{Vocabulaire}
\newtheorem*{solution}{Solution}   

\setlength{\columnseprule}{0.5pt}


%%%%%%%%%%%%%%%%%%%%%%%%%%%%%%%%%%%%%%%%%%%%%%%%%%%%%%%%%%%%%%%%%%%%%%%%%%%%%%%%%%%%%

\newcommand{\bi}{\bigskip}
\newcommand{\dsp}{\displaystyle}
\newcommand{\noi}{\noindent}
\newcommand{\ov}{\overline}
\newcommand{\dsum}{\displaystyle \sum}
\newcommand{\dprod}{\displaystyle \prod}
\newcommand{\dint}{\displaystyle \int}
\newcommand{\dlim}{\displaystyle \lim}

%%%%%%%%%%%%%%%%%%%%%%%%%%%%%%%%%%%%%%%%%%%%%%%%%%%%%%%%%%%%%%%%%%%%%%%%%%%%%%%%%%%%%


%\newcommand{\pgcd}{\mathrm{pgcd}} % pgcd
%\providecommand{\norm}[1]{\lVert#1\rVert} % norme
%\DeclareMathOperator{\Tan}{Tan}  % espace tangent


\newcommand{\N}{\mathbb{N}}
\newcommand{\Z}{\mathbb{Z}}
\newcommand{\Q}{\mathbb{Q}}
\newcommand{\R}{\mathbb{R}}
\newcommand{\C}{\mathbb{C}}
\newcommand{\K}{\mathbb{K}}
\newcommand{\U}{\mathbb{U}}
\newcommand{\Tr}{\text{Tr}\,}
\newcommand{\pg}{\geqslant}
\newcommand{\pp}{\leqslant}
\newcommand{\bul}{\item[$\bullet$]}
\newcommand{\card}{\text{Card}}
\newcommand{\re}{\text{Re}\;}
\newcommand{\im}{\text{Im}\;}
\newcommand{\Ker}{\text{Ker}\;}
\newcommand{\Vect}{\text{Vect}\;}
\newcommand{\rg}{\text{rg}\;}
\newcommand{\TT}{{}^t\!}
\newcommand{\sh}{\text{sh}}
\newcommand{\ch}{\text{ch}}
\newcommand{\Mat}{\text{Mat}}
\usepackage{textcomp}



%%%%%%%%%%%%%%%%%%%%%%%%%%%%%%%%%%%%%%%%%%%%%%%%%%%%%%%%%%%%%%%%%%%%%%%%%%%%%%%%%%%%%%%%%%%%%%%%%%%%%%%%%%%%%%%%%%%%%%%%%%%

\frenchspacing


%%%%%%%%%%%%%%%%%%%%%%%%%%%%%%%%%%%%%%%%%%%%%%%%%%%%%%%%%%%%%%%%%%%%%%%%%%%%%%%%%%%%%%%%%%%%%%%
% Pour une numerotation I, II des sections.
% Pour une numerotation des subsubsections

\setcounter{secnumdepth}{3}
\setcounter{tocdepth}{3}

\renewcommand{\thesection}{\Roman{section})}
\renewcommand{\thesubsection}{\Roman{section}-\arabic{subsection})}
\renewcommand{\thesubsubsection}{\Roman{section}-\arabic{subsection}-\alph{subsubsection}}


%%%%%%%%%%%%%%%%%%%%%%%%%%%%%%%%%%%%%%%%%%%%%%%%%%%%%%%%%%%%
% Pour avoir une enumerate 1) 2)
\frenchbsetup{StandardLists=true}
\usepackage{enumitem}
\setenumerate[1]{label=\arabic*)}


\newcommand{\nom}{Porte conteneur}
\newcommand{\sequence}{03}
\newcommand{\num}{04}
\newcommand{\type}{TD}
\newcommand{\descrip}{Résolution d'un problème en utilisant des méthodes algorithmiques}
\newcommand{\competences}{Alt-C3: Concevoir un algorithme répondant à un problème précisément posé}

\fancyhead[C]{\bfseries{TP \no \num}}

\begin{document}

\begin{center}
{\Large\bf TP \no {\numero} -- \descrip}
\end{center}
\section{Dépassement de capacité}

\paragraph{Question 1:} Écrire dans le tableau suivant le float (simple précision) le plus grand exprimable.

\setlength{\tabcolsep}{0.1cm}
\begin{center}
\begin{scriptsize}
\begin{tabular}{|*{32}{l|}}
\hline
S & \multicolumn{8}{c|}{Exposant} & \multicolumn{23}{c|}{Mantisse} \\
\hline
  &   &  &  &  &  &  &  &  &  &  &  &  &  &  &  &  &  &  &  &  &  &  &  &  &  &  &  &  &  &  &  \\
\hline
\multicolumn{4}{|c|}{} & \multicolumn{4}{c|}{} & \multicolumn{4}{c|}{} & \multicolumn{4}{c|}{} & \multicolumn{4}{c|}{} & \multicolumn{4}{c|}{} & 
\multicolumn{4}{c|}{} & \multicolumn{4}{c|}{} \\
\hline
\end{tabular}
\end{scriptsize}
\end{center}

\paragraph{Question 2:} Calculer sa valeur dans la base décimale.

\paragraph{Question 3:} Calculer la valeur dans la base décimale du float (double précision) le plus grand exprimable.

\paragraph{Question 4:} Entrer dans une console python les commandes suivantes et en déduire si le type de float utilisé est de précision simple ou double.


\begin{verbatim}
>>> 2.0**(1023)
>>> 2.0**(1024)
>>> import sys
>>> sys.float_info.max
\end{verbatim}

\paragraph{Question 5:} Calculer la valeur minimale ($>0$) pour le type de variable utilisé par votre système.

\paragraph{Question 6:} Vérifier votre calcul grâce à la commande suivante.

\begin{verbatim}
>>> import sys
>>> sys.float_info.min
\end{verbatim}

\section{Approximation de calcul}

Lors de la première séance, nous avons remarqué que certains calculs étaient approximés. L'exemple suivant avait été utilisé.

\begin{verbatim}
>>> 1-1/3.-1/3.-1/3.
\end{verbatim}

\paragraph{Question 7:} Calculer le nombre binaire permettant de définir le réel le plus proche de 1/3.

\paragraph{Question 8:} Écrire ce nombre sous la forme suivante $A*2^{exp}$, où $A$ est un entier, et $exp$, l'exposant le plus petit qui permet à $A$ d'être un entier. Vous ferez ce calcul pour un float simple et un float double. (Le calcul sera plus simple si une similitude entre les deux calculs est trouvée)

\paragraph{Question 9:} Déterminer dans ces deux cas la valeur du décimal le plus proche de 1/3, dans le cas des deux types de float.

\paragraph{Question 10:} Calculer alors la valeur de l'opération $1-1/3.-1/3.-1/3.$ en prenant ce nombre approché et comparer cette valeur à celle trouvée en faisant le calcul directement dans la console python.

\ifdef{\public}{\end{document}}{}

\newpage

\pagestyle{correction}

\section{Correction}

\subsection{Dépassement de capacité}

\paragraph{Question 1:}

\setlength{\tabcolsep}{0.1cm}
\begin{center}
\begin{scriptsize}
\begin{tabular}{|*{32}{l|}}
\hline
S & \multicolumn{8}{c|}{Exposant} & \multicolumn{23}{c|}{Mantisse} \\
\hline
0 & 1 & 1 & 1 & 1 & 1 & 1 & 1 & 0 & 1 & 1 & 1 & 1 & 1 & 1 & 1 & 1 & 1 & 1 & 1 & 1 & 1 & 1 & 1 & 1 & 1 & 1 & 1 & 1 & 1 & 1 & 1 \\
\hline
\multicolumn{4}{|c|}{7} & \multicolumn{4}{c|}{F} & \multicolumn{4}{c|}{7} & \multicolumn{4}{c|}{F} & \multicolumn{4}{c|}{F} & \multicolumn{4}{c|}{F} & 
\multicolumn{4}{c|}{F} & \multicolumn{4}{c|}{F} \\
\hline
\end{tabular}
\end{scriptsize}
\end{center}

\paragraph{Question 2:}

\begin{center}
	\begin{tabular}{|c c c|}
	\hline
	Signe & Exposant & Mantisse \\
	\hline
	0 & $\underbrace{111...110}_{8 bits}$ & $\underbrace{111...111}_{23 bits}$ \\
	\hline
	\end{tabular}
\end{center}

\begin{itemize}
 \item Exposant: $2^8-2=254$, exposant simple $254-127=127$,
 \item Le chiffre à calculer est donc $\underbrace{111...111}_{24 bits}\underbrace{000...000}_{104 bits}$,
 \item Ce qui donne en décimal $(2^{24}-1)*2^{104}=3.4028234663852886.10^{38}$
\end{itemize}

\paragraph{Question 3:}

\begin{center}
	\begin{tabular}{|c c c|}
	\hline
	Signe & Exposant & Mantisse \\
	\hline
	0 & $\underbrace{111...110}_{11 bits}$ & $\underbrace{111...111}_{52 bits}$ \\
	\hline
	\end{tabular}
\end{center}

\begin{itemize}
 \item Exposant: $2^{11}-2=2046$, exposant simple $2046-1023=1023$,
 \item Le chiffre à calculer est donc $\underbrace{111...111}_{53 bits}\underbrace{000...000}_{971 bits}$,
 \item Ce qui donne en décimal $(2^{53}-1)*2^{971}=1.7976931348623157.10^{308}$
\end{itemize}

\paragraph{Question 5:}

\begin{center}
	\begin{tabular}{|c c c|}
	\hline
	Signe & Exposant & Mantisse \\
	\hline
	0 & $\underbrace{000...001}_{11 bits}$ & $\underbrace{000...000}_{52 bits}$ \\
	\hline
	\end{tabular}
\end{center}

\begin{itemize}
 \item Exposant: $1$, exposant simple $1-1023=-1022$,
 \item Le chiffre à calculer est donc $1,\underbrace{000...000}_{-1022 bits}$,
 \item Ce qui donne en décimal $1*2^{-1022}=2.2250738585072014.10^{-308}$
\end{itemize}

\paragraph{Question 7:}

\begin{tabular}{c c c c c c c c c c}
0,33.. & x & 2 & = & 0,66.. & = &  0 & + & 0,66.. \\
0,66.. & x & 2 & = & 1,33.. & = &  1 & + & 0,33.. \\
0,33.. & x & 2 & = & 0,66.. & = &  0 & + & 0,66.. \\
0,66.. & x & 2 & = & 1,33.. & = &  1 & + & 0,33.. \\
0,33.. & x & 2 & = & 0,66.. & = &  0 & + & 0,66.. \\
...
\end{tabular}

On remarque un récurrence dans l'écriture du $0,33_{10}$ en binaire: $0,33.._{10}=0,01010.._2$

\paragraph{Question 8:}
\begin{itemize}
 \item 32 bits: $1,\underbrace{0101..}_{23 bits}*2^{-2}$ $=\underbrace{10101010....}_{24 bits}*2^{-25}$
 \item 64 bits:$1,\underbrace{0101..}_{52 bits}*2^{-2}=\underbrace{10101010....}_{53 bits}*2^{-54}$
\end{itemize}

Le passage de $\underbrace{10101010...0}_{24 bits}$ à $\underbrace{10101010...1}_{23 bits}$ se fait par un décalage des bits vers la droite ce qui revient à diviser par deux.

Le passage de $\underbrace{10101010...1}_{53 bits}$ à $\underbrace{10101010...0}_{54 bits}$ se fait par un décalage des bits vers la gauche ce qui revient à multiplier par deux.

\paragraph{Question 9:}
\begin{itemize}
 \item 32 bits: $\underbrace{10101010...0}_{24 bits}=\underbrace{11111111....}_{24 bits}-\underbrace{10101010...1}_{23 bits}$
 \item 64 bits: $\underbrace{10101010...1}_{53 bits}=\underbrace{11111111....}_{54 bits}-\underbrace{10101010...0}_{54 bits}$
\end{itemize}

Le calcul de $\underbrace{11111111....}_{24 bits}$ se fait en ajoutant 1.

\begin{itemize}
 \item 32 bits: $\underbrace{10101010...0}_{24 bits}=(2^{24}-1)-\dfrac{\underbrace{10101010...0}_{24 bits}}{2}$
 \item 64 bits: $\underbrace{10101010...1}_{53 bits}=(2^{54}-1)-2*\underbrace{10101010...1}_{53 bits}$
\end{itemize}

Regroupement des $\underbrace{10101010...0}_{24 bits}$ et $\underbrace{10101010...1}_{53 bits}$.

\begin{itemize}
 \item 32 bits: $\underbrace{10101010...0}_{24 bits}=\dfrac{2}{3}.(2^{24}-1)$
 \item 64 bits: $\underbrace{10101010...1}_{53 bits}=\dfrac{1}{3}.(2^{54}-1)$
\end{itemize}

Le résultat est alors.

\begin{itemize}
 \item 32 bits: $\underbrace{10101010...0}_{24 bits}*2^{-25}=\dfrac{2}{3}.(2^{24}-1)*2^{-25}=$
 \item 64 bits: $\underbrace{10101010...1}_{53 bits}*2^{-54}=\dfrac{1}{3}.(2^{54}-1)*2^{-54}=$
\end{itemize}

\paragraph{Question 10:}
\begin{itemize}
 \item 32 bits: $1-1/3.-1/3.-1/3.=1-2.(2^{24}-1)*2^{-25}=1-(2^{24}-1)*2^{-24}=2^{-24}=5.960464477539063.10^{-8}$
 \item 64 bits: $1-1/3.-1/3.-1/3.=1-1.(2^{54}-1)*2^{-54}=1-(2^{54}-1)*2^{-54}=2^{-54}=5,55111512312578.10^{-17}$
\end{itemize}

\end{document}
