\documentclass[10pt,a4paper]{article}
  \usepackage[french]{babel}
  \usepackage[utf8]{inputenc}
  \usepackage[T1]{fontenc}
  \usepackage{xcolor}
  \usepackage[]{graphicx}
  \usepackage{makeidx}
  \usepackage{textcomp}
  \usepackage{amsmath}
  \usepackage{amssymb}
  \usepackage{stmaryrd}
  \usepackage{fancyhdr}
  \usepackage{lettrine}
  \usepackage{calc}
  \usepackage{boxedminipage}
  \usepackage[french,onelanguage, boxruled,linesnumbered]{algorithm2e}
  \usepackage[colorlinks=false,pdftex]{hyperref}
  \usepackage{minted}
  \usepackage{url}
  \usepackage[locale=FR]{siunitx}
  \usepackage{multicol}
  \usepackage{tikz}
  \makeindex

  %\graphicspath{{../Images/}}

%  \renewcommand\listingscaption{Programme}

  %\renewcommand{\thechapter}{\Alph{chapter}}
  \renewcommand{\thesection}{\Roman{section}}
  %\newcommand{\inter}{\vspace{0.5cm}%
  %\noindent }
  %\newcommand{\unite}{\ \textrm}
  \newcommand{\ud}{\mathrm{d}}
  \newcommand{\vect}{\overrightarrow}
  %\newcommand{\ch}{\mathrm{ch}} % cosinus hyperbolique
  %\newcommand{\sh}{\mathrm{sh}} % sinus hyperbolique

  \textwidth 160mm
  \textheight 250mm
  \hoffset=-1.70cm
  \voffset=-1.5cm
  \parindent=0cm

  \pagestyle{fancy}
  \fancyhead[L]{\bfseries {\large PTSI -- Dorian}}
  \fancyhead[C]{\bfseries{{\type} \no \numero}}
  \fancyhead[R]{\bfseries{\large Informatique}}
  \fancyfoot[C]{\thepage}
  \fancyfoot[L]{\footnotesize R. Costadoat, C. Darreye}
  \fancyfoot[R]{\small \today}
  
  \definecolor{bg}{rgb}{0.9,0.9,0.9}
  
  
  % macro Juliette
  
\usepackage{comment}   
\usepackage{amsthm}  
\theoremstyle{definition}
\newtheorem{exercice}{Exercice}
\newtheorem*{rappel}{Rappel}
\newtheorem*{remark}{Remarque}
\newtheorem*{defn}{Définition}
\newtheorem*{ppe}{Propriété}
\newtheorem{solution}{Solution}

\newcounter{num_quest} \setcounter{num_quest}{0}
\newcounter{num_rep} \setcounter{num_rep}{0}
\newcounter{num_cor} \setcounter{num_cor}{0}

\newcommand{\question}[1]{\refstepcounter{num_quest}\par
~\ \\ \parbox[t][][t]{0.15\linewidth}{\textbf{Question \arabic{num_quest}}}\parbox[t][][t]{0.85\linewidth}{#1\label{q\the\value{num_quest}}}\par
~\ \\}

\newcommand{\reponse}[4][1]
{\noindent
\rule{\linewidth}{.5pt}\\
\textbf{Question\ifthenelse{#1>1}{s}{} \multido{}{#1}{%
\refstepcounter{num_rep}\ref{q\the\value{num_rep}} }:} ~\ \\
\ifdef{\public}{#3 ~\ \\ \feuilleDR{#2}}{#4}
}

\newcommand{\cor}
{\refstepcounter{num_cor}
\noindent
\rule{\linewidth}{.5pt}
\textbf{Question \arabic{num_cor}:} \\
}

  
  \textheight 250mm
  \pagestyle{fancy}
  \fancyfoot[C]{\thepage}
  \fancyhead[LO,LE]{\bfseries {\large PTSI -- Dorian}}
  \fancyhead[RO,RE]{\bfseries{\large Informatique}}
  \fancyhead[CO,CE]{\bfseries{Corrigé des compléments au TP \no 5}}

  \linespread{1}
  
\begin{document}

% En-tête et titre
%      {\Large\bf TD \no 0}\\
%      \vspace{0.5cm}

\begin{center}
{\Large\bf TP Tests, listes et recherches dans une liste\\
Compléments}
\end{center}

%%%%%%%%%%%%%%%%%%%%%%%%%%%%%%%%%%%
%%%%%%%%%%%%%%%%%%%%%%%%%%%%%%%%%%%
\ifdef{\public}{\end{document}}{}
\cleardoublepage
\renewcommand{\type}{Correction TP}
\setcounter{section}{0}
\begin{center}
{\Large\bf {\type} \no {\num} -- \descrip}
\end{center}
%%%%%%%%%%%%%%%%%%%%%%%%%%%%%%%%%%%
%%%%%%%%%%%%%%%%%%%%%%%%%%%%%%%%%%%


\section{Vérification d'un carré magique}
%Vuibert IPT page 88

\begin{enumerate}
\setcounter{enumi}{1}
\item Code de la fonction de vérification.

\begin{listing}
\begin{minted}[linenos,frame=lines]{python}
def verif_carre_1(T):
  n = len(T) #nombre de lignes/colonnes
  ## DETERMINATION DES SOMMES
  resu_somme=[]
  # Boucles sur les lignes
  for i in range(n):
    somme=0
    for j in range(n):
      somme = somme + T[i][j]
    resu_somme.append(somme)
  # Boucles sur les colonnes
  for j in range(n):
    somme=0
    for i in range(n):
      somme = somme + T[i][j]
    resu_somme.append(somme)
  # diagonale 1
  somme=0
  for i in range(n):
    somme = somme + T[i][i]
  resu_somme.append(somme)
  # diagonale 2
  somme=0
  for i in range(n):
    somme = somme + T[i][n-1-i]
  resu_somme.append(somme)

  ##VERIFICATION DU CARRE
  #On verifie que chaque element de la liste est
  #egal au premier element de la liste
  for i in range(1,len(resu_somme)):
    if resu_somme[i] != resu_somme[0]:
      return False # on sort de la fonction en renvoyant False
  # sinon le carre est magique on renvoie True
  return True
\end{minted}
\caption{Programme naïf.}
\label{prog:factorielle}
\end{listing}

\item Chaque calcul de somme nécessite $n$ additions\footnote{On pourrait réduire d'une addition en initialisant \texttt{somme} à \texttt{T[0][0]} au lieu de 0 et en parcourant \texttt{range(1,n)}.}. Il faut faire $2n+2$ calculs de somme ($n$ lignes, $n$ colonnes, 2 diagonales). 

Pour les comparaisons, dans le meilleur des cas, la première comparaison conduit à un carré non magique. 
Dans le meilleur des cas, le nombre d'opérations est donc $(2n+2)n+1=2n^2 + 2n + 1$. La complexité est quadratique.

Dans le pire des cas, c'est-à-dire quand seule la dernière somme est incorrecte ou quand le carré est magique, on réalise $2n+1$ comparaisons. La complexité dans le pire des cas est donc $(2n+2)n+2n+1=2n^2+4n+1$ et est donc quadratique également.

\newpage
\item La fonction n'est pas optimale car on calcule dans tous les cas toutes les sommes avant de faire les comparaisons. Une meilleure approche consisterait à vérifier après chaque calcul de somme si le résultat est identique au résultat précédent ou non. Si ce n'est pas le cas, on sort immédiatement de la fonction, sinon on continue.

La complexité dans le pire des cas est identique à la fonction précédente et atteinte quand le carré est magique ou que l'erreur est sur la dernière comparaison. La seule différence est une comparaison supplémentaire inutile à la première itération de la première boucle où on compare le premier et le dernier élément d'une liste ne contenant à cette itération qu'un seul élément.

La complexité dans le meilleur des cas est atteinte lorsque les deux premières sommes sont différentes. On calcule donc deux sommes, soient $2n$ additions, et on fait deux comparaisons. Le nombre d'opération est donc $2n+2$ et la complexité est linéaire Cet algorithme est donc plus performant.

\begin{listing}[htp]
\begin{minted}[linenos,frame=lines]{python}
def verif_carre_2(T):
  n = len(T) #nombre de lignes/colonnes
  ## DETERMINATION DES SOMMES
  resu_somme=[]
  # Boucles sur les lignes
  for i in range(n):
    somme=0
    for j in range(n):
      somme = somme + T[i][j]
    resu_somme.append(somme)
    # comparaison du dernier element avec le dernier
    if resu_somme[i] != resu_somme[0]:
      return False # on sort de la fonction en renvoyant False
  # Boucles sur les colonnes
  for j in range(n):
    somme=0
    for i in range(n):
      somme = somme + T[i][j]
    resu_somme.append(somme)
    # comparaison du dernier element avec le dernier
    if resu_somme[i] != resu_somme[0]:
      return False # on sort de la fonction en renvoyant False
  # diagonale 1
  somme=0
  for i in range(n):
    somme = somme + T[i][i]
  resu_somme.append(somme)
  # comparaison du dernier element avec le dernier
  if resu_somme[i] != resu_somme[0]:
    return False # on sort de la fonction en renvoyant False
  # diagonale 2
  somme=0
  # comparaison du dernier element avec le dernier
  if resu_somme[i] != resu_somme[0]:
    return False # on sort de la fonction en renvoyant False
  for i in range(n):
    somme = somme + T[i][n-1-i]
  resu_somme.append(somme)
# sinon le carre est magique, on renvoie True
  return True
\end{minted}
\caption{Programme optimisé.}
\label{prog:factorielle}
\end{listing}

\end{enumerate}

\newpage
\section{Calendrier grégorien}
%Vuibert IPT page 108

\begin{enumerate}
 \item Sans tenir compte des années bissextiles :
 \begin{itemize}
  \item On compte 365 jours pour toutes les années entièrement écoulées. 
  \item On ajoute le nombre de jours des mois entièrement écoulés dans l'année en cours. 
  \item On ajoute le nombre de jours dans le mois en cours.                                                                                                                                                                                                           \end{itemize}

\begin{minted}[linenos,frame=lines]{python}
def nombre_de_jours_annees_classiques(jour,mois,annee):
  m = [31,28,31,30,31,30,31,31,30,31,30,31]
  nbjours = (annee-1600)*365 # nombre de jours ecoules pour les annees entieres
  for i in range(mois-1): # mois numerotes de 1 a 12 ; liste indexee de 0 a 11
    nbjours = nbjours + m[i]  # nombre de jours ecoules pour les mois termines
  nbjours = nbjours + jour - 1 # nombre de jours ecoules pour le mois en cours
  return nbjours
\end{minted}

 \item Détermination des années bissextiles.

\begin{minted}[linenos,frame=lines]{python}
def bissextile(annee):
  if annee % 4 == 0:
    if annee % 100 == 0:
      if annee % 400 == 0: 
        bissex = True # annee multiple de 400
      else:
        bissex = False # annee multiple de 4 et de 100 mais pas de 400
    else:
      bissex = True # annee multiple de 4 mais pas de 100
  else:
    bissex = False # annee non multiple de 4
  return bissex
\end{minted}

 \item En tenant compte des années bissextiles :
 \begin{itemize}
  \item On compte 365 ou 366 jours pour toutes les années entièrement écoulées selon qu'elles sont bissextiles ou non. 
  \item On ajoute le nombre de jours des mois classiques entièrement écoulés dans l'année en cours. On ajoute un jour si le mois de février de l'année en cours est terminé et que l'année en cours est bissextile.
  \item On ajoute le nombre de jours dans le mois en cours.
  \end{itemize}

\begin{minted}[linenos,frame=lines]{python}
def nombre_de_jours_toutes_annees(jour,mois,annee):
  m = [31,28,31,30,31,30,31,31,30,31,30,31]
  nbjours = 0
  for an in range(1600,annee): # pour tous les ans entierement ecoules
    if bissextile(an):
      nbjours = nbjours + 366 # si annee bissextile on ajoute 366
    else:
      nbjours = nbjours + 365 # sinon on ajoute 365
  for i in range(mois-1): 
    nbjours = nbjours + m[i] # nombre de jours ecoules pour les mois termines
  if (mois > 3 or mois == 3) and bissextile(annee):
    nbjours = nbjours + 1 # on ajoute 1 si fevrier est termine et
			  # que l'annee en cours bissextile
  nbjours = nbjours + jour - 1 # nombre de jours ecoules pour le mois en cours
  return nbjours  
\end{minted}
 
\newpage
 \item Les semaines font toujours 7 jours. On détermine donc uniquement le nombre de jours écoulés depuis le 1\up{er} janvier 2001 modulo 7. Si la date testée est postérieure au 1\up{er} janvier 2001, le reste est négatif ce qui ne pose pas de souci à python qui gère les indices négatifs dans les listes (-1 étant l'indice du dernier élément, -2 celui de l'avant -dernier, etc.).

\begin{minted}[linenos,frame=lines]{python}
jours = ["lundi","mardi","mercredi","jeudi","vendredi","samedi","dimanche"]

print(jours[
( nombre_de_jours_toutes_annees(1,5,2040) 
- nombre_de_jours_toutes_annees(1,1,2001) ) % 7
])

print(jours[
( nombre_de_jours_toutes_annees(14,7,1789) 
- nombre_de_jours_toutes_annees(1,1,2001) ) % 7
])
\end{minted}

\end{enumerate}

\end{document}
