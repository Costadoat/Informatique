\newcommand{\nom}{Production Éolienne/Solaire}
\newcommand{\sequence}{1}
\newcommand{\numero}{02}
\newcommand{\type}{DS}
\newcommand{\descrip}{Production Éolienne/Solaire}


\documentclass[10pt,a4paper]{article}
  \usepackage[french]{babel}
  \usepackage[utf8]{inputenc}
  \usepackage[T1]{fontenc}
  \usepackage{xcolor}
  \usepackage[]{graphicx}
  \usepackage{makeidx}
  \usepackage{textcomp}
  \usepackage{amsmath}
  \usepackage{amssymb}
  \usepackage{stmaryrd}
  \usepackage{fancyhdr}
  \usepackage{lettrine}
  \usepackage{calc}
  \usepackage{boxedminipage}
  \usepackage[french,onelanguage, boxruled,linesnumbered]{algorithm2e}
  \usepackage[colorlinks=false,pdftex]{hyperref}
  \usepackage{minted}
  \usepackage{url}
  %\usepackage[locale=FR]{siunitx}
  \usepackage{multicol}
  \makeindex

  %\graphicspath{{../Images/}}

  \renewcommand\listingscaption{Programme}

  %\renewcommand{\thechapter}{\Alph{chapter}}
  \renewcommand{\thesection}{\Roman{section}}
  %\newcommand{\inter}{\vspace{0.5cm}%
  %\noindent }
  %\newcommand{\unite}{\ \textrm}
  \newcommand{\ud}{\mathrm{d}}
  \newcommand{\vect}{\overrightarrow}
  %\newcommand{\ch}{\mathrm{ch}} % cosinus hyperbolique
  %\newcommand{\sh}{\mathrm{sh}} % sinus hyperbolique

  \textwidth 160mm
  \textheight 250mm
  \hoffset=-1.70cm
  \voffset=-1.5cm
  \parindent=0cm

  \pagestyle{fancy}
  \fancyhead[L]{\bfseries {\large PTSI -- Dorian}}
  \fancyhead[C]{\bfseries{{\type} \no \numero}}
  \fancyhead[R]{\bfseries{\large Informatique}}
  \fancyfoot[C]{\thepage}
  \fancyfoot[L]{\footnotesize R. Costadoat, J. Genzmer, W. Robert}
  \fancyfoot[R]{\small \today}
  
  \definecolor{bg}{rgb}{0.5,0.5,0.5}
  \definecolor{danger}{RGB}{217,83,79}
  
  \fancypagestyle{correction}{%
  \fancyhf{}
  \lhead{\colorbox{danger}{\begin{minipage}{0.65\paperwidth} \textcolor{white}{\textbf{Correction}} \end{minipage}} }
  \rhead{\includegraphics[width=2cm]{../../img/logo}}
  \lfoot{Juliette Genzmer, Willie Robert, Renaud Costadoat}
  \rfoot{\colorbox{danger}{\begin{minipage}{0.6\paperwidth} \begin{flushright}\textcolor{white}{\textbf{Correction}}\end{flushright} \end{minipage}} }}

  
  % macro Juliette
  
\usepackage{comment}   
\usepackage{amsthm}  
\theoremstyle{definition}
\newtheorem{exercice}{Exercice}
\newtheorem*{rappel}{Rappel}
\newtheorem*{remark}{Remarque}
\newtheorem*{defn}{Définition}
\newtheorem*{ppe}{Propriété}
\newtheorem{solution}{Solution}


\begin{document}
 
On donne en annexe le tableau des résultats actuels des clubs de TOP 14, la première division de rugby.

\paragraph{Objectif}

L'objectif de ce TD est de coder un script qui permettra de :
\begin{itemize}
 \item Calculer le nombre de points de chaque équipe,
 \item Calculer le nombre de matchs joués par chaque équipe,
 \item Classer les équipes en suivant les deux premières règles de la FFR
\end{itemize}

On rappelle quelques points du règlement:

\begin{itemize}
 \item Compte des points :
	\begin{itemize}
 		\item La victoire = 4 points,
 		\item Le match nul = 2 points,
 		\item La défaite = 0 point,
 		\item Le bonus = 1 point.
\end{itemize} 
 \item Bonus offensif (bo): Un point bonus est attribué à une équipe inscrivant 3 essais de plus que l'adversaire,
 \item Bonus défensif (bd): Un point bonus est attribué à une équipe perdant de 5 points ou moins,
 \item Si deux ou plusieurs équipes se trouvent à égalité à l'issue de la phase régulière, leur classement sera établi en tenant compte des facteurs ci-après.
 \item Chaque facteur n'est à prendre en compte que si celui qui le précède n'a pas permis de départager les équipes concernées et d'établir ce classement :
\begin{itemize}
 \item Nombre de points terrain obtenus sur l'ensemble des rencontres ayant opposé entre elles les équipes concernées (y compris le cas échéant points de bonus et points de pénalisation),
 \item Goal-average sur l'ensemble des rencontres de la compétition,
 \item Goal-average sur l'ensemble des rencontres ayant opposé entre elles les équipes restant concernées,
 \item Plus grande différence entre le nombre d'essais marqués et concédés sur l'ensemble des rencontres ayant opposé entre elles les équipes restant concernées,
 \item Plus grande différence entre le nombre d'essais marqués et concédés dans toutes les rencontres de la compétition,
 \item Plus grand nombre de points marqués dans toutes les rencontres de la compétition,
 \item Plus grand nombre d'essais marqués dans toutes les rencontres de la compétition,
 \item Nombre de forfaits n'ayant pas entraîné de forfait général de la compétition,
 \item Place obtenue la saison précédente dans le Championnat de France.
\end{itemize} 
\end{itemize} 
 

\clearpage


\ifdef{\public}{\end{document}}{}

\inputminted{python}{Code/I-TD11.py}


\end{document} 
 
