\newcommand{\nom}{Production Éolienne/Solaire}
\newcommand{\sequence}{1}
\newcommand{\numero}{02}
\newcommand{\type}{DS}
\newcommand{\descrip}{Production Éolienne/Solaire}


\documentclass[10pt,a4paper]{article}
  \usepackage[french]{babel}
  \usepackage[utf8]{inputenc}
  \usepackage[T1]{fontenc}
  \usepackage{xcolor}
  \usepackage[]{graphicx}
  \usepackage{makeidx}
  \usepackage{textcomp}
  \usepackage{amsmath}
  \usepackage{amssymb}
  \usepackage{stmaryrd}
  \usepackage{fancyhdr}
  \usepackage{lettrine}
  \usepackage{calc}
  \usepackage{boxedminipage}
  \usepackage[french,onelanguage, boxruled,linesnumbered]{algorithm2e}
  \usepackage[colorlinks=false,pdftex]{hyperref}
  \usepackage{minted}
  \usepackage{url}
  %\usepackage[locale=FR]{siunitx}
  \usepackage{multicol}
  \makeindex

  %\graphicspath{{../Images/}}

  \renewcommand\listingscaption{Programme}

  %\renewcommand{\thechapter}{\Alph{chapter}}
  \renewcommand{\thesection}{\Roman{section}}
  %\newcommand{\inter}{\vspace{0.5cm}%
  %\noindent }
  %\newcommand{\unite}{\ \textrm}
  \newcommand{\ud}{\mathrm{d}}
  \newcommand{\vect}{\overrightarrow}
  %\newcommand{\ch}{\mathrm{ch}} % cosinus hyperbolique
  %\newcommand{\sh}{\mathrm{sh}} % sinus hyperbolique

  \textwidth 160mm
  \textheight 250mm
  \hoffset=-1.70cm
  \voffset=-1.5cm
  \parindent=0cm

  \pagestyle{fancy}
  \fancyhead[L]{\bfseries {\large PTSI -- Dorian}}
  \fancyhead[C]{\bfseries{{\type} \no \numero}}
  \fancyhead[R]{\bfseries{\large Informatique}}
  \fancyfoot[C]{\thepage}
  \fancyfoot[L]{\footnotesize R. Costadoat, J. Genzmer, W. Robert}
  \fancyfoot[R]{\small \today}
  
  \definecolor{bg}{rgb}{0.5,0.5,0.5}
  \definecolor{danger}{RGB}{217,83,79}
  
  \fancypagestyle{correction}{%
  \fancyhf{}
  \lhead{\colorbox{danger}{\begin{minipage}{0.65\paperwidth} \textcolor{white}{\textbf{Correction}} \end{minipage}} }
  \rhead{\includegraphics[width=2cm]{../../img/logo}}
  \lfoot{Juliette Genzmer, Willie Robert, Renaud Costadoat}
  \rfoot{\colorbox{danger}{\begin{minipage}{0.6\paperwidth} \begin{flushright}\textcolor{white}{\textbf{Correction}}\end{flushright} \end{minipage}} }}

  
  % macro Juliette
  
\usepackage{comment}   
\usepackage{amsthm}  
\theoremstyle{definition}
\newtheorem{exercice}{Exercice}
\newtheorem*{rappel}{Rappel}
\newtheorem*{remark}{Remarque}
\newtheorem*{defn}{Définition}
\newtheorem*{ppe}{Propriété}
\newtheorem{solution}{Solution}


\begin{document}
 
\paragraph{Objectif}

On donne une série d'image au format png sur lesquelles sont dessinés des carrés. L'objectif est de déterminer l'orientation de ces carrés.

\begin{center}
\renewcommand{\verbatimtabsize}{3}
\lstset{frameround=tttt,showstringspaces=false}
\begin{lstlisting}[linewidth=0.85\linewidth,frame=trbl,backgroundcolor=\color{bleuc},rulecolor=\color{bleuf},numbers=right,language=python,breaklines]
# -*- coding: utf-8 -*-
"""

@author: Renaud
"""

from math import asin, pi, cos, sin
from PIL import Image, ImageDraw, ImageFont

    
for image in range(4):
    im = Image.open(str(image)+'.png')
    pix = im.load()
    size=im.size
    print pix[size[0]/2.,size[1]/2.]
    largeur0=int(size[0]/2.)
    longueur0=int(size[1]/2.)
    e=20
    for i in range(largeur0-e,largeur0+e):
        for j in range(longueur0-e,longueur0+e):
            pix[i,j]=(255,0,0)   
    im.save('Copie '+str(image)+'.png')
\end{lstlisting}
\end{center}




\paragraph{Question 1:}

Proposer une fonction \verb?detectcarre(pix)? qui permette de retourner les coordonnées  des angles \verb?hg,hd,bg,bd? du carré.

\paragraph{Question 1:}

Proposer une fonction \verb?taille(hg,hd,bg,bd)? qui permette de retourner les \verb?(largeur,hauteur)? du carré.

\paragraph{Question 3:}

Proposer une fonction \verb?orientation(hg,hd,bg,bd)? qui permette de retourner la rotation du carré.

\end{document} 
 
