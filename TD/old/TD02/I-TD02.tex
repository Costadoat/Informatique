\newcommand{\nom}{Application de la m�thode de Newton}
\newcommand{\sequence}{03}
\newcommand{\num}{02}
\newcommand{\type}{TD}
\newcommand{\descrip}{Utilisation de la m�thode de Newton afin de r�soudre des exercices}
\newcommand{\competences}{Ing-C5: Probl�me stationnaire � une dimension, lin�aire ou non conduisant � la r�solution approch�e d'une �quation alg�brique ou transcendante. M�thode de dichotomie, m�thode de Newton.}

\documentclass[10pt,a4paper]{article}
  \usepackage[french]{babel}
  \usepackage[utf8]{inputenc}
  \usepackage[T1]{fontenc}
  \usepackage{xcolor}
  \usepackage[]{graphicx}
  \usepackage{makeidx}
  \usepackage{textcomp}
  \usepackage{amsmath}
  \usepackage{amssymb}
  \usepackage{stmaryrd}
  \usepackage{fancyhdr}
  \usepackage{lettrine}
  \usepackage{calc}
  \usepackage{boxedminipage}
  \usepackage[french,onelanguage, boxruled,linesnumbered]{algorithm2e}
  \usepackage[colorlinks=false,pdftex]{hyperref}
  \usepackage{minted}
  \usepackage{url}
  %\usepackage[locale=FR]{siunitx}
  \usepackage{multicol}
  \makeindex

  %\graphicspath{{../Images/}}

  \renewcommand\listingscaption{Programme}

  %\renewcommand{\thechapter}{\Alph{chapter}}
  \renewcommand{\thesection}{\Roman{section}}
  %\newcommand{\inter}{\vspace{0.5cm}%
  %\noindent }
  %\newcommand{\unite}{\ \textrm}
  \newcommand{\ud}{\mathrm{d}}
  \newcommand{\vect}{\overrightarrow}
  %\newcommand{\ch}{\mathrm{ch}} % cosinus hyperbolique
  %\newcommand{\sh}{\mathrm{sh}} % sinus hyperbolique

  \textwidth 160mm
  \textheight 250mm
  \hoffset=-1.70cm
  \voffset=-1.5cm
  \parindent=0cm

  \pagestyle{fancy}
  \fancyhead[L]{\bfseries {\large PTSI -- Dorian}}
  \fancyhead[C]{\bfseries{{\type} \no \numero}}
  \fancyhead[R]{\bfseries{\large Informatique}}
  \fancyfoot[C]{\thepage}
  \fancyfoot[L]{\footnotesize R. Costadoat, J. Genzmer, W. Robert}
  \fancyfoot[R]{\small \today}
  
  \definecolor{bg}{rgb}{0.5,0.5,0.5}
  \definecolor{danger}{RGB}{217,83,79}
  
  \fancypagestyle{correction}{%
  \fancyhf{}
  \lhead{\colorbox{danger}{\begin{minipage}{0.65\paperwidth} \textcolor{white}{\textbf{Correction}} \end{minipage}} }
  \rhead{\includegraphics[width=2cm]{../../img/logo}}
  \lfoot{Juliette Genzmer, Willie Robert, Renaud Costadoat}
  \rfoot{\colorbox{danger}{\begin{minipage}{0.6\paperwidth} \begin{flushright}\textcolor{white}{\textbf{Correction}}\end{flushright} \end{minipage}} }}

  
  % macro Juliette
  
\usepackage{comment}   
\usepackage{amsthm}  
\theoremstyle{definition}
\newtheorem{exercice}{Exercice}
\newtheorem*{rappel}{Rappel}
\newtheorem*{remark}{Remarque}
\newtheorem*{defn}{Définition}
\newtheorem*{ppe}{Propriété}
\newtheorem{solution}{Solution}


\section{D�terminer la racine d'une �quation}

Soit une fonction $f(x)=0.07*x^3-x^2+6*x-1$.

Il existe une racine pour ce polyn�me dans l'intervalle $\left[0,1\right]$.

\paragraph{Question 1:} D�terminer la racine de ce polyn�me dans l'intervalle $\left[0,1\right]$ gr�ce � la m�thode de la dichotomie.

\paragraph{Question 2:} Montrer que la m�thode de Newton peut �tre utilis�e pour d�terminer cette racine.

\paragraph{Question 3:} D�terminer la racine de ce polyn�me dans l'intervalle $\left[0,1\right]$ gr�ce � la m�thode de Newton.

\section{Cr�ation de la fonction racine carr�}

\paragraph{Question 4:} D�terminer la fonction $f(x,a)$ qui est nulle lorsque $x$ est la racine carr�e de $a$.

\paragraph{Question 5:} Coder alors la fonction \verb? racine_carree(a)? qui retourne la valeur de la racine de $a$.

Remarque: Il ne faudra pas utiliser les fonctions de la biblioth�que \verb? math ? ni \verb? a**(-1/2.)?. Ces fonctions pourrons �tre utilis�es pour v�rifier votre r�sultat.



\end{document}
