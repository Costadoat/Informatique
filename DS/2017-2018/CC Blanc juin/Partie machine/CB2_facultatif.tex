\documentclass[a4paper,12pt]{article}

\usepackage[a4paper]{geometry}
\geometry{margin={1cm,1.2cm}}
\usepackage[francais]{babel}


\usepackage{multicol}
%\usepackage{nopageno} %pas de numérotation de page
\pagestyle{plain} %numérotation en bas de page, pas d'entête
\usepackage{hyperref}
%\usepackage[latin1]{inputenc}


%%%%%%%%%%%%%%%%%%%%%%%%%%%%%%%%%%%%%%%%%%%%%%%%%%%%%%%%%%%%%%%%%%%%%%%%%%%%%%%%%%%%%

\usepackage[utf8]{inputenc} 
\usepackage{amssymb,amsmath}
\usepackage{stmaryrd}
\usepackage{amsthm}
\usepackage{amscd}
%\usepackage{mathrsfs}
%\usepackage{amsfonts}
%\usepackage[T1]{fontenc}
%\usepackage{theorem}
\usepackage{lscape}
\usepackage{variations}  % pour faire des tableaux de variations
\usepackage{dsfont}
\usepackage{fancyvrb} % pour mettre Verbatim dans une box
\usepackage{moreverb} % pour mettre Verbatim dans une box 
\usepackage{comment} % pour afficher ou non les commentaires, solutions
%\usepackage{slashbox} % pour dans un tabular, couper une case en deux
\usepackage{boxedminipage} % pour cadrer du texte
\usepackage{listings}

% Pour les figures
\usepackage{subfig}
\usepackage{calc} % Pour pouvoir donner des formules dans les d�finitions de longueur
\usepackage{graphicx} % Pour inclure des graphiques 
% Attention : pour inclure des .jpg comme dans l'exemple (ou des .png ou .pdf)
% il faut compiler directement en pdf (commande pdflatex).
% Pour inclure des .eps, il faut compiler avec latex + dvips + ps2pdf.
\usepackage{psfrag}
\usepackage{color}

%%%%%%%%%%%%%%%%%%%%%%%%%%%%%%%%%%%%%%%%%%%%%%%%%%%%%%%%%%%%%%%%%%%%%%%%%%%%%%%%%%%%%

\theoremstyle{definition}
\newtheorem{thm}{Théorème}
%\theorembodyfont{\rmfamily}
\newtheorem*{defn}{Définition}
\newtheorem{exercice}{Exercice}
\newtheorem*{problem}{Problème}
\newtheorem{prop}{Proposition}
\newtheorem{corollaire}{Corollaire}
\newtheorem*{lemme}{Lemme}
\newtheorem*{remark}{Remarque}
\newtheorem*{notation}{Notation}
\newtheorem*{ex}{Exemple}
\newtheorem*{ppe}{Propriété}
\newtheorem*{meth}{Méthode}
\newtheorem*{rappel}{Rappel}
\newtheorem*{voca}{Vocabulaire}
\newtheorem*{solution}{Solution}   

\setlength{\columnseprule}{0.5pt}


%%%%%%%%%%%%%%%%%%%%%%%%%%%%%%%%%%%%%%%%%%%%%%%%%%%%%%%%%%%%%%%%%%%%%%%%%%%%%%%%%%%%%

\newcommand{\bi}{\bigskip}
\newcommand{\dsp}{\displaystyle}
\newcommand{\noi}{\noindent}
\newcommand{\ov}{\overline}
\newcommand{\dsum}{\displaystyle \sum}
\newcommand{\dprod}{\displaystyle \prod}
\newcommand{\dint}{\displaystyle \int}
\newcommand{\dlim}{\displaystyle \lim}

%%%%%%%%%%%%%%%%%%%%%%%%%%%%%%%%%%%%%%%%%%%%%%%%%%%%%%%%%%%%%%%%%%%%%%%%%%%%%%%%%%%%%


%\newcommand{\pgcd}{\mathrm{pgcd}} % pgcd
%\providecommand{\norm}[1]{\lVert#1\rVert} % norme
%\DeclareMathOperator{\Tan}{Tan}  % espace tangent


\newcommand{\N}{\mathbb{N}}
\newcommand{\Z}{\mathbb{Z}}
\newcommand{\Q}{\mathbb{Q}}
\newcommand{\R}{\mathbb{R}}
\newcommand{\C}{\mathbb{C}}
\newcommand{\K}{\mathbb{K}}
\newcommand{\U}{\mathbb{U}}
\newcommand{\Tr}{\text{Tr}\,}
\newcommand{\pg}{\geqslant}
\newcommand{\pp}{\leqslant}
\newcommand{\bul}{\item[$\bullet$]}
\newcommand{\card}{\text{Card}}
\newcommand{\re}{\text{Re}\;}
\newcommand{\im}{\text{Im}\;}
\newcommand{\Ker}{\text{Ker}\;}
\newcommand{\Vect}{\text{Vect}\;}
\newcommand{\rg}{\text{rg}\;}
\newcommand{\TT}{{}^t\!}
\newcommand{\sh}{\text{sh}}
\newcommand{\ch}{\text{ch}}
\newcommand{\Mat}{\text{Mat}}
\usepackage{textcomp}



%%%%%%%%%%%%%%%%%%%%%%%%%%%%%%%%%%%%%%%%%%%%%%%%%%%%%%%%%%%%%%%%%%%%%%%%%%%%%%%%%%%%%%%%%%%%%%%%%%%%%%%%%%%%%%%%%%%%%%%%%%%

\frenchspacing


%%%%%%%%%%%%%%%%%%%%%%%%%%%%%%%%%%%%%%%%%%%%%%%%%%%%%%%%%%%%%%%%%%%%%%%%%%%%%%%%%%%%%%%%%%%%%%%
% Pour une numerotation I, II des sections.
% Pour une numerotation des subsubsections

\setcounter{secnumdepth}{3}
\setcounter{tocdepth}{3}

\renewcommand{\thesection}{\Roman{section})}
\renewcommand{\thesubsection}{\Roman{section}-\arabic{subsection})}
\renewcommand{\thesubsubsection}{\Roman{section}-\arabic{subsection}-\alph{subsubsection}}


%%%%%%%%%%%%%%%%%%%%%%%%%%%%%%%%%%%%%%%%%%%%%%%%%%%%%%%%%%%%
% Pour avoir une enumerate 1) 2)
\frenchbsetup{StandardLists=true}
\usepackage{enumitem}
\setenumerate[1]{label=\arabic*)}



\excludecomment{solution}



\begin{document}
\noindent {\large PTSI\hfill 2017-2018}\\

\begin{center}
{\LARGE  \textbf{Devoir machine}  \bigskip }

\rule{2.39cm}{0.05cm}

\bi 

{\large Informatique}\bigskip

{\large Samedi 29 mai}

\rule{2.39cm}{0.05cm}
\end{center}

\begin{center}
\fbox{ \begin{minipage}{1\textwidth} 
\begin{itemize}
\bul Faire tous les exercices dans un même fichier NomPrenom.py que vous sauvegarderez dans le dossier.
\bul Mettez en commentaire l'exercice que vous traitez.
\bul N'oubliez pas de commentez votre code.
\bul Aucune aide ne sera donnée. %GNA
\end{itemize}
\end{minipage}}
\end{center}


\section*{Exercice en plus}
\noi On cherche à écrire une fonction qui donne la liste de toutes les parties d'un ensemble d'entiers naturels $E$ : $\mathcal{P}(E)$.\\
Pour cela, on associe une liste $L$ composée de $0$ et de $1$ à une partie, de telle sorte qu'on ne prenne que les éléments de $E$ qui ont un indice qui correspond à 1 dans la liste $L$.\\
Par exemple, si $E=[1,7,4,11]$, la partie associée à $L=[0,1,0,1]$ est $[7,11]$.
\begin{enumerate}
\item Ecrire une fonction \verb?partie? qui prend en argument une liste d'entiers naturels $E$ et une liste $L$ de même longueur, composée de 0 et 1 et qui renvoie la partie de $E$ associée.
\item En commentaire, donner l'écriture binaire de 23. 
\item Ecrire une fonction \verb?binaire? qui prend en argument un entier $k$ et qui renvoie son nombre binaire sous forme de liste. Par exemple, \verb?binaire(2)? renvoie $[1,0]$.\\
On n'utilisera pas la fonction \verb?bin?.%GNA ?
\item Ecrire une fonction \verb?P? qui prend comme argument $E$ et renvoie la liste de toutes les parties de $E$.
\end{enumerate}




\end{document}








\section*{Exercice 2}
Soit $n$ un entier naturel non nul et une liste \verb?t? de longueur $n$ dont les termes valent 0 ou 1. Le but de cet exercice est de trouver le nombre maximal de 0 contigus dans \verb?t? (c'est-à-dire figurant dans des cases consécutives). Par exemple, le nombre maximal de zéros contigus dans la liste \verb?t1? suivante vaut 4 :
\[\begin{array}{|c|c|c|c|c|c|c|c|c|c|c|c|c|c|c|c|}
\hline 
\verb?i?  & 0&1&2&3&4&5&6&7&8&9&10&11&12&13&14\\ \hline
 \verb?t1[i]? &0&1&1&1&0&0&0&1&0&1&1&0&0&0&0 \\ \hline
\end{array} \]
\begin{enumerate}
\item Ecrire une fonction \verb?nombreZeros(t,i)?, prenant en paramètres une liste \verb?t?, de longueur $n$ et un indice $i$ compris entre $0$ et $n-1$ et renvoyant :
\[ \left\lbrace \begin{array}{ll}
0 \quad  \text{ si } \verb?t?[i]=1\\
\text{le nombre de zéros consécutifs dans } \verb?t? \text{ à partir de }\verb?t?[i],\quad \text{ si } \verb?t?[i]=0
\end{array}\right.\]
Par exemple, les appels \verb?nombreZeros(t1,4), nombreZeros(t1,1), nombreZeros(t1,8)? renvoient respectivement les valeurs 3,0 et 1.
\item Comment obtenir le nombre maximal de zéros contigus d'une liste \verb?t? connaissant la liste des \verb?nombreZeros(t,i)? pour $0\pp i\pp n-1$ ?\\
En déduire une fonction \verb?nombreZerosMax(t)?, de paramètre \Verb?t?, renvoyant le nombre maximal de 0 contigus d'une liste \verb?t? non vide. On utilisera la fonction \verb?nombreZeros?.
\item Importer ? %GNA
\item Complexité ? %GNA
\item Plus performant ? %GNA
\end{enumerate}
