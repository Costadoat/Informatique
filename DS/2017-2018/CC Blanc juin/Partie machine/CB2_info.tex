\documentclass[a4paper,12pt]{article}

\usepackage[a4paper]{geometry}
\geometry{margin={1cm,1.2cm}}
\usepackage[francais]{babel}


\usepackage{multicol}
%\usepackage{nopageno} %pas de numérotation de page
\pagestyle{plain} %numérotation en bas de page, pas d'entête
\usepackage{hyperref}
%\usepackage[latin1]{inputenc}


%%%%%%%%%%%%%%%%%%%%%%%%%%%%%%%%%%%%%%%%%%%%%%%%%%%%%%%%%%%%%%%%%%%%%%%%%%%%%%%%%%%%%

\usepackage[utf8]{inputenc} 
\usepackage{amssymb,amsmath}
\usepackage{stmaryrd}
\usepackage{amsthm}
\usepackage{amscd}
%\usepackage{mathrsfs}
%\usepackage{amsfonts}
%\usepackage[T1]{fontenc}
%\usepackage{theorem}
\usepackage{lscape}
\usepackage{variations}  % pour faire des tableaux de variations
\usepackage{dsfont}
\usepackage{fancyvrb} % pour mettre Verbatim dans une box
\usepackage{moreverb} % pour mettre Verbatim dans une box 
\usepackage{comment} % pour afficher ou non les commentaires, solutions
%\usepackage{slashbox} % pour dans un tabular, couper une case en deux
\usepackage{boxedminipage} % pour cadrer du texte
\usepackage{listings}

% Pour les figures
\usepackage{subfig}
\usepackage{calc} % Pour pouvoir donner des formules dans les d�finitions de longueur
\usepackage{graphicx} % Pour inclure des graphiques 
% Attention : pour inclure des .jpg comme dans l'exemple (ou des .png ou .pdf)
% il faut compiler directement en pdf (commande pdflatex).
% Pour inclure des .eps, il faut compiler avec latex + dvips + ps2pdf.
\usepackage{psfrag}
\usepackage{color}

%%%%%%%%%%%%%%%%%%%%%%%%%%%%%%%%%%%%%%%%%%%%%%%%%%%%%%%%%%%%%%%%%%%%%%%%%%%%%%%%%%%%%

\theoremstyle{definition}
\newtheorem{thm}{Théorème}
%\theorembodyfont{\rmfamily}
\newtheorem*{defn}{Définition}
\newtheorem{exercice}{Exercice}
\newtheorem*{problem}{Problème}
\newtheorem{prop}{Proposition}
\newtheorem{corollaire}{Corollaire}
\newtheorem*{lemme}{Lemme}
\newtheorem*{remark}{Remarque}
\newtheorem*{notation}{Notation}
\newtheorem*{ex}{Exemple}
\newtheorem*{ppe}{Propriété}
\newtheorem*{meth}{Méthode}
\newtheorem*{rappel}{Rappel}
\newtheorem*{voca}{Vocabulaire}
\newtheorem*{solution}{Solution}   

\setlength{\columnseprule}{0.5pt}


%%%%%%%%%%%%%%%%%%%%%%%%%%%%%%%%%%%%%%%%%%%%%%%%%%%%%%%%%%%%%%%%%%%%%%%%%%%%%%%%%%%%%

\newcommand{\bi}{\bigskip}
\newcommand{\dsp}{\displaystyle}
\newcommand{\noi}{\noindent}
\newcommand{\ov}{\overline}
\newcommand{\dsum}{\displaystyle \sum}
\newcommand{\dprod}{\displaystyle \prod}
\newcommand{\dint}{\displaystyle \int}
\newcommand{\dlim}{\displaystyle \lim}

%%%%%%%%%%%%%%%%%%%%%%%%%%%%%%%%%%%%%%%%%%%%%%%%%%%%%%%%%%%%%%%%%%%%%%%%%%%%%%%%%%%%%


%\newcommand{\pgcd}{\mathrm{pgcd}} % pgcd
%\providecommand{\norm}[1]{\lVert#1\rVert} % norme
%\DeclareMathOperator{\Tan}{Tan}  % espace tangent


\newcommand{\N}{\mathbb{N}}
\newcommand{\Z}{\mathbb{Z}}
\newcommand{\Q}{\mathbb{Q}}
\newcommand{\R}{\mathbb{R}}
\newcommand{\C}{\mathbb{C}}
\newcommand{\K}{\mathbb{K}}
\newcommand{\U}{\mathbb{U}}
\newcommand{\Tr}{\text{Tr}\,}
\newcommand{\pg}{\geqslant}
\newcommand{\pp}{\leqslant}
\newcommand{\bul}{\item[$\bullet$]}
\newcommand{\card}{\text{Card}}
\newcommand{\re}{\text{Re}\;}
\newcommand{\im}{\text{Im}\;}
\newcommand{\Ker}{\text{Ker}\;}
\newcommand{\Vect}{\text{Vect}\;}
\newcommand{\rg}{\text{rg}\;}
\newcommand{\TT}{{}^t\!}
\newcommand{\sh}{\text{sh}}
\newcommand{\ch}{\text{ch}}
\newcommand{\Mat}{\text{Mat}}
\usepackage{textcomp}



%%%%%%%%%%%%%%%%%%%%%%%%%%%%%%%%%%%%%%%%%%%%%%%%%%%%%%%%%%%%%%%%%%%%%%%%%%%%%%%%%%%%%%%%%%%%%%%%%%%%%%%%%%%%%%%%%%%%%%%%%%%

\frenchspacing


%%%%%%%%%%%%%%%%%%%%%%%%%%%%%%%%%%%%%%%%%%%%%%%%%%%%%%%%%%%%%%%%%%%%%%%%%%%%%%%%%%%%%%%%%%%%%%%
% Pour une numerotation I, II des sections.
% Pour une numerotation des subsubsections

\setcounter{secnumdepth}{3}
\setcounter{tocdepth}{3}

\renewcommand{\thesection}{\Roman{section})}
\renewcommand{\thesubsection}{\Roman{section}-\arabic{subsection})}
\renewcommand{\thesubsubsection}{\Roman{section}-\arabic{subsection}-\alph{subsubsection}}


%%%%%%%%%%%%%%%%%%%%%%%%%%%%%%%%%%%%%%%%%%%%%%%%%%%%%%%%%%%%
% Pour avoir une enumerate 1) 2)
\frenchbsetup{StandardLists=true}
\usepackage{enumitem}
\setenumerate[1]{label=\arabic*)}



\excludecomment{solution}



\begin{document}
\noindent {\large PTSI\hfill 2017-2018}\\

\begin{center}
{\LARGE  \textbf{Devoir machine}  \bigskip }

\rule{2.39cm}{0.05cm}

\bi 

{\large Informatique}\bigskip

{\large Samedi 29 mai}

\rule{2.39cm}{0.05cm}
\end{center}

\begin{center}
\fbox{ \begin{minipage}{1\textwidth} 
\begin{itemize}
\bul Faire tous les exercices dans un même fichier NomPrenom.py que vous sauvegarderez dans le dossier.
\bul Mettez en commentaire l'exercice que vous traitez.
\bul N'oubliez pas de commentez votre code.
\bul Aucune aide ne sera donnée. %GNA
\end{itemize}
\end{minipage}}
\end{center}

\section*{Exercice 1}
\noi Afficher votre nom et prénom.



\section*{Exercice 2}
\noi Dans cet exercice, une matrice sera une liste de liste. Par exemple, la matrice $A=\begin{pmatrix}
2&1&3\\0&-1&7
\end{pmatrix}$ sera représentée par : \verb?A=[[2,1,3],[0,-1,7]]?. \\
La taille de $A$ est $[2,3]$, deux lignes, trois colonnes.\\
L'objectif est de définir une fonction qui calcule le produit matriciel entre deux matrices lorsque leur format est compatible.\\
Soit deux matrices $A=(a_{ij})_{\underset{0\pp j\pp p-1}{0\pp i\pp n-1}}$ et $B=(b_{ij})_{\underset{0\pp j\pp l-1}{0\pp i\pp q-1}}$. La matrice $A$ est de taille $[n,p]$ et $B$ de taille $[q,l]$. Elles ont un format compatible si $p=q$. \\
Dans ce cas, on peut définir la matrice produit $C=AB=(c_{ij})_{\underset{0\pp j\pp l-1}{0\pp i\pp n-1}}$ de taille $n\times l$ par :
\[c_{ij}=\dsum_{k=0}^{p-1}a_{ik}b_{kj}\]
\begin{enumerate}
\item Que faut-il mettre à la place des pointillés pour que la ligne suivante renvoie 7 ? : 
\[\verb?A[...][...]?\]
\item Programmer une fonction \verb?taille? qui prend comme argument une matrice $A$ et renvoie sa taille. Sur notre exemple, \verb?taille(A)? renvoie \verb?[2,3]?.
\item Ecrire une fonction \verb?compatible? qui prend comme entrée deux matrices $A$ et $B$ et renvoie \verb?True? si le produit $AB$ est possible, \verb?False? sinon.\\
Par exemple, avec \verb?B=[[2],[0],[1]]?, \verb?produit(A,B)? renvoie \verb?True? et \verb?produit(B,A)? renvoie \verb?False?.
\item Ecrire une fonction \verb?produit? qui prend comme argument deux matrices $A$ et $B$ et qui renvoie \verb?False? si leur format n'est pas compatible et qui renvoie le produit $AB$ sinon.\\
Par exemple, \verb?produit(A,B)? renvoie \verb?[[7],[7]]? et \verb?produit(B,A)? renvoie \verb?False?.
\end{enumerate}

\vfill{\textbf{\Large TOURNEZ LA PAGE}

\newpage



\section*{Exercice 3}
\noi Un nombre de $p$ chiffres est dit narcissique si la somme de ses chiffres à la puissance $p$ lui est égale.
\begin{enumerate}
\item Soit \verb?n=93084?. Vérifier qu'il est un nombre narcissique. 
\end{enumerate}
L'objectif est de coder une fonction qui déterminera si un entier est narcissique ou pas. Pour ça, on aura besoin d'avoir accès aux chiffres de l'entier. Une méthode est de transformer l'entier en chaine de caractères à l'aide de la fonction \verb?str?.
\begin{enumerate}
\setcounter{enumi}{1}
\item Quel est le type de \verb?str(n)? ? Que renvoie \verb?len(str(n))? ? Que renvoie \verb?str(n)[3]? ? 
\item Ecrire une fonction \verb?narcisse? d'argument $n$ renvoyant un booléen indiquant si $n$ est narcissique ou non.
\item Afficher les nombres narcissiques compris entre 1 et 10 000.
%\item Déterminer tous les entiers compris entre 1 et 10 000 à la fois narcissique et premiers.
%\item Ecrire une fonction \verb?narcis_suivant(n,N)?, d'argument $n$ et $N$ qui renvoie le premier entier narcissique supérieur à $n$ mais inférieur à $N$ et qui renvoie \verb?False? sinon.
\end{enumerate}



\section*{Exercice 4}
\begin{enumerate}
\item \label{Q1} A un nombre $c$ quelconque, on associe la suite $(u_n)_{n\in\N}$ définie par :
\[u_0=0\quad \text{ et }\quad u_{n+1}=u_n^2+c \quad \text{pour }n\pg 0\]
Pour $c=0.5$, afficher les valeurs de $u_0,u_1,\cdots$ jusqu'à $u_{10}$.
\item On pose : $m=10$ et $M=20$.\\
S'il existe, on note $k$ le plus petit entier tel que l'on ait : $0\pp k\pp m$ et $|u_k|>M$.\\
En vous aidant des valeurs affichées à la question \ref{Q1}, pour $c=0.5$, est-ce que $k$ existe ? Si oui, quelle est sa valeur ? \textit{(réponse en commentaire).}
\item On définit alors la fonction $f$ par :
\[f\colon c\mapsto \left\lbrace\begin{array}{ll}
k & \text{s'il existe}\\
m+1 & \text{sinon}
\end{array} \right.\]
\begin{enumerate}
\item Donner le code définissant la fonction $f$.
\item Afficher \verb?f(0.5)?
\item Tracer l'allure de la courbe représentative de $f$ sur $[-1,2]$. Pour cela, créer une liste \verb?LX? de 400 valeurs équiréparties entre -1 et 2 inclus.\\
%\textit{On pourra utiliser les fonctions \verb?plot? et \verb?show? de la sous-bibliothèque \verb?matplotlib.pyplot.?}
%\item Construire le tableau des valeurs $f(x+iy)$ où $x$ prend 101 valeurs comprises entre -2 et 0.5 et $y$ prend 101 valeurs entre -1.1 et 1.1.\\
%On rappelle que $1+2i$ s'écrit \verb?1+2j?
%\item Tracer l'image que code ce tableau. \textit{On pourra utiliser les fonctions \verb?imshow? et \verb?show? de la sous bibliothèque \verb?matplotlib.pyplot.?}
\end{enumerate}
\end{enumerate}

\end{document}









\section*{Exercice 2}
Soit $n$ un entier naturel non nul et une liste \verb?t? de longueur $n$ dont les termes valent 0 ou 1. Le but de cet exercice est de trouver le nombre maximal de 0 contigus dans \verb?t? (c'est-à-dire figurant dans des cases consécutives). Par exemple, le nombre maximal de zéros contigus dans la liste \verb?t1? suivante vaut 4 :
\[\begin{array}{|c|c|c|c|c|c|c|c|c|c|c|c|c|c|c|c|}
\hline 
\verb?i?  & 0&1&2&3&4&5&6&7&8&9&10&11&12&13&14\\ \hline
 \verb?t1[i]? &0&1&1&1&0&0&0&1&0&1&1&0&0&0&0 \\ \hline
\end{array} \]
\begin{enumerate}
\item Ecrire une fonction \verb?nombreZeros(t,i)?, prenant en paramètres une liste \verb?t?, de longueur $n$ et un indice $i$ compris entre $0$ et $n-1$ et renvoyant :
\[ \left\lbrace \begin{array}{ll}
0 \quad  \text{ si } \verb?t?[i]=1\\
\text{le nombre de zéros consécutifs dans } \verb?t? \text{ à partir de }\verb?t?[i],\quad \text{ si } \verb?t?[i]=0
\end{array}\right.\]
Par exemple, les appels \verb?nombreZeros(t1,4), nombreZeros(t1,1), nombreZeros(t1,8)? renvoient respectivement les valeurs 3,0 et 1.
\item Comment obtenir le nombre maximal de zéros contigus d'une liste \verb?t? connaissant la liste des \verb?nombreZeros(t,i)? pour $0\pp i\pp n-1$ ?\\
En déduire une fonction \verb?nombreZerosMax(t)?, de paramètre \Verb?t?, renvoyant le nombre maximal de 0 contigus d'une liste \verb?t? non vide. On utilisera la fonction \verb?nombreZeros?.
\item Importer ? %GNA
\item Complexité ? %GNA
\item Plus performant ? %GNA
\end{enumerate}
