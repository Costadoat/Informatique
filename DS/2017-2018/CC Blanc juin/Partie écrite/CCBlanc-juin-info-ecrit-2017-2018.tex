\documentclass[10pt,a4paper]{article}
  \usepackage[french]{babel}
  \usepackage[utf8]{inputenc}
  \usepackage[T1]{fontenc}
  \usepackage{xcolor}
  \usepackage[]{graphicx}
  \usepackage{makeidx}
  \usepackage{textcomp}
  \usepackage{amsmath}
  \usepackage{amssymb}
  \usepackage{stmaryrd}
  \usepackage{fancyhdr}
  \usepackage{lettrine}
  \usepackage{calc}
  \usepackage{boxedminipage}
  \usepackage[french,onelanguage, boxruled,linesnumbered]{algorithm2e}
  \usepackage[colorlinks=false,pdftex]{hyperref}
  \usepackage{minted}
  \usepackage{url}
  %\usepackage[locale=FR]{siunitx}
  \usepackage{multicol}
  \makeindex

  %\graphicspath{{../Images/}}

%  \renewcommand\listingscaption{Programme}

  %\renewcommand{\thechapter}{\Alph{chapter}}
  \renewcommand{\thesection}{\Roman{section}}
  %\newcommand{\inter}{\vspace{0.5cm}%
  %\noindent }
  %\newcommand{\unite}{\ \textrm}
  \newcommand{\ud}{\mathrm{d}}
  \newcommand{\vect}{\overrightarrow}
  %\newcommand{\ch}{\mathrm{ch}} % cosinus hyperbolique
  %\newcommand{\sh}{\mathrm{sh}} % sinus hyperbolique

  \textwidth 160mm
  \textheight 250mm
  \hoffset=-1.70cm
  \voffset=-1.5cm
  \parindent=0cm

  \pagestyle{fancy}
  \fancyhead[L]{\bfseries {\large PTSI -- Dorian}}
  \fancyhead[C]{\bfseries{{\type} {\num}}}
  \fancyhead[R]{\bfseries{\large Informatique}}
  \fancyfoot[C]{\thepage}
  \fancyfoot[L]{\footnotesize R. Costadoat, J. Genzmer, W. Robert}
  \fancyfoot[R]{\small \today}
  
  \definecolor{bg}{rgb}{0.9,0.9,0.9}
  
  
  % macro Juliette
  
\usepackage{comment}   
\usepackage{amsthm}  
\theoremstyle{definition}
\newtheorem{exercice}{Exercice}
\newtheorem*{rappel}{Rappel}
\newtheorem*{remark}{Remarque}
\newtheorem*{defn}{Définition}
\newtheorem*{ppe}{Propriété}
\newtheorem{solution}{Solution}

\parindent= 0pt
\textheight 250mm

  \pagestyle{fancy}
  \fancyfoot[C]{\thepage}
  \fancyhead[LO,LE]{\bfseries {\large PTSI -- Dorian}}
  \fancyhead[RO,RE]{\bfseries{\large Informatique}}
  \fancyhead[CO,CE]{Concours blanc juin 2018}
  \fancyfoot[L]{Samedi 29 mai 2018.}
  
\begin{document}

 \begin{center}
  \begin{large}
  \fbox{Concours blanc -- Partie écrite. Durée : 1 heure}
  \end{large}
 \end{center}

\begin{boxedminipage}{\textwidth} 
Lorsqu'on écrit un code Python : faire attention à ce que les indentations soient visibles sur la copie ; commenter le code de façon à expliquer les grandes étapes de l'algorithme en ajoutant un commentaire en fin de ligne de code après le symbole $\sharp$.
\end{boxedminipage}

\section{Période d'un pendule simple -- Exercice}
\textit{Les quatre questions peuvent être traitées indépendamment l'une de l'autre en supposant .}
Le système étudié consiste en un point matériel $M$ de masse $m$ relié par une barre rigide de masse nulle et de longueur $\ell$ à un point $O$ dans le référentiel terrestre supposé galiléen. Le système est soumis à la pesanteur et la position du point matériel $M$ est repérée par l'angle $\theta$ entre la verticale et la droite $(OM)$. Le vecteur accélération de la pesanteur est noté $\vec g$.

Le pendule est lâché sans vitesse initiale et avec un angle initial $0< \theta_0 < \pi$. On montre qu'en l'absence de frottements la période $T$ du mouvement $M$ est donnée par la relation : $$T = 4\sqrt{\frac{\ell}{2g}}\int\limits_0^{\theta_0} \frac{1}{\sqrt{\cos(\theta)-\cos(\theta_0)}}\ud\theta$$

\begin{enumerate}
 \item Écrire le code python d'une fonction \texttt{f(theta,theta0,g,l)} qui prend comme arguments l'angle $\theta$, la condition initiale $\theta_0$, l'accélération de la pesanteur $g$, la longueur du fil $\ell$ et qui renvoie la valeur de $4\sqrt{\tfrac{\ell}{2g}}\tfrac{1}{\sqrt{\cos(\theta)-\cos(\theta_0)}}$.
\end{enumerate}

Dans toute la suite, vous pouvez appeler dans vos codes et programmes la fonction \texttt{f} même si vous n'avez pas répondu à la première question.

\begin{enumerate}
\setcounter{enumi}{1}
 \item Écrire le code python d'une fonction \texttt{periodRect(f,theta0,g,l,n)} qui prend comme arguments la fonction $f$, la condition initiale $\theta_0$, l'accélération de la pesanteur $g$, la longueur du fil $\ell$ et un entier $n$ et qui calcule la valeur approchée de la période du mouvement en utilisant la méthode d'intégration des rectangles grâce à $n$ rectangles. 
\end{enumerate} 

Dans toute la suite, vous pouvez appeler dans vos codes et programmes la fonction \texttt{periodRect} même si vous n'avez pas répondu à la question précédente.

On sait que la période du mouvement est quasiment indépendante de l'amplitude pour les « petites amplitudes » (on parle d'isochronisme) et elle est alors approximativement égale $2\pi\sqrt{\tfrac{\ell}{g}}$ (période propre). Mais quand l'amplitude devient « grande », la période en dépend (perte d'isochronisme). Dans les conditions initiales précédentes, l'amplitude est égale $\theta_0$.

\begin{enumerate}
\setcounter{enumi}{2}
 \item Proposer un programme écrit en python permettant de tracer la courbe des valeurs de périodes calculées pour 100 valeurs d'angle initial régulièrement espacées dans l'intervalle $]0\,;\,\pi[$. La période $T$ sera évaluée grâce à la fonction \texttt{periodRect} et un nombre de rectangles $n=10000$. On prend : $g=\SI{9,81}{\meter\per\second\squared}$ et $\ell = \SI{1,00}{\meter}$.
 
 \item En utilisant une méthode dichotomique, proposer un programme  écrit en python permettant de trouver à un dix-millième près la première valeur de $\theta_0$ pour laquelle la période est plus grande de plus de \SI{1}{\percent} de la valeur $2\pi\sqrt{\tfrac{\ell}{g}}$. L'exécution de ce programme devrait afficher à l'écran l'angle cherché.
\end{enumerate}

\newpage

\section{Étude d'un algorithme -- Questions de cours}

On considère la fonction $f$ définie pour $a \in \mathbb{N}$ et traduite par le programme écrit en python suivant :

\begin{listing}
\begin{minted}[linenos,frame=lines]{python}
def f(a):
    i=1
    n=0
    while(i<=a):
	if i%2 ==0:
	    n=n+i
	i=i+1
	return n
\end{minted}
\caption{Fonction à étudier.}
\label{prog:fonctionf}
\end{listing}

\begin{enumerate}
 \item Que réalise l'opérateur \texttt{\%} ? 
 \item À votre avis quel est le résultat retourné attendu par le concepteur de la fonction \texttt{f} ?
 \item Expliquer pourquoi le programme proposé ne retourne pas le résultat attendu.
 \item Proposer une modification du code afin que ce soit le cas.
 \item Une fois le code corrigé, détailler ligne à ligne l'évolution des valeurs des variables \texttt{i} et \texttt{n} pour $a=5$.
 \item Déterminer la complexité du programme pour $a$ quelconque.
 \item Proposer une réécriture du programme utilisant une boucle \texttt{for} à la place de la boucle \texttt{while}.
\end{enumerate}


\end{document}


