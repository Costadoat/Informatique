\input{../../../headers/dsinfoheaders}

\begin{center}
{\Large\bf {\type} \no {\numero} -- \descrip}
\end{center}

\SetKw{KwFrom}{de} 

\begin{boxedminipage}{.9\textwidth} 
\begin{itemize}
 \item Faire tous les exercices dans un fichier {NomPrenom.py} à sauvegarder,
 \item mettre en commentaire l'exercice et la question traités (ex: \# Exercice 1),
 \item ne pas oublier pas de commenter ce qui est fait dans votre code (ex: \# Je crée une fonction pour calculer la racine d'un nombre),
 \item il est possible de demander un déblocage pour une question, mais celle-ci sera notée 0,
 \item il faut vérifier avant de partir que le code peut s'exécuter et qu'il affiche les résultats que vous attendez. Les lignes de code qui doivent s'exécuter sont décommentées.
\end{itemize}
\end{boxedminipage}

%\title{Devoir d'informatique en Python \\ \large Simulation d’un distributeur automatique}

\section*{Contexte}
On souhaite programmer le fonctionnement simplifié d’un distributeur automatique de boissons.  
Le distributeur propose au départ~:
\begin{itemize}
    \item Soda : 2 € (stock initial : 5),
    \item Café : 1 € (stock initial : 5),
    \item Jus : 3 € (stock initial : 5).
\end{itemize}

\bigskip
L’objectif est de réaliser un programme en Python qui simule les achats d’un utilisateur.  

\section{Choix de la boisson}

\begin{minted}[linenos]{python}
import random

# Stocks de départ
soda = 5
cafe = 5
jus = 5

0 = achats

print("Bienvenue dans le distributeur automatique !)

# Boucle tant qu’il reste des boissons
while soda > 0 or cafe > 0 or jus > 0
    print("\n=== Menu ===")
    print("1 - Soda (2 euros) [stock :", soda, "]")
    print("2 - Café (1 euros) [stock :", cafe, "]")
    print("3 - Jus  (3 euros) [stock :", jus, "]")
    print("0 - Quitter")

    choix = int(input("Sélection ? "))

    if choix == 0:
        print("Merci, à bientôt !")
        break
    else:
        # Cas Soda
        if choix == 1:
            print("Soda demandé")

print("\n=== Fin du programme ===")
print("Nombre total d’achats :", achats)
\end{minted}

Ce programme permet de choisir la boisson en entrant un entier au clavier. La suite du programme n'étant pas codée, il faut entrer 0 pour quitter et retrouver la console.

\question{\textbf{Recopier} le code, \textbf{corriger} les 3 erreurs qui se sont glissées à l'intérieur (penser à lire les erreurs sur la console) et l'\textbf{exécuter} afin de vérifier que la saisie du choix est possible.}

\section{Paiement et service}

Nous nous intéressons maintenant au cas de figure où l'utilisateur a entré 1 avec son clavier afin de sélectionner un soda.

Le programme doit maintenant:
\begin{enumerate}
    \item Vérifier que le stock est suffisant. Si le produit est épuisé, afficher un message d’erreur.
    \item Demander combien d’argent l’utilisateur insère.
    \item Vérifier que la somme est suffisante :
    \begin{itemize}
        \item si oui, calculer la monnaie à rendre,
        \item si non, refuser la vente et rendre l’argent.
    \end{itemize}
    \item Modifier la quantité de soda en stock,
    \item Modifier le nombre total d'achat de la personne.
\end{enumerate}

\begin{minted}[linenos]{python}
if choix == 1:
    print("Soda demandé")
    if .........:
        print("Plus de soda en stock.")
    else:
        # Argent inséré
        argent = int(input("Insérez votre argent : "))
        if argent ......:
            print("Pas assez d’argent.")
        else:
            print("Vous avez acheté un soda.")
            monnaie = ..........
            if monnaie > 0:
                print("Monnaie rendue :", monnaie, "euros")
            soda .....
            achats .....
\end{minted}

\question{\textbf{Recopier} le code à sa bonne place dans le script (attention à l'indentation), le \textbf{compléter} et l'\textbf{exécuter} afin de vérifier que le fonctionnement lors du choix de boisson.}

\question{A partir de cet extrait, \textbf{coder} la suite du script permettant de la gestion l'achat d'un café ou d'un jus. Il faudra aussi prévoir un message d'erreur si l'utilisateur effectue une sélection erronée.}


\section{Organisation en fonctions}

Afin d'améliorer la lisibilité du code, il est préférable d'utiliser les deux fonctions :
\begin{itemize}
    \item \texttt{affichage\_menu(soda, cafe, jus)},
    \item \texttt{servir\_boisson(nom, stock, prix, achats)}.
\end{itemize}

La suite du code pourrait alors s'écrire comme suit.
\begin{minted}[linenos]{python}
    if choix == 0:
        print("Merci, à bientôt !")
        break
    else:
        # Cas Soda
        if choix == 1:
            soda,achats=servir_boisson('soda',soda,2,achats)
        # Cas Café
        elif choix == 2:
            cafe,achats=servir_boisson('cafe',cafe,1,achats)
        # Cas Jus
        elif choix == 3:
            jus,achats=servir_boisson('jus',jus,3,achats)
        else:
            print("Choix invalide.")
\end{minted}

\question{Écrire les deux fonctions \texttt{affichage\_menu} et \texttt{servir\_boisson} et les tester à l'aide du code fourni.}

\section{Bonus}

\question{\textbf{Ajouter} une option \og produit gratuit\fg, à la fonction \texttt{servir\_boisson}, qui une fois sur 10 offre une boisson au client (utiliser l'opération \texttt{random.randrange(10) == 0}).}

\begin{center}
\Large{FIN}
\end{center}

\cleardoublepage

\ifdef{\public}{\end{document}}{\pagestyle{correction}}

\begin{center}
\Large{Correction}
\end{center}

\reponse{}

\begin{minted}{python}
import random

# Stocks de départ
soda = 5
cafe = 5
jus = 5

achats = 0

print("Bienvenue dans le distributeur automatique !")

# Boucle tant qu’il reste des boissons
while soda > 0 or cafe > 0 or jus > 0:
    print("\n=== Menu ===")
    print("1 - Soda (2 euros) [stock :", soda, "]")
    print("2 - Café (1 euros) [stock :", cafe, "]")
    print("3 - Jus  (3 euros) [stock :", jus, "]")
    print("0 - Quitter")

    choix = int(input("Sélection ? "))

    if choix == 0:
        print("Merci, à bientôt !")
        break
    else:
        # Cas Soda
        if choix == 1:
            print("Soda demandé")

print("\n=== Fin du programme ===")
print("Nombre total d’achats :", achats)
\end{minted}

\reponse{}

\begin{minted}{python}
import random

# Stocks de départ
soda = 5
cafe = 5
jus = 5

achats = 0

print("Bienvenue dans le distributeur automatique !")

# Boucle tant qu’il reste des boissons
while soda > 0 or cafe > 0 or jus > 0:
    print("\n=== Menu ===")
    print("1 - Soda (2 euros) [stock :", soda, "]")
    print("2 - Café (1 euros) [stock :", cafe, "]")
    print("3 - Jus  (3 euros) [stock :", jus, "]")
    print("0 - Quitter")

    choix = int(input("Sélection ? "))

    if choix == 0:
        print("Merci, à bientôt !")
        break
    else:
        # Cas Soda
        if choix == 1:
            if soda == 0:
                print("Plus de soda en stock.")
            else:
                argent = int(input("Insérez votre argent : "))
                if argent < 2:
                    print("Pas assez d’argent.")
                else:
                    print("Vous avez acheté un soda.")
                    monnaie = argent - 2
                    if monnaie > 0:
                        print("Monnaie rendue :", monnaie, "euros")
                    soda -= 1
                    achats += 1

print("\n=== Fin du programme ===")
print("Nombre total d’achats :", achats)
\end{minted}

\reponse{}

\begin{minted}{python}
import random

# Stocks de départ
soda = 5
cafe = 5
jus = 5

achats = 0

print("Bienvenue dans le distributeur automatique !")

# Boucle tant qu’il reste des boissons
while soda > 0 or cafe > 0 or jus > 0:
    print("\n=== Menu ===")
    print("1 - Soda (2 euros) [stock :", soda, "]")
    print("2 - Café (1 euros) [stock :", cafe, "]")
    print("3 - Jus  (3 euros) [stock :", jus, "]")
    print("0 - Quitter")

    choix = int(input("Sélection ? "))

    if choix == 0:
        print("Merci, à bientôt !")
        break
    else:
        # Cas Soda
        if choix == 1:
            if soda == 0:
                print("Plus de soda en stock.")
            else:
                argent = int(input("Insérez votre argent : "))
                if argent < 2:
                    print("Pas assez d’argent.")
                else:
                    print("Vous avez acheté un soda.")
                    monnaie = argent - 2
                    if monnaie > 0:
                        print("Monnaie rendue :", monnaie, "euros")
                    soda -= 1
                    achats += 1

        # Cas Café
        elif choix == 2:
            if cafe == 0:
                print("Plus de café en stock.")
            else:
                argent = int(input("Insérez votre argent : "))
                if argent < 1:
                    print("Pas assez d’argent.")
                else:
                    print("Vous avez acheté un café.")
                    monnaie = argent - 1
                    if monnaie > 0:
                        print("Monnaie rendue :", monnaie, "euros")
                    cafe -= 1
                    achats += 1

        # Cas Jus
        elif choix == 3:
            if jus == 0:
                print("Plus de jus en stock.")
            else:
                argent = int(input("Insérez votre argent : "))
                if argent < 3:
                    print("Pas assez d’argent.")
                else:
                    print("Vous avez acheté un jus.")
                    monnaie = argent - 3
                    if monnaie > 0:
                        print("Monnaie rendue :", monnaie, "euros")
                    jus -= 1
                    achats += 1
        else:
            print("Choix invalide.")

print("\n=== Fin du programme ===")
print("Nombre total d’achats :", achats)
\end{minted}

\reponse{}

\begin{minted}{python}
import random

def affichage_menu(soda, cafe, jus):
    print("\n=== Menu ===")
    print("1 - Soda (2 euros) [stock :", soda, "]")
    print("2 - Café (1 euros) [stock :", cafe, "]")
    print("3 - Jus  (3 euros) [stock :", jus, "]")
    print("0 - Quitter")

def servir_boisson(nom,stock,prix,achats):

    if stock == 0:
        print("Plus de "+nom+" en stock.")
    else:
        argent = int(input("Insérez votre argent : "))
        if argent < prix:
            print("Pas assez d’argent.")
        else:
            print("Vous avez acheté un "+nom+".")
            monnaie = argent - prix
            if monnaie > 0:
                print("Monnaie rendue :", monnaie, "euros")
            stock -= 1
            achats += 1
    return stock,achats

# Stocks de départ
soda = 5
cafe = 5
jus = 5

achats = 0

print("Bienvenue dans le distributeur automatique !")

# Boucle tant qu’il reste des boissons
while soda > 0 or cafe > 0 or jus > 0:
    affichage_menu(soda, cafe, jus)
    
    choix = int(input("Sélection ? "))

    if choix == 0:
        print("Merci, à bientôt !")
        break
    else:
        # Cas Soda
        if choix == 1:
            soda,achats=servir_boisson('soda',soda,2,achats)
        # Cas Café
        elif choix == 2:
            cafe,achats=servir_boisson('cafe',cafe,1,achats)
        # Cas Jus
        elif choix == 3:
            jus,achats=servir_boisson('jus',jus,3,achats)
        else:
            print("Choix invalide.")

print("\n=== Fin du programme ===")
print("Nombre total d’achats :", achats)
\end{minted}

\reponse{}

\begin{minted}{python}
import random

def servir_boisson(nom,stock,prix,achats):

    if stock == 0:
        print("Plus de "+nom+" en stock.")
    elif random.randrange(10) == 0:
        print("On vous offre un "+nom+" !")
        stock -= 1
        achats += 1
    else:
        argent = int(input("Insérez votre argent : "))
        if argent < prix:
            print("Pas assez d’argent.")
        else:
            print("Vous avez acheté un "+nom+".")
            monnaie = argent - prix
            if monnaie > 0:
                print("Monnaie rendue :", monnaie, "euros")
            stock -= 1
            achats += 1
    return stock,achats

# Stocks de départ
soda = 5
cafe = 5
jus = 5

achats = 0

print("Bienvenue dans le distributeur automatique !")

# Boucle tant qu’il reste des boissons
while soda > 0 or cafe > 0 or jus > 0:
    affichage_menu(soda, cafe, jus)
    
    choix = int(input("Sélection ? "))

    if choix == 0:
        print("Merci, à bientôt !")
        break
    else:
        # Cas Soda
        if choix == 1:
            soda,achats=servir_boisson('soda',soda,2,achats)
        # Cas Café
        elif choix == 2:
            cafe,achats=servir_boisson('cafe',cafe,1,achats)
        # Cas Jus
        elif choix == 3:
            jus,achats=servir_boisson('jus',jus,3,achats)
        else:
            print("Choix invalide.")

print("\n=== Fin du programme ===")
print("Nombre total d’achats :", achats)
\end{minted}

\end{document}
