\newcommand{\nom}{Porte conteneur}
\newcommand{\sequence}{03}
\newcommand{\num}{04}
\newcommand{\type}{TD}
\newcommand{\descrip}{Résolution d'un problème en utilisant des méthodes algorithmiques}
\newcommand{\competences}{Alt-C3: Concevoir un algorithme répondant à un problème précisément posé}
\documentclass[10pt,a4paper]{article}
  \usepackage[french]{babel}
  \usepackage[utf8]{inputenc}
  \usepackage[T1]{fontenc}
  \usepackage{xcolor}
  \usepackage[]{graphicx}
  \usepackage{makeidx}
  \usepackage{textcomp}
  \usepackage{amsmath}
  \usepackage{amssymb}
  \usepackage{stmaryrd}
  \usepackage{fancyhdr}
  \usepackage{lettrine}
  \usepackage{calc}
  \usepackage{boxedminipage}
  \usepackage[french,onelanguage, boxruled,linesnumbered]{algorithm2e}
  \usepackage[colorlinks=false,pdftex]{hyperref}
  \usepackage{minted}
  \usepackage{url}
  \usepackage[locale=FR]{siunitx}
  \usepackage{multicol}
  \usepackage{tikz}
  \makeindex

  %\graphicspath{{../Images/}}

%  \renewcommand\listingscaption{Programme}

  %\renewcommand{\thechapter}{\Alph{chapter}}
  \renewcommand{\thesection}{\Roman{section}}
  %\newcommand{\inter}{\vspace{0.5cm}%
  %\noindent }
  %\newcommand{\unite}{\ \textrm}
  \newcommand{\ud}{\mathrm{d}}
  \newcommand{\vect}{\overrightarrow}
  %\newcommand{\ch}{\mathrm{ch}} % cosinus hyperbolique
  %\newcommand{\sh}{\mathrm{sh}} % sinus hyperbolique

  \textwidth 160mm
  \textheight 250mm
  \hoffset=-1.70cm
  \voffset=-1.5cm
  \parindent=0cm

  \pagestyle{fancy}
  \fancyhead[L]{\bfseries {\large PTSI -- Dorian}}
  \fancyhead[C]{\bfseries{{\type} \no \numero}}
  \fancyhead[R]{\bfseries{\large Informatique}}
  \fancyfoot[C]{\thepage}
  \fancyfoot[L]{\footnotesize R. Costadoat, C. Darreye}
  \fancyfoot[R]{\small \today}
  
  \definecolor{bg}{rgb}{0.9,0.9,0.9}
  
  
  % macro Juliette
  
\usepackage{comment}   
\usepackage{amsthm}  
\theoremstyle{definition}
\newtheorem{exercice}{Exercice}
\newtheorem*{rappel}{Rappel}
\newtheorem*{remark}{Remarque}
\newtheorem*{defn}{Définition}
\newtheorem*{ppe}{Propriété}
\newtheorem{solution}{Solution}

\newcounter{num_quest} \setcounter{num_quest}{0}
\newcounter{num_rep} \setcounter{num_rep}{0}
\newcounter{num_cor} \setcounter{num_cor}{0}

\newcommand{\question}[1]{\refstepcounter{num_quest}\par
~\ \\ \parbox[t][][t]{0.15\linewidth}{\textbf{Question \arabic{num_quest}}}\parbox[t][][t]{0.85\linewidth}{#1\label{q\the\value{num_quest}}}\par
~\ \\}

\newcommand{\reponse}[4][1]
{\noindent
\rule{\linewidth}{.5pt}\\
\textbf{Question\ifthenelse{#1>1}{s}{} \multido{}{#1}{%
\refstepcounter{num_rep}\ref{q\the\value{num_rep}} }:} ~\ \\
\ifdef{\public}{#3 ~\ \\ \feuilleDR{#2}}{#4}
}

\newcommand{\cor}
{\refstepcounter{num_cor}
\noindent
\rule{\linewidth}{.5pt}
\textbf{Question \arabic{num_cor}:} \\
}


%\excludecomment{solution}
%\excludecomment{exercice}

\begin{document}

\begin{center}
{\Large\bf {\type} \no {\num}}
\end{center}

\SetKw{KwFrom}{de} 

Faites les deux exercices sur deux feuilles séparées.

\section*{Exercice 1 --  Résolution d'équations}
Soit $f$ une fonction donnée. On cherche à calculer une approximation d'une solution de l'équation : $f(x)=0$.
\begin{enumerate}
\item Avec la méthode de la dichotomie : \\
Écrire une fonction \verb?dicho? qui prend comme entr\' ee la fonction $f$ \` a \' etudier, les bornes initiales $a$ et $b$, la pr\' ecision $\epsilon$ et qui renvoie $\dfrac{a_n+b_n}{2}$, approximation d’une solution \` a $\epsilon$ pr\` es.
\begin{solution}~\\
\vspace{-1cm}
\begin{minted}[linenos,frame=lines]{python}
def dicho(f,a,b,eps):               
    while (b-a)>2*eps:
        c=(float(a)+b)/2
        if f(a)*f(c)<0:
            b=c
        else:
            a=c    
    return((a+b)/2.)
\end{minted}    
\end{solution}
\item Avec la méthode de Newton :\\
\' Ecrire une fonction \verb?Newton? qui prend comme entr\' ee la fonction \` a \' etudier, sa d\' eriv\' ee, une valeur initiale de la suite $x_0$, la pr\' ecision $\epsilon$ et qui renvoie $x_n$ approximation de la solution.
\begin{solution}~\\
\vspace{-1cm}
\begin{minted}[frame=lines]{python}
def Newton(f,fprime,x0,epsilon):
    x=float(x0)
    compteur=0
    # l ecart entre xn et x(n+1) est f(xn)/fprime(xn)
    while abs(f(x)/fprime(x))>epsilon and compteur<1000 : 
        x=x-f(x)/fprime(x)
        compteur=compteur+1
    return(u)        
\end{minted}    
\end{solution}
\item Donner un avantage et un inconvénient de la méthode de Newton par rapport à la méthode de la dichotomie.
\begin{solution}
La méthode de Newton converge plus rapidement que la dichotomie. Mais elle ne converge pas forcément. Alors que si $f(a)$ et $f(b)$ sont de signes opposés, la dichotomie converge nécessairement vers l'une des solutions. 
\end{solution}
\end{enumerate}





\section*{Exercice 2 -- Approximation d'une intégrale}
Soit $f$ une fonction donnée. On souhaite approximer la valeur de l'intégrale $\displaystyle \int_a^b f$ avec différentes méthodes.
\subsection{Cas où f est connue}
Dans cette partie, on travaille avec une fonction $f$ connue sur tout l'intervalle $[a,b]$.
On subdivise l'intervalle $[a,b]$ en $n$ intervalles du type $[x_i,x_{i+1}]$, de même longueur avec : \\ $a=x_0<x_1<\cdots <x_n=b$.
\begin{enumerate}
\item Avec la méthode des rectangles : \\
sur chaque intervalle $[x_i,x_{i+1}]$, on approxime $f$ par une constante. \\
Coder un algorithme qui calcule une approximation de $\displaystyle\int_0^{\frac{3\pi}{2}}\cos(x)dx$ avec 2019 intervalles.
\begin{solution}~\\
\vspace{-1cm}
\begin{minted}[frame=lines]{python}
x=linspace(0,3*pi/2,2020)
integrale=0
for i in range(2019):
   integrale=integrale+cos(x[i])*(x[i+1]-x[i])	
print integrale	
\end{minted}
\end{solution}
\item Avec la méthode de Simpson : \\
sur chaque intervalle $[x_i,x_{i+1}]$, on approxime $f$ par un polynôme de degré deux. Dans ce cas, l'intégrale est approximée par :
\[    {\displaystyle   \int_a^b f(x)dx\approx\frac {b-a}{6n}\left(\sum_{i=0}^{n-1} f(x_i)+4f\left( \frac{x_i+x_{i+1}}{2}\right)+f(x_{i+1}) \right)}\]
Ecrire un programme qui calcule une approximation de $\displaystyle   \int_a^b f$ par la méthode de Simpson.\\ 
(On supposera que \verb?f?, \verb?n?, \verb?a? et \verb?b? sont déjà définis dans le programme.)
\begin{solution}~\\
\vspace{-1cm}
\begin{minted}[frame=lines]{python}
x=linspace(a,b,n+1)
integrale=0
for i in range(n):
   integrale=integrale+(b-a)*1./(6*n)*(f(x[i])+4*f((x[i+1]+x[i])/2.)+f(x[i+1]))
print integrale	
\end{minted}
\end{solution}
\end{enumerate}
\subsection{Avec lecture de fichiers}
Dans cette partie, la fonction $f$ n'est connue qu'en certains points, dont les coordonn\' ees sont situ\' ees dans un fichier. Ce fichier \texttt{coordonnee.csv}, situé dans le répertoire de travail, contient une quinzaine de lignes selon le modèle suivant :
\[0.0;1.00988282142\]
\[0.1;1.0722126449\]
Chaque ligne contient deux valeurs flottantes séparées par un point-virgule, représentant respectivement l'abscisse et l'ordonnée d'un point. Les points sont ordonnés par abscisses croissantes.
\begin{enumerate}
\item Rappeler quels sont les différents modes d'ouverture d'un fichier texte à l'aide de Python, et les différences entre ces modes.
\begin{solution}
Il est possible d'ouvrir un fichier dans plusieurs \textit{modes} : lire (\verb|'r'|), ajouter à la fin (\verb|'a'|), écrire (\verb|'w'|). 
\end{solution}
\item \'Ecrire une séquence d'instructions en Python permettant d'effectuer les opérations suivantes sur le fichier \texttt{coordonnee.csv} :\\
Ouvrir le fichier en lecture, le lire et construire deux listes de flottants : la liste \verb?LX? des abscisses et la liste \verb?LY? des ordonnées contenues dans ce fichier.
\begin{solution}~\\
\vspace{-1cm}
\begin{minted}[frame=lines]{python}
fichier=open('coordonnee.csv','r')

LX=[]
LY=[]

for ligne in fichier.readlines():
    LX.append(float(ligne.strip().split(";")[0]))
    LY.append(float(ligne.strip().split(";")[1]))
\end{minted}
\end{solution}
\item Avec la méthode des trapèzes : \\
sur chaque intervalle, on approxime $f$ par un polynôme de degré 1.\\
Ecrire une fonction \verb?trapeze? qui prend comme entrée deux listes de taille $n+1$ : $(x_0,\cdots x_n)$ et $(y_0,\cdots,y_{n})$ et qui renvoie :
\[\displaystyle \sum_{i=0}^{n-1} (x_{i+1}-x_i)\dfrac{y_{i+1}+y_i}{2}\]
\begin{solution}~\\
\vspace{-1cm}
\begin{minted}[frame=lines]{python}
def trapeze(LX,LY):
    integrale=0
    for i in range(len(LX)-1) :
        integrale=integrale+(LX[i+1]-LX[i])*(Y[i+1]+Y[i])*1./2
    return integrale
\end{minted}
\end{solution}
\item Valeur moyenne et écart-type :\\
Par définition, la valeur moyenne et l'écart-type d'une fonction $f$ sur $[x_0,x_n]$ sont donnés par :
\begin{itemize}
\item valeur moyenne : $M_f=\displaystyle{ \frac{1}{x_n-x_0} \int_{x_0}^{x_n} f(x)dx}$\medskip 
\item écart-type : $\displaystyle{\sigma_f=\sqrt{\dfrac{1}{x_n-x_0}\int_{x_0}^{x_n}(f(x)-M_f)^2dx}}$
\end{itemize}
Ecrire un programme qui calcule une valeur approximative de $\sigma_f$. (On pourra faire appel à la fonction \verb?trapeze?)
\begin{solution}~\\
\vspace{-1cm}
\begin{minted}[frame=lines]{python}
Mf=1./(LX[-1]-LX[0])*trapeze(LX,LY)

LY_sigma=[]
for y in LY:
	LY_sigma.append((y-Mf)**2)

variance=1./(LX[-1]-LX[0])*trapeze(LX,LY_sigma)		
sigma=sqrt(variance)	
\end{minted}

Méthode 2 : 
\begin{minted}[frame=lines]{python}
Mf=1./(LX[-1]-LX[0])*trapeze(LX,LY)

LY_sigma=(LY-Mf)**2

variance=1./(LX[-1]-LX[0])*trapeze(LX,LY_sigma)		
sigma=sqrt(variance)
\end{minted}

\end{solution}
\end{enumerate}



\end{document}
