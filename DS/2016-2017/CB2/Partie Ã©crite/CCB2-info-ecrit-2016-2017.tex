\documentclass[10pt,a4paper]{article}
  \usepackage[french]{babel}
  \usepackage[utf8]{inputenc}
  \usepackage[T1]{fontenc}
  \usepackage{xcolor}
  \usepackage[]{graphicx}
  \usepackage{makeidx}
  \usepackage{textcomp}
  \usepackage{amsmath}
  \usepackage{amssymb}
  \usepackage{stmaryrd}
  \usepackage{fancyhdr}
  \usepackage{lettrine}
  \usepackage{calc}
  \usepackage{boxedminipage}
  \usepackage[french,onelanguage, boxruled,linesnumbered]{algorithm2e}
  \usepackage[colorlinks=false,pdftex]{hyperref}
  \usepackage{minted}
  \usepackage{url}
  \usepackage[locale=FR]{siunitx}
  \usepackage{multicol}
  \usepackage{tikz}
  \makeindex

  %\graphicspath{{../Images/}}

%  \renewcommand\listingscaption{Programme}

  %\renewcommand{\thechapter}{\Alph{chapter}}
  \renewcommand{\thesection}{\Roman{section}}
  %\newcommand{\inter}{\vspace{0.5cm}%
  %\noindent }
  %\newcommand{\unite}{\ \textrm}
  \newcommand{\ud}{\mathrm{d}}
  \newcommand{\vect}{\overrightarrow}
  %\newcommand{\ch}{\mathrm{ch}} % cosinus hyperbolique
  %\newcommand{\sh}{\mathrm{sh}} % sinus hyperbolique

  \textwidth 160mm
  \textheight 250mm
  \hoffset=-1.70cm
  \voffset=-1.5cm
  \parindent=0cm

  \pagestyle{fancy}
  \fancyhead[L]{\bfseries {\large PTSI -- Dorian}}
  \fancyhead[C]{\bfseries{{\type} \no \numero}}
  \fancyhead[R]{\bfseries{\large Informatique}}
  \fancyfoot[C]{\thepage}
  \fancyfoot[L]{\footnotesize R. Costadoat, C. Darreye}
  \fancyfoot[R]{\small \today}
  
  \definecolor{bg}{rgb}{0.9,0.9,0.9}
  
  
  % macro Juliette
  
\usepackage{comment}   
\usepackage{amsthm}  
\theoremstyle{definition}
\newtheorem{exercice}{Exercice}
\newtheorem*{rappel}{Rappel}
\newtheorem*{remark}{Remarque}
\newtheorem*{defn}{Définition}
\newtheorem*{ppe}{Propriété}
\newtheorem{solution}{Solution}

\newcounter{num_quest} \setcounter{num_quest}{0}
\newcounter{num_rep} \setcounter{num_rep}{0}
\newcounter{num_cor} \setcounter{num_cor}{0}

\newcommand{\question}[1]{\refstepcounter{num_quest}\par
~\ \\ \parbox[t][][t]{0.15\linewidth}{\textbf{Question \arabic{num_quest}}}\parbox[t][][t]{0.85\linewidth}{#1\label{q\the\value{num_quest}}}\par
~\ \\}

\newcommand{\reponse}[4][1]
{\noindent
\rule{\linewidth}{.5pt}\\
\textbf{Question\ifthenelse{#1>1}{s}{} \multido{}{#1}{%
\refstepcounter{num_rep}\ref{q\the\value{num_rep}} }:} ~\ \\
\ifdef{\public}{#3 ~\ \\ \feuilleDR{#2}}{#4}
}

\newcommand{\cor}
{\refstepcounter{num_cor}
\noindent
\rule{\linewidth}{.5pt}
\textbf{Question \arabic{num_cor}:} \\
}

\parindent=10pt
\textheight 250mm

  \pagestyle{fancy}
%  \fancyfoot[C]{\thepage}
  \fancyhead[LO,LE]{\bfseries {\large PTSI -- Dorian}}
  \fancyhead[RO,RE]{\bfseries{\large Informatique}}
  \fancyhead[CO,CE]{Concours blanc juin 2017}

\begin{document}

 \begin{center}
  \begin{large}
  \fbox{Concours blanc -- Partie écrite. Durée : 1 heure}
  \end{large}
 \end{center}

\begin{boxedminipage}{\textwidth} 
Lorsqu'on écrit un code Python : faire attention à ce que les indentations soient visibles sur la copie ; commenter le code de façon à expliquer les grandes étapes de l'algorithme en ajoutant un commentaire en fin de ligne de code après le symbole $\sharp$.
\end{boxedminipage}
 
\section{Base de données}

Soit une base de données contenant une table \texttt{Ville} dont le schéma relationnel est le suivant : 

\noindent\texttt{Ville (ville\_id:int, ville\_departement:int, ville\_nom\_reel:char, ville\_code\_postal:int, ville\_population:int, ville\_longitude\_dms:int, ville\_latitude\_dms:int)}

\begin{enumerate}
\item Si on devait représenter les données de la table sous forme de tableau : combien contiendrait-il de colonnes ?                                                                                                            
\item Proposer une requête SQL permettant de lister toutes les informations de la table.
\item Proposer une requête SQL permettant de lister le nom de toutes les villes dont la population est supérieure à mille habitants.
\item Proposer une requête SQL permettant de lister le nom et la population de toutes les villes dont la population est supérieure à mille habitants.
\end{enumerate}

\section{Étude d'un algorithme}

On considère la fonction \texttt{d} définie par le code python suivant :

\begin{listing}
\begin{minted}[linenos,frame=lines]{python}
def d(n):
    L=[]
    for nombre in range(1,n+1):
        if n%nombre==0:
            L.append(nombre)
    return L
\end{minted}
\caption{Fonction à étudier.}
\label{prog:fonctiond}
\end{listing}

\begin{enumerate}
 \item Pourquoi, à la ligne 3, n'écrit-on pas plutôt \texttt{for nombre in range(n+1):} ?
 \item Quel est le résultat de l'appel \texttt{d(4)} ? On détaillera l'évolution pas à pas de la variable \texttt{L} dans la boucle, en justifiant.
 \item Donner, sans justification, le résultat de l'appel \texttt{d(10)}.
 \item Que fait la fonction \texttt{d} ?
 \item Proposer une petite amélioration de la fonction, permettant de toujours réaliser une itération de moins de la boucle \texttt{for} sans jamais changer le résultat de la fonction.
 \item Un diviseur \textit{non trivial} d'un entier \texttt{n} est un diviseur de \texttt{n} différent de \texttt{1} et \texttt{n}. Écrire une fonction \texttt{DNT}, d'argument \texttt{n}, renvoyant la somme des carrés des diviseurs non-triviaux de l'entier \texttt{n}.
 \item Écrire la suite des instructions permettant d'afficher tous les nombres entiers inférieurs à 1000 et égaux à la somme des carrés de leurs diviseurs non-triviaux.
\end{enumerate}

\section{Question de cours -- Méthode d'Euler}

On a le problème de Cauchy suivant :

\begin{equation*}
\left\{\begin{aligned}
\frac{\ud U_c}{\ud t} &= \frac{e(t)-U_c(t)}{RC}\\
U_c(0) &= 0
\end{aligned}
\right.
\end{equation*}

Données : $R = \SI{300}{\ohm}$ ; $e(t) = \SI{6}{\volt} ; $C = \SI{10}{\milli\farad}

Proposer un code écrit en python permettant de résoudre le problème de Cauchy précédent par la méthode d'Euler.
\end{document}


