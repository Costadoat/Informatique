\newcommand{\nom}{Production Éolienne/Solaire}
\newcommand{\sequence}{1}
\newcommand{\numero}{02}
\newcommand{\type}{DS}
\newcommand{\descrip}{Production Éolienne/Solaire}

\documentclass[10pt,a4paper]{article}
  \usepackage[french]{babel}
  \usepackage[utf8]{inputenc}
  \usepackage[T1]{fontenc}
  \usepackage{xcolor}
  \usepackage[]{graphicx}
  \usepackage{makeidx}
  \usepackage{textcomp}
  \usepackage{amsmath}
  \usepackage{amssymb}
  \usepackage{stmaryrd}
  \usepackage{fancyhdr}
  \usepackage{lettrine}
  \usepackage{calc}
  \usepackage{boxedminipage}
  \usepackage[french,onelanguage, boxruled,linesnumbered]{algorithm2e}
  \usepackage[colorlinks=false,pdftex]{hyperref}
  \usepackage{minted}
  \usepackage{url}
  %\usepackage[locale=FR]{siunitx}
  \usepackage{multicol}
  \makeindex

  %\graphicspath{{../Images/}}

%  \renewcommand\listingscaption{Programme}

  %\renewcommand{\thechapter}{\Alph{chapter}}
  \renewcommand{\thesection}{\Roman{section}}
  %\newcommand{\inter}{\vspace{0.5cm}%
  %\noindent }
  %\newcommand{\unite}{\ \textrm}
  \newcommand{\ud}{\mathrm{d}}
  \newcommand{\vect}{\overrightarrow}
  %\newcommand{\ch}{\mathrm{ch}} % cosinus hyperbolique
  %\newcommand{\sh}{\mathrm{sh}} % sinus hyperbolique

  \textwidth 160mm
  \textheight 250mm
  \hoffset=-1.70cm
  \voffset=-1.5cm
  \parindent=0cm

  \pagestyle{fancy}
  \fancyhead[L]{\bfseries {\large PTSI -- Dorian}}
  \fancyhead[C]{\bfseries{{\type} {\num}}}
  \fancyhead[R]{\bfseries{\large Informatique}}
  \fancyfoot[C]{\thepage}
  \fancyfoot[L]{\footnotesize R. Costadoat, J. Genzmer, W. Robert}
  \fancyfoot[R]{\small \today}
  
  \definecolor{bg}{rgb}{0.9,0.9,0.9}
  
  
  % macro Juliette
  
\usepackage{comment}   
\usepackage{amsthm}  
\theoremstyle{definition}
\newtheorem{exercice}{Exercice}
\newtheorem*{rappel}{Rappel}
\newtheorem*{remark}{Remarque}
\newtheorem*{defn}{Définition}
\newtheorem*{ppe}{Propriété}
\newtheorem{solution}{Solution}


\excludecomment{solution}
%\excludecomment{exercice}

\begin{document}

\begin{center}
{\Large\bf {\type} \no {\numero} -- \descrip}
\end{center}

\SetKw{KwFrom}{de}

\fbox{Les codes en python doivent être commentés et les indentations dans le 
code doivent être visibles.}

\section{Question de cours -- Algorithme d'Euclide}

Soient $a \in \mathbb N$ et $b \in \mathbb N^*$. Donner le code en python d'une fonction qui réalise l'algorithme d'Euclide déterminant le plus grand diviseur commun de $a$ et $b$.

\section{Exercice de TP -- Complexités}
%Cf TP4

On suppose que $n$ a été affecté (et vaut un entier strictement positif). Combien d’\textit{opérations élémentaires} les programmes suivant vont-ils exécuter ?

% \begin{boxedminipage}{\textwidth}
% \begin{minted}[]{python}
% s = 0
% for i in range(n):
%   s = s + i**2
% \end{minted}
% \end{boxedminipage}

\begin{enumerate}
\item Premier programme.

\begin{minted}[linenos]{python}
s = 0
for i in range(n):
  for j in range(i,n):
    s = s + j**2
\end{minted}


\item Second programme.

\begin{minted}[linenos]{python}
while n < 10**10:
  n = n * 2
\end{minted}
\end{enumerate}

\section{Exercice -- Nombres heureux}
%D'après MPSI LLG info-llg.fr
%Première question : CB2-machine 2016-2017

\begin{enumerate}

\item Soit \verb?n=5271?. Quel est le quotient, noté \verb?q? dans la division euclidienne de \verb?n? par 10 ? Quel est le reste, noté \verb?r? ? On recommence la division par 10 à partir de \verb?q?. Donner les nouveaux quotient et reste. Que peut-on constater ?

\item Rédiger en Python une fonction \verb?somme_carre(n)? qui prend en argument un entier naturel \verb?n? et qui retourne la somme des carrés de ses chiffres en base 10. (On supposera cette fonction connue et correcte pour la suite de l'exercice et on pourra donc la réutiliser.)
\end{enumerate}

Un entier naturel est dit heureux lorsqu’en faisant la somme des carrés de ses chiffres en base 10 puis en réitérant ce procédé on finit par aboutir à 1. Dans le cas contraire il est dit malheureux. 

Par exemple, 7 est un nombre heureux puisque la suite qui lui est associée est : $t_0= 7$, $t_ 1= 7^2= 49$, $t_2= 4^2+ 9^2= 97$, $t_3= 9^2+ 7^2= 130$,    $t_4= 1^2+ 3^2+ 0^2= 10$, $t_5= 1^2+ 0^2= 1$. 

En revanche, 8 est un nombre malheureux puisque la suite qui lui est associée est : $t_0= 8$, $t_1= 8^2= 64$, $t_2= 6^2+ 4^2= 52$, $t_3= 5^2+ 2^2= 29$, $t_4= 2^2+ 9^2= 85$, $t_5= 8^2+ 5^2= 89$, $t_6= 8^2+ 9^2= 145$, $t_7= 1^2+ 4^2+ 5^2= 42$, $t_8= 4^2+ 2^2= 20$, $t_9= 2^2+ 0^2= 4$, $t_{10}= 4^2 = 16$, $t_{11}= 1^2+ 6^2= 37$, $t_{12}= 3^2+ 7^2= 58$, $t_{13}= 5^2+ 8^2= 89$, $t_{14}= \dots$ et il n’est pas nécessaire de poursuivre le calcul puisque le nombre 89 a déjà été obtenu ; la séquence 89, 145, 42, 20, 4, 16, 37, 58 va se répéter indéfiniment. En fait, il est possible de prouver (et nous admettrons ce résultat) qu’un nombre est malheureux si et seulement s’il atteint l’un des nombres de la séquence exposée ci-dessus. 

\begin{enumerate}
\setcounter{enumi}{2}
\item Rédiger en Python une fonction \verb?heureux(n)? qui prend en argument un entier naturel \verb?n? et qui retourne la valeur booléenne \verb?True? lorsque n est un nombre heureux, et \verb?False? sinon.
\end{enumerate}

\end{document}
