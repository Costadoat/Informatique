\documentclass[a4paper,12pt]{article}

%\usepackage[a4paper]{geometry}
%\geometry{margin={1cm,1.2cm}}
%\usepackage[francais]{babel}
%\usepackage{nopageno} %pas de numérotation de page
%\pagestyle{plain} %numérotation en bas de page, pas d'entête
%\usepackage{hyperref}
%\usepackage[latin1]{inputenc}

%%%%%%%%%%%%%%%%%%%%%%%%%%%%%%%%%%%%%%%%%%%%%%%%%%%%%%%%%%%%%%%%%%%%%%%%%%%%%%%%%%%%%

\usepackage{amsthm}
\usepackage{amscd}
%\usepackage{mathrsfs}
%\usepackage{amsfonts}
%\usepackage[T1]{fontenc}
%\usepackage{theorem}
\usepackage{lscape}
\usepackage{variations}  % pour faire des tableaux de variations
\usepackage{dsfont}
\usepackage{fancyvrb} % pour mettre Verbatim dans une box

% Pour les figures
\usepackage{subfig}
%\usepackage{calc} % Pour pouvoir donner des formules dans les d�finitions de longueur
%\usepackage{graphicx} % Pour inclure des graphiques 
% Attention : pour inclure des .jpg comme dans l'exemple (ou des .png ou .pdf)
% il faut compiler directement en pdf (commande pdflatex).
% Pour inclure des .eps, il faut compiler avec latex + dvips + ps2pdf.
\usepackage{psfrag}
%\usepackage{color}

%%%%%%%%%%%%%%%%%%%%%%%%%%%%%%%%%%%%%%%%%%%%%%%%%%%%%%%%%%%%%%%%%%%%%%%%%%%%%%%%%%%%%

\theoremstyle{definition}
\newtheorem*{thm}{Théorème}
%\theorembodyfont{\rmfamily}
\newtheorem*{defn}{Définition}
\newtheorem{exercice}{Exercice}
\newtheorem*{problem}{Problème}
\newtheorem*{prop}{Proposition}
\newtheorem*{corollaire}{Corollaire}
\newtheorem*{lemme}{Lemme}
\newtheorem*{remark}{Remarque}
\newtheorem*{notation}{Notation}
\newtheorem*{ex}{Exemple}
\newtheorem*{ppe}{Propriété}
\newtheorem*{meth}{Méthode}
\newtheorem*{rappel}{Rappel}
\newtheorem*{voca}{Vocabulaire}
\setlength{\columnseprule}{0.5pt}


%%%%%%%%%%%%%%%%%%%%%%%%%%%%%%%%%%%%%%%%%%%%%%%%%%%%%%%%%%%%%%%%%%%%%%%%%%%%%%%%%%%%%

\newcommand{\bi}{\bigskip}
\newcommand{\dsp}{\displaystyle}
\newcommand{\noi}{\noindent}
\newcommand{\ov}{\overline}
\newcommand{\dsum}{\displaystyle \sum}
\newcommand{\dprod}{\displaystyle \prod}
\newcommand{\dint}{\displaystyle \int}
\newcommand{\dlim}{\displaystyle \lim}

%%%%%%%%%%%%%%%%%%%%%%%%%%%%%%%%%%%%%%%%%%%%%%%%%%%%%%%%%%%%%%%%%%%%%%%%%%%%%%%%%%%%%


%\newcommand{\pgcd}{\mathrm{pgcd}} % pgcd
%\providecommand{\norm}[1]{\lVert#1\rVert} % norme
%\DeclareMathOperator{\Tan}{Tan}  % espace tangent


\newcommand{\N}{\mathbb{N}}
\newcommand{\Z}{\mathbb{Z}}
\newcommand{\Q}{\mathbb{Q}}
\newcommand{\R}{\mathbb{R}}
\newcommand{\C}{\mathbb{C}}
\newcommand{\K}{\mathbb{K}}
\newcommand{\U}{\mathbb{U}}
\newcommand{\Tr}{\text{Tr}\,}
\newcommand{\pg}{\geqslant}
\newcommand{\pp}{\leqslant}
\newcommand{\bul}{\item[$\bullet$]}
\newcommand{\card}{\text{Card}}
\newcommand{\re}{\text{Re}\;}
\newcommand{\im}{\text{Im}\;}
\newcommand{\Ker}{\text{Ker}\;}
\newcommand{\Vect}{\text{Vect}\;}
\newcommand{\rg}{\text{rg}\;}
\newcommand{\TT}{{}^t\!}
%\newcommand{\sh}{\text{sh}}
%\newcommand{\ch}{\text{ch}}
\newcommand{\Mat}{\text{Mat}}
\usepackage{textcomp}



%%%%%%%%%%%%%%%%%%%%%%%%%%%%%%%%%%%%%%%%%%%%%%%%%%%%%%%%%%%%%%%%%%%%%%%%%%%%%%%%%%%%%%%%%%%%%%%%%%%%%%%%%%%%%%%%%%%%%%%%%%%




\excludecomment{solution}



\begin{document}
\noindent {\large PTSI\hfill 2016-2017}\\

\begin{center}
{\LARGE  \textbf{Devoir sur table n°2}  \bigskip }

\rule{2.39cm}{0.05cm}

\bi 

{\large INFORMATIQUE }\bigskip

{\large Jeudi 15 D\' ecembre}

\rule{2.39cm}{0.05cm}
\end{center}


\noi\fbox{ \begin{minipage}{1\textwidth} 
\begin{center}
\textbf{Rappel des consignes}
\end{center}
Lorsqu'on \' ecrit un code Python, :
\begin{itemize}
\item faire attention \` a ce que les indentations soient visibles sur la copie;
\item commenter le code de fa\c con \` a expliquer les grandes \' etapes de l'algorithme en ajoutant un commentaire en fin de ligne de code apr\` es le symbole $\sharp$.
\end{itemize}
\end{minipage}}

\section*{Exercice 1}
\noi Ecrire une fonction \verb?renverser? qui \` a une liste renvoie la liste renvers\' ee.\\
Remarque : On n'utilisera \textbf{pas} la m\' ethode \verb?reverse? d\' ej\` a impl\' ement\' ee dans Python.
\begin{verbatim}
>>>L=[2,8,-1,7]
>>>renverser(L)
[7,-1,8,2]
\end{verbatim}

\section*{Exercice 2}
\begin{enumerate}
\item Ecrire une fonction \verb?maximum(liste)? qui renvoie le maximum d'une liste de nombres non tri\' ee.\\
Remarque : On n'utilisera \textbf{pas} la fonction \verb?max? d\' ej\` a impl\' ement\' ee dans Python.
\begin{verbatim}
>>>L=[2,8,-7,3]
>>>maximum(L)
8
\end{verbatim}
\item Si on note $n$ la longueur de la liste, montrer que la complexit\' e de l'algorithme obtenu est $O(n)$\footnote{C'est-\` a-dire que le nombre d'op\' erations s'\' ecrit : $an+b$}. 
\end{enumerate}


\section*{Exercice 3}
\noi L'objectif de cet exercice est de faire une liste des triangles qui v\' erifient les trois conditions suivantes :
\begin{itemize}
\item les c\^ ot\' es des triangles sont de mesure enti\` ere ;
\item les triangles sont rectangles ;
\item les triangles sont de p\' erim\` etre $p$ (la valeur de $p$ \' etant fix\' ee).
\end{itemize}
\begin{enumerate}
\newpage
\item Une fonction rectangle :
\begin{enumerate}
\item Ecrire une fonction \verb?rectangle(a,b,c)? qui prend comme entr\' ee trois entiers positifs et renvoie \verb?True? si le triangle dont les c\^ ot\' es de mesures $a$, $b$ et $c$ est un triangle rectangle et \verb?False? sinon.\\
Exemple : 
\begin{verbatim}
rectangle(4,3,5)
>>> True
rectangle(2,7,1)
>>> False
\end{verbatim}
\begin{solution}~\\
\begin{verbatim}
def rectangle(a,b,c):
    A=a**2
    B=b**2
    C=c**2
    if A==B+C or B==A+C or C==A+B:
        return(True)
    else:
        return(False)
\end{verbatim}
\end{solution}
\item On note $N_{rect}$ le nombre d'op\' erations de cet algorithme. Calculer $N_{rect}$.
\begin{solution}
\begin{itemize}
\item Trois affectations, trois multiplications
\item trois tests avec trois additions
\item un return 
\end{itemize}
$N_{rect}=13$
\end{solution}
\end{enumerate}
\item Une premi\` ere fonction :
\begin{enumerate}
\item On consid\` ere la fonction \verb?triangle? suivante. Que renvoie-t-elle ?
\begin{verbatim}
def triangle(p):
   Liste=[]
   for a in range(1,p+1):
      for b in range(1,p+1):
         for c in range(1,p+1):
            if a+b+c==p:
               Liste.append((a,b,c))
   return(Liste)               
\end{verbatim}
\begin{solution}
La fonction construit tous les triplets $(a,b,c)$ d'entiers compris entre 1 et $p$. Elle ne garde que ceux qui sont les mesures d'un triangle rectangle. Le p\' erim\` etre de ces triangles par contre n'est pas fix\' e. Il peut varier entre 3 et $3p$. 
\end{solution}
\item Modifier la fonction \verb?triangle? en une fonction \verb?triangle2? pour qu'elle renvoie la liste des triangles qui v\' erifient les trois conditions \' enonc\' ees au d\' ebut de l'exercice.
\begin{solution}
\begin{verbatim}
def triangle2(p):
   Liste=[]
   for a in range(1,p+1):
      for b in range(1,p+1):
         for c in range(1,p+1):
            if a+b+c==p:
               if rectangle(a,b,c):                  
                  Liste.append((a,b,c))
   return(Liste)               
\end{verbatim}
\end{solution}
\item D\' eterminer le nombre d'op\' erations effectu\' ees dans la fonction \verb?triangle2?. 
\begin{solution}
Pour la fonction \verb?triangle2? :
\begin{itemize}
\item une assignation
\item $p$ it\' erations de chaque boucle, donc $p^3$ it\' erations de :
\begin{itemize}
\item 2 additions
\item 1 test
\item fonction rectangle : $N_{rect}$=13
\item 1 test 
\item un append
\end{itemize}
\item un return 
\end{itemize}
Le nombre d'op\' erations du premier algorithme est : \fbox{$N_{op}=2+18p^3$}
\end{solution}
\end{enumerate}
\item Une deuxi\` eme fonction :
\begin{enumerate}
\item On introduit la fonction suivante :
\begin{verbatim}
def triangle3(p):
   Liste=[]
   for a in range(1,p//3+1):
      for b in range(a,(p-a)//2+1):
            if rectangle(a,b,p-a-b):
               Liste.append((a,b,p-a-b))
   return(Liste)                  
\end{verbatim}
\noi Expliquer pourquoi cette fonction renvoie aussi la liste des triangles cherch\' es.
\begin{solution}
La liste renvoy\' ee contient des triplets $(a,b,c)$ tels que $(a,b,c)$ v\' erifient la condition \verb?rectangle? et $c=p-a-b$. Donc ces triplets sont associ\' es \` a des triangles rectangles de p\' erim\` etre $p$. \\
V\' erifions qu'on obtient bien tous les triangles cherch\' es avec cette fonction :\\
\verb?a,b? et \verb?c? sont les trois longueurs des c\^ ot\' es. On choisit qu'ils seront dans l'ordre croissant : $a\pp b\pp c$.\\
Le plus petit des c\^ ot\' es est n\' ecessairement inf\' erieur \` a $\frac{p}{3}$. En effet, si $a>\frac{p}{3}$, alors le p\' erim\` etre est : $a+b+c>3\times p/3=p$. Le p\' erim\` etre du triangle ne peut pas \^ etre $p$. \\
Comme $a$ est un entier : $a\pp  \lfloor \frac{p}{3}\rfloor$. Donc $a$ parcourt : $1,2,\cdots \lfloor \frac{p}{3}\rfloor$.\\
\verb?p//3? renvoie le quotient dans la division euclidienne de $p$ par 3, c'est donc $\lfloor \frac{p}{3}\rfloor$. Alors \verb?range(1,p//3+1)? est la liste $[1,2,3,\cdots, \lfloor \frac{p}{3}\rfloor]$.\\
Maintenant que $a$ est fix\' e, les deux autres longueurs v\' erifient : $b\pg a$ et $b+c=p-a$. Encore une fois, l'une des deux longueurs est n\' ecessairement inf\' erieures \` a $\frac{p-a}{2}$ (si $b$ et $c$ sont strictement plus grand que $\frac{p-a}{2}$, alors $a+b+c>p$.) On choisit : $b\pp \frac{p-a}{2}$. Donc \verb?b in range(a,(p-a)//2+1)?.\\
Enfin, comme la longueur du p\' erim\` etre est fix\' ee : $c=p-a-b$. \\
On obtient donc bien la liste cherch\' ee.
\end{solution}
\item Montrer que le nombre d'op\' erations $N_{op}$ de la fonction \verb?triangle3? v\' erifie :
\[N_{op}\pp 2+17\left( \dfrac{p^2}{12}+\dfrac{p}{12} \right)\]
\begin{solution}
Second algorithme :
\begin{itemize}
\item une assignation
\item $\lfloor \frac{p}{3}\rfloor $ it\' erations de la premi\` ere boucle,
\item $\lfloor \frac{p-a}{2}\rfloor $ it\' erations de la deuxi\` eme boucle,
\begin{itemize}
\item 2 soustractions
\item fonction rectangle : $N_{rect}$=13
\item 1 test 
\item un append
\end{itemize}
\item un return
\end{itemize}
La complexit\' e du premier algorithme est : $N_{op}=2+\dsum_{a=1}^{\lfloor \frac{p}{3}\rfloor}\dsum_{b=a}^{\lfloor \frac{p-a}{2}\rfloor} 17$.\\
On utilisera l'encadrement : $x-1\pp \lfloor x\rfloor \pp x$.\\
$N_{op}=2+17\dsum_{a=1}^{\lfloor \frac{p}{3}\rfloor} (\lfloor \frac{p-a}{2}\rfloor -a+1)\pp 2+17 \dsum_{a=1}^{\lfloor \frac{p}{3}\rfloor} (\frac{p-a}{2}+1-a)$\\
$N_{op}\pp 2+17 \dsum_{a=1}^{\lfloor \frac{p}{3}\rfloor} (\frac{p}{2}+1-\frac{3}{2}a)=2+17\left[(\frac{p}{2}+1)(\lfloor \frac{p}{3}\rfloor)-\frac{3}{2}\dfrac{\lfloor \frac{p}{3}\rfloor(\lfloor \frac{p}{3}\rfloor+1)}{2}\right]$.\\
$N_{op}\pp 2+17 \dfrac{p}{3}\left[(\frac{p}{2}+1)-\frac{3}{2}\dfrac{(\frac{p}{3}+1)}{2}\right]=2+17(\dfrac{p^2}{12}+\dfrac{p}{12} )$.\\
\fbox{$N_{op}\pp 2+17\left( (\dfrac{p^2}{12}+\dfrac{p}{12} \right)$}
\end{solution}
\item Comparer la complexit\' e des deux algorithmes. Lequel est le plus rapide pour des grandes valeurs de $p$ ?
\begin{solution}
Pour le premier, $N_{op}=2+18p^3$. Donc la complexit\' e est en $O(p^3)$.\\
Pour le second, $N_{op}\pp 2+17\left( (\dfrac{p^2}{12}+\dfrac{p}{12} \right)$. Donc la complexit\' e est en $O(p^2)$.\\
Quand $p$ est grand, le deuxi\` eme algorithme est le plus rapide.
\end{solution}
\end{enumerate} 
\end{enumerate}



\end{document}