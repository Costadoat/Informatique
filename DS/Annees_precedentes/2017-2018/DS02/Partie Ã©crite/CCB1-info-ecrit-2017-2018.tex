\documentclass[10pt,a4paper]{article}
  \usepackage[french]{babel}
  \usepackage[utf8]{inputenc}
  \usepackage[T1]{fontenc}
  \usepackage{xcolor}
  \usepackage[]{graphicx}
  \usepackage{makeidx}
  \usepackage{textcomp}
  \usepackage{amsmath}
  \usepackage{amssymb}
  \usepackage{stmaryrd}
  \usepackage{fancyhdr}
  \usepackage{lettrine}
  \usepackage{calc}
  \usepackage{boxedminipage}
  \usepackage[french,onelanguage, boxruled,linesnumbered]{algorithm2e}
  \usepackage[colorlinks=false,pdftex]{hyperref}
  \usepackage{minted}
  \usepackage{url}
  \usepackage[locale=FR]{siunitx}
  \usepackage{multicol}
  \usepackage{tikz}
  \makeindex

  %\graphicspath{{../Images/}}

%  \renewcommand\listingscaption{Programme}

  %\renewcommand{\thechapter}{\Alph{chapter}}
  \renewcommand{\thesection}{\Roman{section}}
  %\newcommand{\inter}{\vspace{0.5cm}%
  %\noindent }
  %\newcommand{\unite}{\ \textrm}
  \newcommand{\ud}{\mathrm{d}}
  \newcommand{\vect}{\overrightarrow}
  %\newcommand{\ch}{\mathrm{ch}} % cosinus hyperbolique
  %\newcommand{\sh}{\mathrm{sh}} % sinus hyperbolique

  \textwidth 160mm
  \textheight 250mm
  \hoffset=-1.70cm
  \voffset=-1.5cm
  \parindent=0cm

  \pagestyle{fancy}
  \fancyhead[L]{\bfseries {\large PTSI -- Dorian}}
  \fancyhead[C]{\bfseries{{\type} \no \numero}}
  \fancyhead[R]{\bfseries{\large Informatique}}
  \fancyfoot[C]{\thepage}
  \fancyfoot[L]{\footnotesize R. Costadoat, C. Darreye}
  \fancyfoot[R]{\small \today}
  
  \definecolor{bg}{rgb}{0.9,0.9,0.9}
  
  
  % macro Juliette
  
\usepackage{comment}   
\usepackage{amsthm}  
\theoremstyle{definition}
\newtheorem{exercice}{Exercice}
\newtheorem*{rappel}{Rappel}
\newtheorem*{remark}{Remarque}
\newtheorem*{defn}{Définition}
\newtheorem*{ppe}{Propriété}
\newtheorem{solution}{Solution}

\newcounter{num_quest} \setcounter{num_quest}{0}
\newcounter{num_rep} \setcounter{num_rep}{0}
\newcounter{num_cor} \setcounter{num_cor}{0}

\newcommand{\question}[1]{\refstepcounter{num_quest}\par
~\ \\ \parbox[t][][t]{0.15\linewidth}{\textbf{Question \arabic{num_quest}}}\parbox[t][][t]{0.85\linewidth}{#1\label{q\the\value{num_quest}}}\par
~\ \\}

\newcommand{\reponse}[4][1]
{\noindent
\rule{\linewidth}{.5pt}\\
\textbf{Question\ifthenelse{#1>1}{s}{} \multido{}{#1}{%
\refstepcounter{num_rep}\ref{q\the\value{num_rep}} }:} ~\ \\
\ifdef{\public}{#3 ~\ \\ \feuilleDR{#2}}{#4}
}

\newcommand{\cor}
{\refstepcounter{num_cor}
\noindent
\rule{\linewidth}{.5pt}
\textbf{Question \arabic{num_cor}:} \\
}

\parindent=10pt
\textheight 250mm

  \pagestyle{fancy}
%  \fancyfoot[C]{\thepage}
  \fancyhead[LO,LE]{\bfseries {\large PTSI -- Dorian}}
  \fancyhead[RO,RE]{\bfseries{\large Informatique}}
  \fancyhead[CO,CE]{DS 16 décembre 2017}

\begin{document}

 \begin{center}
  \begin{large}
  \fbox{DS Informatique -- Partie écrite. Durée : 1 heure}
  \end{large}
 \end{center}

\begin{boxedminipage}{\textwidth} 
Lorsqu'on écrit un code Python : faire attention à ce que les indentations soient visibles sur la copie ; commenter le code de façon à expliquer les grandes étapes de l'algorithme en ajoutant un commentaire en fin de ligne de code après le symbole $\sharp$.
\end{boxedminipage}
 

\section{Question de TP -- Comparaison entre deux méthodes d'exponentiation}
\label{sec:ComparaisonExponentiation}
%wack page 104 + cours Renaud C05

On propose deux programmes différents pour l'exponentiation d'un réel positif $k$ par un entier strictement positif $n$ (c'est-à-dire le calcul de $k^n$). On suppose que $k$ est affecté d'une valeur réelle positive.

\begin{listing}
\begin{minted}[linenos,frame=lines]{python}
p = 1
c = n
while c > 0:
  p = p * k
  c = c - 1
\end{minted}
\caption{Première méthode d'exponentiation.}
\label{prog:exponentiationnaive}
\end{listing}

\begin{listing}
\begin{minted}[linenos,frame=lines]{python}
p = 1
c = n
while c > 0:
  if c%2 == 1:
    p = p * k
  k = k**2  
  c = c//2
\end{minted}
\caption{Deuxième méthode d'exponentiation.}
\label{prog:exponentiationrapide}
\end{listing}

\begin{enumerate}

\item Pour le programme~\ref{prog:exponentiationnaive}, identifier et prouver le variant et l'invariant de boucle.

\item On prend $n=13$. Pour chacun des deux programmes~\ref{prog:exponentiationnaive} et \ref{prog:exponentiationrapide}, déterminer et noter les valeurs ou expressions successives de $p$, $c$ et $k$ ainsi que le nombre d'itérations de la boucle Tant que.  

\item Compter le nombre d'opérations élémentaires dans chacune des boucles. En déduire lequel des deux programmes réalise une « exponentiation rapide ». 

\item On peut montrer que la \textit{complexité temps} du programme d'exponentiation rapide est $O(\log(n))$. Quelle est la \textit{complexité temps} de l'autre programme ?

\end{enumerate}

\section{Question de cours -- Algorithme d'Euclide}

Soient $a \in \mathbb N$ et $b \in \mathbb N^*$. Donner le code en python d'une fonction qui réalise l'algorithme d'Euclide déterminant le plus grand diviseur commun de $a$ et $b$.

\section{Exercice -- Expression de durées}
%http://pcsi.kleber.free.fr/IPT/doc/TP03_petits_exercices.pdf
% E.Bougnol, J.J. Fleck, M. Heckmann et M. Kostyra, Kléber, PCSI

On suppose déjà connue du programme deux listes \texttt{L1} et \texttt{L2} auxquelles sont affectées des durées $d_1$ et $d_2$ en jours, heures, minutes et secondes sous la forme de quatre entiers \texttt{[jours,heures,minutes,secondes]}. Donner un code python permettant d'exprimer la durée totale $d=d_1+d_2$ en jours, heures, minutes et secondes de façon unique. Par exemple : pour \texttt{L1=[1,23,45,57]} et \texttt{L2=[1,4,17,28]}, le résultat doit être la liste \texttt{[3,4,3,25]}.

Piste suggérée : d'abord convertir les durées en secondes puis convertir la somme des durées en jours, heures, minutes et secondes.

\end{document}


