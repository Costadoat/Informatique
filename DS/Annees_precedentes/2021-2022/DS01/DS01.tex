\newcommand{\nom}{Production Éolienne/Solaire}
\newcommand{\sequence}{1}
\newcommand{\numero}{02}
\newcommand{\type}{DS}
\newcommand{\descrip}{Production Éolienne/Solaire}

\documentclass[10pt,a4paper]{article}
  \usepackage[french]{babel}
  \usepackage[utf8]{inputenc}
  \usepackage[T1]{fontenc}
  \usepackage{xcolor}
  \usepackage[]{graphicx}
  \usepackage{makeidx}
  \usepackage{textcomp}
  \usepackage{amsmath}
  \usepackage{amssymb}
  \usepackage{stmaryrd}
  \usepackage{fancyhdr}
  \usepackage{lettrine}
  \usepackage{calc}
  \usepackage{boxedminipage}
  \usepackage[french,onelanguage, boxruled,linesnumbered]{algorithm2e}
  \usepackage[colorlinks=false,pdftex]{hyperref}
  \usepackage{minted}
  \usepackage{url}
  %\usepackage[locale=FR]{siunitx}
  \usepackage{multicol}
  \makeindex

  %\graphicspath{{../Images/}}

%  \renewcommand\listingscaption{Programme}

  %\renewcommand{\thechapter}{\Alph{chapter}}
  \renewcommand{\thesection}{\Roman{section}}
  %\newcommand{\inter}{\vspace{0.5cm}%
  %\noindent }
  %\newcommand{\unite}{\ \textrm}
  \newcommand{\ud}{\mathrm{d}}
  \newcommand{\vect}{\overrightarrow}
  %\newcommand{\ch}{\mathrm{ch}} % cosinus hyperbolique
  %\newcommand{\sh}{\mathrm{sh}} % sinus hyperbolique

  \textwidth 160mm
  \textheight 250mm
  \hoffset=-1.70cm
  \voffset=-1.5cm
  \parindent=0cm

  \pagestyle{fancy}
  \fancyhead[L]{\bfseries {\large PTSI -- Dorian}}
  \fancyhead[C]{\bfseries{{\type} {\num}}}
  \fancyhead[R]{\bfseries{\large Informatique}}
  \fancyfoot[C]{\thepage}
  \fancyfoot[L]{\footnotesize R. Costadoat, J. Genzmer, W. Robert}
  \fancyfoot[R]{\small \today}
  
  \definecolor{bg}{rgb}{0.9,0.9,0.9}
  
  
  % macro Juliette
  
\usepackage{comment}   
\usepackage{amsthm}  
\theoremstyle{definition}
\newtheorem{exercice}{Exercice}
\newtheorem*{rappel}{Rappel}
\newtheorem*{remark}{Remarque}
\newtheorem*{defn}{Définition}
\newtheorem*{ppe}{Propriété}
\newtheorem{solution}{Solution}


\usepackage{enumitem}

\setenumerate[1]{align=left,label=\arabic*}
\setenumerate[2]{before=\stepcounter{enumi},label*=.\arabic*,leftmargin=1.2em,align=left}


\ifdef{\public}{\excludecomment{solution}}


\begin{document}

\begin{center}
{\Large\bf {\type} \no {\numero} -- \descrip}
\end{center}

\SetKw{KwFrom}{de} 

\begin{boxedminipage}{.9\textwidth} 
\begin{itemize}
 \item Faire tous les exercices dans un même fichier {NomPrenom.py} à sauvegarder,
 \item mettre en commentaire l'exercice et la question traités (ex: \# Exercice 1),
 \item ne pas oublier pas de commenter ce qui est fait dans votre code (ex: \# Je crée une fonction pour calculer la racine d'un nombre),
% \item il est possible de demander un déblocage pour une question, mais celle-ci sera notée 0,
 \item il faut vérifier avant de partir que le code peut s'exécuter et qu'il affiche les résultats que vous attendez.
\end{itemize}
\end{boxedminipage}

\section*{Recherche des racines d'un polynôme}

Soit le polynôme suivant $f(x)=x^4-79.x^2-66.x+432$, on cherche à déterminer ses racines (qui sont toutes entières et comprises dans l'intervalle $\left[-10,10\right]$.

\begin{enumerate}
\item Créer la fonction \verb?f(x)? qui reçoit en entrée la valeur $x$ et retourne en sortie la valeur de $f(x)$.

On donne l'exemple suivant pour l'écriture d'une fonction:

\begin{minted}{python}
def f(x):
    return x
\end{minted}

\begin{solution}~\ \\
\begin{minted}{python}
def f(x):
    return x**4-79*x**2-66*x+432
\end{minted}
\end{solution}

\item Afficher le résultat de $f(0)$, cette valeur attendue doit vous permettre de valider votre fonction.

\begin{solution}~\ \\
\begin{minted}{python}
print(f(0))
\end{minted}
\end{solution}

\item A l'aide d'une fonction \verb?range? et d'une boucle \verb?for?, tester pour $x$ toutes les valeurs de $-10$ à $10$ (incluses) la valeur de $f(x)$:
\begin{itemize}
 \item si $f(x)==0$, afficher la valeur de x,
 \item si $f(x)>0$, afficher '+',
 \item si $f(x)<0$, afficher '-'.
\end{itemize}

\begin{solution}~\ \\
\begin{minted}{python}
for x in range(-10,11,1):
    if f(x)==0:
        print(x)
    elif  f(x)<0:
        print('-')
    else:
        print('+')
\end{minted}
\end{solution}

\section*{Affichage du colloscope}

On souhaite créer une fonction qui renvoie les matières que l'on a en colle en fonction de la semaine :
\begin{itemize}
\item Si le numéro de la semaine est impair afficher "Colles de Math et de Physique",
\item Si le numéro de la semaine est pair afficher "Colles de SI et d'Anglais",
\item Si le numéro de la semaine est un multiple de 8, afficher "Colles de SI, d'Anglais et de Français" et ne pas afficher les deux possibilités précédentes.
\end{itemize}

 \item Écrire l'algorithme correspondant en python et le tester pour la semaine 3. 

\begin{solution}~\ \\
\begin{minted}{python}
semaine=3

if semaine%8==0:
    print("Colles de SI, d'Anglais et de Français")
elif semaine%2==0:
    print("Colles de SI et d'Anglais")
else:
    print("Colles de Math et de Physique")
\end{minted}
\end{solution}

 \item Transformer cet algorithme en une fonction \verb?colloscope(semaine)? et afficher le résultats pour les semaines $4,8$ et $9$.

\begin{solution}~\ \\
\begin{minted}{python}
def colloscope(semaine):
    if semaine%8==0:
        return "Colles de SI, d'Anglais et de Français"
    elif semaine%2==0:
        return "Colles de SI et d'Anglais"
    else:
        return "Colles de Math et de Physique"

print(colloscope(4))
print(colloscope(8))
print(colloscope(9))
\end{minted}
\end{solution}

\section*{Fonction permutation}

On souhaite créer ici une fonction \verb?permuter(a,b)? qui retourne \verb?b,a?.

La solution suivante a été proposée, mais le résultat n'est pas celui attendu.

\begin{minted}{python}
def permuter(a,b):
    a=b
    b=a
    return(a,b)
\end{minted}

\item Proposer une fonction \verb?permuter(a,b)?, qui retourne réellement \verb?b,a? et tester le résultat en affichant \verb?permuter(1,2)?

\begin{solution}~\ \\
\begin{minted}{python}
def permuter(a,b):
    return(b,a)
    
def permuter(a,b): # Deuxième solution
    x=a
    a=b
    b=x
    return(a,b)
\end{minted}
\end{solution}

\section*{Compter sur Verlaine}

On part de l'extrait suivant de \og Mon rêve familier \fg de Paul Verlaine:\\
 \verb?texte="Je fais souvent ce rêve étrange et pénétrant."?

Le code suivant permet d'afficher chaque lettre de ce texte.
\begin{minted}{python}
for lettre in texte:
    print(lettre)
\end{minted}

\item Proposer un script qui permet de compter le nombre de 'e' présents dans l'extrait (on considérera comme 'e' tous ses dérivées 'é', 'è', 'ë' et 'ê'). Le comptage devra être automatique et le code utilisé devra pouvoir être utilisé quel que soit l'extrait en entrée.

\begin{solution}~\ \\
\begin{minted}{python}
count=0
for lettre in texte:
    if lettre == 'e' or lettre == 'é' or lettre == 'ê':
        count=count+1
print(count)
\end{minted}
\end{solution}

\end{enumerate}
\end{document}
