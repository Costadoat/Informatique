\newcommand{\nom}{Production Éolienne/Solaire}
\newcommand{\sequence}{1}
\newcommand{\numero}{02}
\newcommand{\type}{DS}
\newcommand{\descrip}{Production Éolienne/Solaire}

\documentclass[10pt,a4paper]{article}
  \usepackage[french]{babel}
  \usepackage[utf8]{inputenc}
  \usepackage[T1]{fontenc}
  \usepackage{xcolor}
  \usepackage[]{graphicx}
  \usepackage{makeidx}
  \usepackage{textcomp}
  \usepackage{amsmath}
  \usepackage{amssymb}
  \usepackage{stmaryrd}
  \usepackage{fancyhdr}
  \usepackage{lettrine}
  \usepackage{calc}
  \usepackage{boxedminipage}
  \usepackage[french,onelanguage, boxruled,linesnumbered]{algorithm2e}
  \usepackage[colorlinks=false,pdftex]{hyperref}
  \usepackage{minted}
  \usepackage{url}
  %\usepackage[locale=FR]{siunitx}
  \usepackage{multicol}
  \makeindex

  %\graphicspath{{../Images/}}

%  \renewcommand\listingscaption{Programme}

  %\renewcommand{\thechapter}{\Alph{chapter}}
  \renewcommand{\thesection}{\Roman{section}}
  %\newcommand{\inter}{\vspace{0.5cm}%
  %\noindent }
  %\newcommand{\unite}{\ \textrm}
  \newcommand{\ud}{\mathrm{d}}
  \newcommand{\vect}{\overrightarrow}
  %\newcommand{\ch}{\mathrm{ch}} % cosinus hyperbolique
  %\newcommand{\sh}{\mathrm{sh}} % sinus hyperbolique

  \textwidth 160mm
  \textheight 250mm
  \hoffset=-1.70cm
  \voffset=-1.5cm
  \parindent=0cm

  \pagestyle{fancy}
  \fancyhead[L]{\bfseries {\large PTSI -- Dorian}}
  \fancyhead[C]{\bfseries{{\type} {\num}}}
  \fancyhead[R]{\bfseries{\large Informatique}}
  \fancyfoot[C]{\thepage}
  \fancyfoot[L]{\footnotesize R. Costadoat, J. Genzmer, W. Robert}
  \fancyfoot[R]{\small \today}
  
  \definecolor{bg}{rgb}{0.9,0.9,0.9}
  
  
  % macro Juliette
  
\usepackage{comment}   
\usepackage{amsthm}  
\theoremstyle{definition}
\newtheorem{exercice}{Exercice}
\newtheorem*{rappel}{Rappel}
\newtheorem*{remark}{Remarque}
\newtheorem*{defn}{Définition}
\newtheorem*{ppe}{Propriété}
\newtheorem{solution}{Solution}


\ifdef{\prive}{}{\excludecomment{solution}}

\begin{document}

\begin{center}
{\Large\bf {\type} \no {\numero} -- \descrip}
\end{center}

\SetKw{KwFrom}{de} 

\begin{boxedminipage}{.9\textwidth} 
\begin{itemize}
 \item Faire tous les exercices dans un même fichier {NomPrenom.py} à sauvegarder ;
 \item mettre en commentaire l'exercice traité (ex: \# Exercice 1) ;
% \item ne pas oublier pas de commenter ce qui est fait dans votre code (ex: \# Je créé une fonction pour calculer la racine d'un nombre) ;
 \item il est possible de demander un déblocage pour une question \textbf{entre 8h et 14h} sur la classe virtuelle de vos enseignants. Il vous sera envoyé par mail ;
  \item \textbf{vous avez 3h} pour faire le sujet ;
 \item il faut vérifier avant de téléverser votre fichier que le code peut s'exécuter et qu'il affiche les résultats que vous attendez ;
 \item téléversez votre fichier réponse .py sur le site de dépôt de fichier de la PTSI.
\end{itemize}
\end{boxedminipage}





\section*{Exercice 1} 
Soit la suite $(u_n)_{n\in\mathbb{N}}$  définie par :
$$u_0=N~(\text{entier naturel non nul)~~et~~}\forall n\in \mathbb{N},~~u_{n+1}=\begin{cases} u_n/2 \text{ si }u_n\text{ est pair } \\ 3u_n+1\text{ si }u_n\text{ est impair}\end{cases}$$
\begin{enumerate}
\item \`A l'aide notamment des instructions \texttt{plot} et \texttt{show} de la bibliothèque \texttt{matplotlib.pyplot}, représenter graphiquement les $50$ premiers termes de la suite $(u_n)$ avec $N=7$, puis $N=18$.
\item On conjecture que pour tout entier naturel $N$ non nul, il existe un plus petit entier $n$  tel que $u_n=1$. Que se passe-t-il si c'est le cas ?  (on mettre la réponse en commentaire).\\
Si cet entier existe, on l'appelle ``durée de vol'' de la suite $(u_n)$ de valeur initiale $N$.
\item \'Ecrire une fonction \texttt{vol} d'argument un entier $N$, renvoyant la durée de vol de la suite $(u_n)$ de valeur initiale $N$.
\item Représenter la durée de vol en fonction du point de départ $N$, pour les valeurs de $N$ inférieures à $1000$.
\item Représenter la distribution des durées de vol à l'aide d'un histogramme (on pourra utiliser la commande \texttt{hist} de \texttt{matplotlib.pyplot}).
\end{enumerate}




\section*{Exercice 2}
Pour un entier naturel $n\geq 2$, on appelle \textbf{diviseurs propres de $n$} les entiers naturels strictement inférieurs à $n$ qui divisent $n$. \\
Par exemple la liste des diviseurs propres de $100$ est $[1,2,4,5,10,20,25,50]$.\\
On va s'intéresser à la somme de ces diviseurs propres. Pour $100$, elle vaut par exemple $117$.
\begin{enumerate}
\item \'Ecrire une fonction \texttt{LDP} d'argument un entier naturel $n$ qui renvoie la liste de ses diviseurs propres. La tester pour $n=100$.
\item \'Ecrire une fonction \texttt{SDP} d'argument un entier naturel $n$ qui renvoie la somme de ses diviseurs propres. La tester pour $n=100$.
\item On dit qu'un entier naturel est \textbf{parfait} s'il est égal à la somme de ses diviseurs propres. \'Ecrire une fonction \texttt{parfaits} d'argument un entier naturel $K$ qui renvoie la liste des entiers $p$ parfaits inférieurs ou égaux à $K$. La tester pour $K=2000$.
\item On dit que deux entiers distincts sont \textbf{amicaux} si chacun est égal à la somme des diviseurs propres de l'autre. \'Ecrire une fonction \texttt{amicaux} d'argument un entier naturel $K$ qui renvoie la liste de tous les couples $(p,q)$ d'entiers amicaux tels que $p<q \leq K$. Tester \texttt{amicaux} pour $K=300$, $K=1300$, $K=5000$.\\
Remarque : si le temps de calcul est trop long, il faut peut-être optimiser votre programme. %reponse : [(220, 284), (1184, 1210), (2620, 2924)]
\end{enumerate}




\section*{Exercice 3}
\begin{enumerate}
\item Télécharger le fichier \texttt{algo-pi.txt} qui est avec l'énoncé et enregistrer-le dans le dossier où vous travaillez.
\item Le fichier \texttt{algo-pi.txt} contient les premières décimales de $\pi$, sur une seule ligne et sans espace entre les chiffres.\\
Récupérer le contenu de ce fichier sous forme d'une chaîne de caractères que l'on nommera \texttt{decpi}.
\item Faire afficher les $10$ premiers caractères de \texttt{decpi}, ses \texttt{10} derniers, ainsi que le nombre \texttt{nbdec} de caractères de \texttt{decpi}.
\item Écrire une fonction \verb?estIci? de trois arguments : deux chaînes de caractères $P$ et $M$ et un entier \verb?i? et qui renvoie \verb?True? si \verb?M? est dans \verb?P? \` a la position \verb?i? et \verb?-1? sinon.\\ 
Par exemple \texttt{estIci('le','Bonjour le monde',8)} renvoie \texttt{True}.
\item Écrire une fonction \texttt{trouve} de deux arguments, deux chaînes de caractères $P$ et $M$, qui renvoie un entier naturel $p$ si $M$ est une sous-chaîne de $P$ commençant à la position $p$, et $-1$ si $M$ n'est pas
une sous-chaîne de $P$. On n'utilisera pas la méthode \texttt{find} de la classe \texttt{str}.\\
Par exemple, \texttt{trouve("BanquePT","an")} donne \texttt{1}, \texttt{trouve("BanquePT","PT")} donne \texttt{6}, alors que \texttt{trouve("BanquePT","PSI")} donne \texttt{-1}. Comparer avec \texttt{"BanquePT".find("an")}, \textit{etc}.
\item Les nombres \texttt{"14159"}, \texttt{"123456"}, \texttt{"12345"}, et \texttt{"1789"} se trouvent-ils dans \texttt{decpi} ?
\item Si les nombres précédents sont dans \texttt{decpi}, en combien d'exemplaires et à quelle(s) position(s) sont-ils ?
%\item Le fichier \texttt{algo-pi-false.txt} se trouvant dans le répertoire \texttt{data}GNA contient les premières décimales de $\pi$, avec quelques chiffres erronés. Combien et en quelles positions ?
\end{enumerate}







\section*{Exercice 4}
Pour tout ensemble non vide, identifié ici à une liste d'éléments deux à deux distincts, $E = [e_0,\dots , e_{n-1}]$ de taille $n$, chacune de ses parties $A$ peut être codée sous forme d'une liste \texttt{C} à $n$ éléments contenant des zéros et des uns :\\
$$\text{\texttt{C[i] = 1} si $e_i\in A$, et \texttt{C[i] = 0} sinon.}$$
Par exemple, les parties $\emptyset$, $[a]$, $[b]$ et $[a, b]$ de la liste $[a, b]$ sont respectivement codées par les listes $[0, 0]$, $[1, 0]$, $[0, 1]$ et $[1, 1]$.
\begin{enumerate}
\item Écrire une fonction \texttt{decoder} de deux arguments $E$ et $C$ renvoyant la partie de la liste $E$  codée par le code $C$.\\
Ainsi, \texttt{decoder([2,3,5,7],[1,0,0,1])} donne \texttt{[2,7]} ; de même \texttt{decoder([2,3,5,7],[0,0,0,0])}
donne \texttt{[]}.
\item Écrire une fonction \texttt{coder} de deux arguments $E$ et $A$ renvoyant le code de la partie $A$ de la liste $E$.\\
Ainsi \texttt{coder([2,3,5,7],[2,7])} donne \texttt{[1,0,0,1]}.
\item Écrire une fonction \texttt{incrementer} d'argument une liste $C$ de zéros et de uns de taille $n$, représentant sur $n$ bits l'écriture en base $2$ d'un entier naturel $k$, et renvoyant la liste représentant sur $n$ bits l'écriture en base $2$ de l'entier $(k + 1)$.\\
Par exemple, \texttt{[0,0,1,0,1,1,1]} est l'écriture en base $2$ sur $7$ bits de $23$ et \texttt{[0,0,1,1,0,0,0]}, l'écriture en base $2$ sur $7$ bits de $24$.\\
Remarque : on considère que $k$ sera toujours pris inférieur à $2^n-1$ de sorte que le problème du nombre qui suit \texttt{[1,1,1,...,1,1]} ne se pose pas.
\item En déduire la fonction \texttt{parties} d'argument $E$ renvoyant la liste des parties de la liste $E$.
%\item Écrire une fonction \texttt{p\_parties} de deux arguments, un ensemble $E$ non vide de taille $n$ et un entier naturel $p\leqslant n$, qui renvoie la liste des parties de $E$ de taille $p$.
\end{enumerate}










 \end{document}
 
 
 




\begin{exercice}
Dans la liste des entiers naturels non nuls, on barre un nombre sur $2$ en commençant par barrer le deuxième :
$$ 1~,~ \cancel{2}~,~3~,~\cancel{4}~,~5~,~\cancel{6}~,~7~,~\cancel{8}~,~9~,~\cancel{10}~,~11~,~\cancel{12}~,~13~,~\cdots$$
Puis dans la liste restante, on barre une nombre sur $3$ en commençant par barrer le troisième :
$$1~,~3~,~\cancel{5}~,~7~,~9~,~\cancel{11}~,~13~,~\cdots$$
puis on barre un nombre sur $4$, un nombre sur $5$, etc.\\
Et ceci à l'infini pour obtenir ``\textit{la liste des nombres de type $J$''}.
\begin{enumerate}
\item \'Ecrire une fonction \texttt{enlever} de deux arguments, une liste $L$ et un entier naturel $i$, qui renvoie une liste $S$ construite en ne gardant dans la liste $L$ que les éléments dont le rang n'est pas multiple de $i$. Par exemple, \texttt{enlever([1,3,5,7,9,11,13],3)} doit donner \texttt{[1,3,7,9,13]}.
\item \'Ecrire une suite d'instructions donnant la liste des nombres de type $J$ inférieurs ou égaux à $100$.
\item \'Ecrire une fonction \texttt{LJ} d'argument $n$ renvoyant la liste des nombres de type $J$ inférieurs ou égaux à $n$.
\item \'Ecrire une fonction \texttt{U} d'argument $n$ renvoyant $u_n$, le nombre de nombres de type $J$ inférieurs ou égaux à $n$.
\item Vers quelle limite $\ell$ semble tendre $4n/u_n^2$ quand $n$ tend vers l'infini ?
\item Déterminer le premier $n$ pour lequel la différence en valeur absolue entre $4n/u_n^2$ et $\ell$ est inférieure à $10^{-3}$.
\end{enumerate}
\end{exercice} 
 
 
\begin{exercice}
{Dans cet exercice, un segment $[a,b]$ ($a\l b$) est représenté par une liste de taille $2$ : \texttt{[a,b]}.
\begin{enumerate}
\item Deux segments sont disjoints si leur intersection est vide. \'Ecrire une fonction \texttt{disjoints} de deux arguments \texttt{i1} et \texttt{i2} qui teste si les segments \texttt{i1} et \texttt{i2} sont disjoints. La fonction renvoie le booléen \texttt{True} s'ils sont disjoints, \texttt{False} sinon.
\item Une ``\textit{liste bien formée}'' est une liste de segments qui vérifie les propriétés suivantes :
\begin{itemize}
\item les segments sont deux à deux disjoints ;
\item les segments de la liste sont classés par ordre croissant, en considérant qu'un segment \texttt{i1} est strictement plus petit qu'un segment \texttt{i2} si et seulement si le maximum de \texttt{i1} est strictement inférieur au minimum de \texttt{i2}. 
\end{itemize}
\'Ecrire une fonction \texttt{decouper} de deux arguments, une valeur \texttt{x} et une liste de segments \texttt{L} bien formée, qui renvoie un triplet décomposé de :
\begin{itemize}
\item une liste bien formée contenant les segments strictement inférieurs à la valeur \texttt{x} ;
\item une liste bien formée contenant, s'il existe, le segment contenant la valeur \texttt{x}, et sinon, une liste vide ;
\item une liste bien formée contenant les segments strictement supérieurs à la valeur \texttt{x}.
\end{itemize}
\item \'Ecrire une fonction \texttt{inserer} de deux arguments, un segment \texttt{i} et une liste de segments \texttt{L} bien formée. Cette fonction renvoie une liste bien formée de segments contenant les segments de \texttt{L} qui sont disjoints de \texttt{i} et :
\begin{itemize}
\item soit l'intervalle \texttt{i} si tous les segments contenus dans \texttt{L} lui sont disjoints ;
\item soit le segment union de \texttt{i} et des segments contenus dans \texttt{L} qui ne sont pas disjoints de \texttt{i}.
\end{itemize}
\end{enumerate}
}
\end{exercice}


 
 
 
 
 
 
 
 
 
 
 
 
 