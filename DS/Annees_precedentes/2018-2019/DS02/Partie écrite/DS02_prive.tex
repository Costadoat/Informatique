\newcommand{\nom}{Production Éolienne/Solaire}
\newcommand{\sequence}{1}
\newcommand{\numero}{02}
\newcommand{\type}{DS}
\newcommand{\descrip}{Production Éolienne/Solaire}

\documentclass[10pt,a4paper]{article}
  \usepackage[french]{babel}
  \usepackage[utf8]{inputenc}
  \usepackage[T1]{fontenc}
  \usepackage{xcolor}
  \usepackage[]{graphicx}
  \usepackage{makeidx}
  \usepackage{textcomp}
  \usepackage{amsmath}
  \usepackage{amssymb}
  \usepackage{stmaryrd}
  \usepackage{fancyhdr}
  \usepackage{lettrine}
  \usepackage{calc}
  \usepackage{boxedminipage}
  \usepackage[french,onelanguage, boxruled,linesnumbered]{algorithm2e}
  \usepackage[colorlinks=false,pdftex]{hyperref}
  \usepackage{minted}
  \usepackage{url}
  %\usepackage[locale=FR]{siunitx}
  \usepackage{multicol}
  \makeindex

  %\graphicspath{{../Images/}}

%  \renewcommand\listingscaption{Programme}

  %\renewcommand{\thechapter}{\Alph{chapter}}
  \renewcommand{\thesection}{\Roman{section}}
  %\newcommand{\inter}{\vspace{0.5cm}%
  %\noindent }
  %\newcommand{\unite}{\ \textrm}
  \newcommand{\ud}{\mathrm{d}}
  \newcommand{\vect}{\overrightarrow}
  %\newcommand{\ch}{\mathrm{ch}} % cosinus hyperbolique
  %\newcommand{\sh}{\mathrm{sh}} % sinus hyperbolique

  \textwidth 160mm
  \textheight 250mm
  \hoffset=-1.70cm
  \voffset=-1.5cm
  \parindent=0cm

  \pagestyle{fancy}
  \fancyhead[L]{\bfseries {\large PTSI -- Dorian}}
  \fancyhead[C]{\bfseries{{\type} {\num}}}
  \fancyhead[R]{\bfseries{\large Informatique}}
  \fancyfoot[C]{\thepage}
  \fancyfoot[L]{\footnotesize R. Costadoat, J. Genzmer, W. Robert}
  \fancyfoot[R]{\small \today}
  
  \definecolor{bg}{rgb}{0.9,0.9,0.9}
  
  
  % macro Juliette
  
\usepackage{comment}   
\usepackage{amsthm}  
\theoremstyle{definition}
\newtheorem{exercice}{Exercice}
\newtheorem*{rappel}{Rappel}
\newtheorem*{remark}{Remarque}
\newtheorem*{defn}{Définition}
\newtheorem*{ppe}{Propriété}
\newtheorem{solution}{Solution}


\excludecomment{solution}
%\excludecomment{exercice}

\begin{document}

\begin{center}
{\Large\bf {\type} \no {\num} -- \descrip}
\end{center}

\SetKw{KwFrom}{de} 

\section{Exercice de cours -- Classement d'une liste}

On considère une liste de valeurs flottantes non classées. La liste des valeurs est déjà définie dans le programme et est stockée dans une variable appelée \texttt{tab}.

Écrire en langage Python un algorithme qui permet d'obtenir à la fin une nouvelle liste des valeurs classées par ordre croissant, stockée dans une variable appelée \texttt{tab2}.

Le programme suivant qui permet de classer une liste.
\begin{minted}{python}
tab2=[]
for i in range(1,len(tab)-1):
    min=i
    for j in range(i+1,len(tab)):
        if tab[j] < tab[min]:
            min=j
    tmp=tab[i]
    tab[i]=tab[min]
    tab[min]=tmp
    tab2.append(tab[min])
\end{minted}


\section{Exercice de TP -- Recherche d'un mot dans une cha\^ ine de caract\` eres}
\begin{enumerate}
\item \begin{minted}[frame=lines]{python}
def estIci(motif,texte,i):
    k = 0
    p = len(motif)
    # on parcourt le texte à partir de l'index i 
    # tant que on ne dépasse pas la longueur de motif 
    # et que texte et motif sont identiques
    while k<p and motif[k] == texte[k+i]:
        k = k+1
    # si on a parcouru p termes, alors, motif est dans texte à l'index i
    if k==p:
        resultat=True
    else:
        resultat=False    	
    return resultat                
\end{minted}
\item Le pire des cas s'obtient avec un texte type : \verb?texte='aaaaa...aaaaa'? et un motif du type : \verb?'aa..aab'?. On note $n$ la longueur du texte et $p$ celle du motif.\\
Compté grossièrement : dans \verb?estIci(motif,texte,i)?, on effectue $p$ tests. On répète $n-p$ fois cette opération dans \verb?recherche(motif,texte)?. Donc on s'attend à une complexité en : \fbox{$O((n-p)p)$.} \\
On obtient $O(np)$ si $n>>p$\bigskip \\
Plus précisément, avec le pire des cas :\\
dans \verb?estIci(motif,texte,i)?, on a :
\begin{itemize}
\item 2 affectations avant la boucle
\item on répète $p$ fois :
\begin{itemize}
\item 2 tests
\item 1 affectation
\item 1 addition
\end{itemize}
\item 1 test
\item 1 affectation
\end{itemize}
\fbox{Au total, on a comme nombres d'opérations : $2+4p+2=4p+4=O(p)$.}\\
 \verb?recherche(motif,texte)?, on a :
 \begin{itemize}
 \item 2 affectations
 \item dans la boucle else : 2 affectations
 \item on répète $n-p$ fois :
 \begin{itemize}
 \item 2 tests
 \item 1 affectation
 \item 1 calcul de \verb?estIci?
 \item 1 afffectation
 \item 1 addition
 \end{itemize}
 \end{itemize}
Au total, on a comme nombres d'opérations : \\
 \fbox{$2+2+(n-p)\times (5+4p+4)=4+(n-p)(4p+9)=O((n-p)p)$.}
\end{enumerate}


\section{Exercice -- Croissance d'une suite}
\begin{minted}[frame=lines]{python}
n=len(u)

i=0
while i<=n-2 and u[i]<u[i+1]:
    i=i+1

if i==len(u)-1:
    print 'u est strictement croissante'
else:
    print 'a partir du rang',i,'la suite n est plus strictement croissante'
\end{minted}


\end{document}
