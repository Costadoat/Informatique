\documentclass[a4paper,12pt]{article}

\input{../../../macro_Juliette}

\excludecomment{solution}

\begin{document}
\noindent {PTSI \hfill 2018-2019}

 \begin{center}
  \begin{large}
  \fbox{DS Informatique n°2 -- Partie machine. Durée : 1 heure 30}
  \end{large}
 \end{center}

\begin{boxedminipage}{.9\textwidth} 
\begin{itemize}
 \item Faire tous les exercices dans un même fichier NomPrenom.py que vous sauvegarderez,
 \item Mettez en commentaire l'exercice que vous traitez,
 \item N'oubliez pas de commentez votre code,
 \item Il est possible de demander un déblocage pour une question, mais celle-ci sera notée 0.
\end{itemize}
\end{boxedminipage}

\section*{Réponse temporelle théorique}
\noi On considère la tension au bornes d'un condensateur lors de la charge : $u(t)=E\cdot(1-e^{\frac{-t}{\tau}})$, avec $E=10V$ et $\tau=0,06s$.

\begin{enumerate}
\item \begin{enumerate}
\item Définir une fonction \verb?f(t)? qui retourne $u(t)$ pour une entrée $t$ (vous utiliserez la bibliothèque numpy de calcul numérique).
\begin{solution}~\ \\
\begin{verbatim}
import numpy as np

E=10
tau=0.06
def f(t):
    return E*(1-np.exp(-t/tau))
\end{verbatim}
\end{solution}
\item En utilisant la bibliothèque \verb?matplotlib?, tracer cette fonction pour $t\in[0,1]$ avec des incréments de $0,01s$.
\begin{solution}~\ \\
\begin{verbatim}
import matplotlib.pyplot as plt

t=np.arange(0,1,0.01)
plt.plot(t,f(t))
\end{verbatim}
\end{solution}
\end{enumerate}
\item \begin{enumerate}
\item On notera par la suite : $\displaystyle \lim_{t\to +\infty}u(t)=u(+\infty)$ et on fera l'hypothèse que la valeur peut être assimilée à \verb?f(1)?. Afficher la valeur $0,95\cdot u(+\infty)$.
\begin{solution}~\ \\
\begin{verbatim}
print 0.95*f(1)
\end{verbatim}
\end{solution}
\item Définir une fonction \verb?g(t)? qui retourne $u(t)-0,95\cdot u(+\infty)$ pour une entrée $t$.
\begin{solution}~\ \\
\begin{verbatim}
def g(t):
    return f(t)-0.95*f(1)
\end{verbatim}
\end{solution}
\item Définir une fonction \verb?dichotomie(h,p)? qui permet grâce à la méthode de la dichotomie de trouver une racine d'une fonction $h$.
\begin{solution}~\ \\
\begin{verbatim}
p=10**(-6)

def dichotomie(h,p):
    a=-10
    i=1
    b=10
    while (b-a)>p :
        if h(a)*h((a+b)/2)<=0:
            a, b=a, (a+b)/2.
        else:
            a, b=(a+b)/2., b
        i=i+1
        e=b-a
    return "Resultat {} (Erreur:{})".format(a, e)
\end{verbatim}
\end{solution}
\item Utiliser cette fonction \verb?dichotomie? afin de déterminer le temps de réponse à 5\%.
\begin{solution}~\ \\
\begin{verbatim}
print dichotomie(g,p)
\end{verbatim}
\end{solution}
\item Exécuter la commande $3*tau$ pour vérifier que le résultat de la question précédente est juste.
\begin{solution}~\ \\
\begin{verbatim}
print 3*tau
\end{verbatim}
\end{solution}
\end{enumerate}
\end{enumerate}

\section*{Réponse temporelle expérimentale}

Le fichier \verb?trace.csv? donné correspond à une mesure du déplacement du Control'X. Les colonnes sont séparées par des \verb?;? et les lignes par des retours à la ligne \verb?\n?:
\begin{itemize}
 \item la première colonne est à mettre dans une liste \verb?temps?,
 \item la seconde colonne est à mettre dans une liste \verb?consigne?,
 \item la troisième colonne est à mettre dans une liste \verb?sortie?.
\end{itemize}

Le code suivant permet d'ouvrir le fichier \verb?trace.csv? et de remplir les listes \verb?temps?, \verb?consigne? et \verb?sortie? à partir des données du fichier csv.

\begin{verbatim}
fichier=open("trace.csv",'r')
contenu=fichier.read()
fichier.close
lignes=contenu.split("\n")
temps=[]
consigne=[]
sortie=[]
for ligne in lignes[0:-1]:
    temps.append(float(ligne.split(';')[0]))
    consigne.append(float(ligne.split(';')[1]))
    sortie.append(float(ligne.split(';')[2]))
\end{verbatim}

\begin{enumerate}
\item \begin{enumerate}
\item Tracer une figure avec en ordonnée les données de la liste \verb?sortie? et en abscisse celles de la liste \verb?temps?.
 \begin{solution}~\ \\
\begin{verbatim}
plt.plot(temps,sortie,'g')
\end{verbatim}
%plt.plot(temps,consigne,'b')
\end{solution}
\item Tracer en rouge sur la figure les limites de +5\% et -5\% de la valeur finale.
 \begin{solution}~\ \\
\begin{verbatim}
smoins=0.95*sortie[np.shape(temps)[0]-1]
splus=1.05*sortie[np.shape(temps)[0]-1]
plt.plot(temps,splus*np.ones(np.shape(temps)[0]),'r')
plt.plot(temps,smoins*np.ones(np.shape(temps)[0]),'r')
\end{verbatim}
\end{solution}
\item En déduire graphiquement le temps de réponse à 5\%.
 \begin{solution}~\ \\
\begin{verbatim}
smoins=0.95*sortie[np.shape(temps)[0]-1]
splus=1.05*sortie[np.shape(temps)[0]-1]
plt.plot(temps,splus*np.ones(np.shape(temps)[0]),'r')
plt.plot(temps,smoins*np.ones(np.shape(temps)[0]),'r')
\end{verbatim}
\end{solution}
\end{enumerate}
 \item Écrire un script permettant de déterminer ce temps de réponse à 5\%.
  \begin{solution}~\ \\
\begin{verbatim}
tr=0
for i in range(len(sortie)):
    if sortie[i] < splus and sortie[i] > smoins and (sortie[i-1] > splus or
                                                    sortie[i-1] < smoins) :
        tr=temps[i]
        
print tr
\end{verbatim}
\end{solution}
\end{enumerate}
\end{document}
