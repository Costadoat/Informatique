\newcommand{\nom}{Porte conteneur}
\newcommand{\sequence}{03}
\newcommand{\num}{04}
\newcommand{\type}{TD}
\newcommand{\descrip}{Résolution d'un problème en utilisant des méthodes algorithmiques}
\newcommand{\competences}{Alt-C3: Concevoir un algorithme répondant à un problème précisément posé}
\documentclass[10pt,a4paper]{article}
  \usepackage[french]{babel}
  \usepackage[utf8]{inputenc}
  \usepackage[T1]{fontenc}
  \usepackage{xcolor}
  \usepackage[]{graphicx}
  \usepackage{makeidx}
  \usepackage{textcomp}
  \usepackage{amsmath}
  \usepackage{amssymb}
  \usepackage{stmaryrd}
  \usepackage{fancyhdr}
  \usepackage{lettrine}
  \usepackage{calc}
  \usepackage{boxedminipage}
  \usepackage[french,onelanguage, boxruled,linesnumbered]{algorithm2e}
  \usepackage[colorlinks=false,pdftex]{hyperref}
  \usepackage{minted}
  \usepackage{url}
  \usepackage[locale=FR]{siunitx}
  \usepackage{multicol}
  \usepackage{tikz}
  \makeindex

  %\graphicspath{{../Images/}}

%  \renewcommand\listingscaption{Programme}

  %\renewcommand{\thechapter}{\Alph{chapter}}
  \renewcommand{\thesection}{\Roman{section}}
  %\newcommand{\inter}{\vspace{0.5cm}%
  %\noindent }
  %\newcommand{\unite}{\ \textrm}
  \newcommand{\ud}{\mathrm{d}}
  \newcommand{\vect}{\overrightarrow}
  %\newcommand{\ch}{\mathrm{ch}} % cosinus hyperbolique
  %\newcommand{\sh}{\mathrm{sh}} % sinus hyperbolique

  \textwidth 160mm
  \textheight 250mm
  \hoffset=-1.70cm
  \voffset=-1.5cm
  \parindent=0cm

  \pagestyle{fancy}
  \fancyhead[L]{\bfseries {\large PTSI -- Dorian}}
  \fancyhead[C]{\bfseries{{\type} \no \numero}}
  \fancyhead[R]{\bfseries{\large Informatique}}
  \fancyfoot[C]{\thepage}
  \fancyfoot[L]{\footnotesize R. Costadoat, C. Darreye}
  \fancyfoot[R]{\small \today}
  
  \definecolor{bg}{rgb}{0.9,0.9,0.9}
  
  
  % macro Juliette
  
\usepackage{comment}   
\usepackage{amsthm}  
\theoremstyle{definition}
\newtheorem{exercice}{Exercice}
\newtheorem*{rappel}{Rappel}
\newtheorem*{remark}{Remarque}
\newtheorem*{defn}{Définition}
\newtheorem*{ppe}{Propriété}
\newtheorem{solution}{Solution}

\newcounter{num_quest} \setcounter{num_quest}{0}
\newcounter{num_rep} \setcounter{num_rep}{0}
\newcounter{num_cor} \setcounter{num_cor}{0}

\newcommand{\question}[1]{\refstepcounter{num_quest}\par
~\ \\ \parbox[t][][t]{0.15\linewidth}{\textbf{Question \arabic{num_quest}}}\parbox[t][][t]{0.85\linewidth}{#1\label{q\the\value{num_quest}}}\par
~\ \\}

\newcommand{\reponse}[4][1]
{\noindent
\rule{\linewidth}{.5pt}\\
\textbf{Question\ifthenelse{#1>1}{s}{} \multido{}{#1}{%
\refstepcounter{num_rep}\ref{q\the\value{num_rep}} }:} ~\ \\
\ifdef{\public}{#3 ~\ \\ \feuilleDR{#2}}{#4}
}

\newcommand{\cor}
{\refstepcounter{num_cor}
\noindent
\rule{\linewidth}{.5pt}
\textbf{Question \arabic{num_cor}:} \\
}


\excludecomment{solution}
%\excludecomment{exercice}

\begin{document}

\begin{center}
{\Large\bf {\type} \no {\num} -- \descrip}
\end{center}

\SetKw{KwFrom}{de} 

\fbox{Les codes en python doivent être commentés et les indentations dans le 
code doivent être visibles.}

\section{Exercice de cours -- Calcul de factorielle}
\begin{enumerate}
\item Ecrire une fonction \verb?factorielle? qui prend comme entrée $n$, un entier positif et qui renvoie la valeur de $n!$.\\
(On rappelle que, par convention, $0!=1$.)
\item Montrer que la complexité en temps de l'algorithme précédent est linéaire, c'est-à-dire en $O(n)$.
\end{enumerate}


\section{Exercice de TP -- Recherche du maximum dans une liste de nombres.}
\begin{enumerate}
\item Ecrire une fonction \verb?maximum(liste)? qui prend comme entrée une liste de nombres non tri\' ee et renvoie le maximum de cette liste.
\begin{minted}{python}
>>>L=[2,8,-7,3]
>>>maximum(L)
8
\end{minted}
\item Ecrire une fonction \verb?positionMax(liste)? qui renvoie le maximum et la position de ce maximum pour une liste de nombres :
\begin{minted}{python}
>>>positionMax(L)
(8,1)
\end{minted}
\end{enumerate}

\section{Exercice -- Variant de boucle}
Soit $n$ et $m$ deux entiers naturels non nuls donnés. On considère l'algorithme suivant : 
\begin{minted}[frame=lines]{python}
while n!=m :
	if n>m :
		n=n-m
	else:
		m=m-n	
\end{minted}
\begin{enumerate}
\item Rappeler la définition d'un variant de boucle.
\item Prouver que la boucle se termine en montrant que Max$(n,m)$ est un variant de boucle.
\end{enumerate}


\section{Exercice -- Evaluation d'un polynôme}
Soit $P$ un polynôme : $P(x)=a_0+a_1x+a_2x^2+\cdots +a_nx^n$. \\
On représente $P$ par une liste qui contient ses coefficients : $L_p=(a_0,a_1,\cdots,a_n)$.
\begin{enumerate}
\item Ecrire une fonction \verb?evalu(Lp,b)? qui prend comme entrée une liste de réels $L_p$ de taille quelconque et un réel $b$. La fonction renvoie la valeur de $P(b)$.
\item Ecrire une fonction \verb?evalu2(Lp,b)? qui renvoie la valeur de $P(b)$ mais sans utiliser la fonction puissance \verb?**?.
\item Combien fait-on de multiplications dans \verb?evalu2? ?
\item La méthode de Horner permet de calculer $P(b)$ en utilisant moins d'opérations de multiplication. Elle se base sur le constat suivant :
\[P(b)=(((a_nb+a_{n-1})\times b+a_{n-2})\times b+\cdots a_2)\times b+a_1)\times b+a_0\] 
Ecrire une fonction \verb?evaluHorner(Lp,b)? qui renvoie la valeur de $P(b)$ en utilisant le principe précédent.
\item Combien fait-on de multiplications avec la méthode de Horner ?
\end{enumerate}





\ifdef{\public}{\end{document}}{}

\newpage 


\begin{center}
{\Large\bf Correction DS \no {\num} -- \descrip}
\end{center}

\setcounter{section}{2}
\section{Exercice -- Variant de boucle}
\begin{enumerate}
\item $\max(n,m)$ est un entier ;
\item Au départ, $n$ et $m$ sont strictement positifs. A l'étape $k$,  si $n_k>0$ et $m_k>0$, alors $n_{k+1}$ et $m_{k+1}$ sont encore strictement positifs. En effet : si $n_k>m_k$, alors $n_{k+1}=n_k-m_k > 0$. Si $n_k<m_k$, alors $m_{k+1}=m_k-n_k > 0$. \\
Donc $\max (m,n)$ est positif.
\item $\max(m,n)$ est strictement décroissante. En effet, dans la boucle while, $n_k\neq m_k$. Il y a donc deux possibilités. Si $n_k>m_k$ alors $\max(n_k,m_k)=n_k$. Dans ce cas : $n_{k+1}=n_k-m_k<n_k=\max(n_k,m_k)$. Donc $n_{k+1}<\max(n_k,m_k)$ et $m_{k+1}=m_k<n_k=\max(n_k,m_k)$. Finalement, les deux nombres $n_{k+1}$ et $m_{k+1}$ sont strictement inférieurs à $\max(n_k,m_k)$. Donc : $\max(n_{k+1},m_{k+1})<\max(n_k,m_k)$.\\
L'autre cas $n_k<m_k$ est similaire.\\
\end{enumerate}
\fbox{Donc $\max(n,m)$ est un variant de boucle. Le programme s'arrête.}


\section{Exercice -- Evaluation d'un polynôme}
\begin{enumerate}
\item 
\begin{minted}[frame=lines]{python}
def evalu(Lp,x0):
    deg=len(Lp)-1
    s=0
    for i in range(deg+1):
        s=s+Lp[i]*x0**i
    return(s) 
\end{minted}
\item 
\begin{minted}[frame=lines]{python}
def evalu2(Lp,x0):
    deg=len(Lp)-1
    s=0
    x=1
    for i in range(deg+1):
        s=s+Lp[i]*x
        x=x*x0
    return(s)
\end{minted}
\item On effectue deux multiplications dans la boucle. Donc \fbox{on effectue $2(\deg(P)+1)$ multiplications.}
\item 
\begin{minted}[frame=lines]{python}
def evaluHorner(Lp,x0):
    deg=len(Lp)-1
    s=Lp[deg]
    for i in range(1,deg+1):
        s=x0*s+Lp[deg-i]
    return(s)
\end{minted}
\item On effectue un multiplication dans la boucle. Donc \fbox{on effectue $\deg(P)$ multiplications.}
\end{enumerate}
\end{document}





    


