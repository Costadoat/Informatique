\newcommand{\nom}{Porte conteneur}
\newcommand{\sequence}{03}
\newcommand{\num}{04}
\newcommand{\type}{TD}
\newcommand{\descrip}{Résolution d'un problème en utilisant des méthodes algorithmiques}
\newcommand{\competences}{Alt-C3: Concevoir un algorithme répondant à un problème précisément posé}
\documentclass[10pt,a4paper]{article}
  \usepackage[french]{babel}
  \usepackage[utf8]{inputenc}
  \usepackage[T1]{fontenc}
  \usepackage{xcolor}
  \usepackage[]{graphicx}
  \usepackage{makeidx}
  \usepackage{textcomp}
  \usepackage{amsmath}
  \usepackage{amssymb}
  \usepackage{stmaryrd}
  \usepackage{fancyhdr}
  \usepackage{lettrine}
  \usepackage{calc}
  \usepackage{boxedminipage}
  \usepackage[french,onelanguage, boxruled,linesnumbered]{algorithm2e}
  \usepackage[colorlinks=false,pdftex]{hyperref}
  \usepackage{minted}
  \usepackage{url}
  \usepackage[locale=FR]{siunitx}
  \usepackage{multicol}
  \usepackage{tikz}
  \makeindex

  %\graphicspath{{../Images/}}

%  \renewcommand\listingscaption{Programme}

  %\renewcommand{\thechapter}{\Alph{chapter}}
  \renewcommand{\thesection}{\Roman{section}}
  %\newcommand{\inter}{\vspace{0.5cm}%
  %\noindent }
  %\newcommand{\unite}{\ \textrm}
  \newcommand{\ud}{\mathrm{d}}
  \newcommand{\vect}{\overrightarrow}
  %\newcommand{\ch}{\mathrm{ch}} % cosinus hyperbolique
  %\newcommand{\sh}{\mathrm{sh}} % sinus hyperbolique

  \textwidth 160mm
  \textheight 250mm
  \hoffset=-1.70cm
  \voffset=-1.5cm
  \parindent=0cm

  \pagestyle{fancy}
  \fancyhead[L]{\bfseries {\large PTSI -- Dorian}}
  \fancyhead[C]{\bfseries{{\type} \no \numero}}
  \fancyhead[R]{\bfseries{\large Informatique}}
  \fancyfoot[C]{\thepage}
  \fancyfoot[L]{\footnotesize R. Costadoat, C. Darreye}
  \fancyfoot[R]{\small \today}
  
  \definecolor{bg}{rgb}{0.9,0.9,0.9}
  
  
  % macro Juliette
  
\usepackage{comment}   
\usepackage{amsthm}  
\theoremstyle{definition}
\newtheorem{exercice}{Exercice}
\newtheorem*{rappel}{Rappel}
\newtheorem*{remark}{Remarque}
\newtheorem*{defn}{Définition}
\newtheorem*{ppe}{Propriété}
\newtheorem{solution}{Solution}

\newcounter{num_quest} \setcounter{num_quest}{0}
\newcounter{num_rep} \setcounter{num_rep}{0}
\newcounter{num_cor} \setcounter{num_cor}{0}

\newcommand{\question}[1]{\refstepcounter{num_quest}\par
~\ \\ \parbox[t][][t]{0.15\linewidth}{\textbf{Question \arabic{num_quest}}}\parbox[t][][t]{0.85\linewidth}{#1\label{q\the\value{num_quest}}}\par
~\ \\}

\newcommand{\reponse}[4][1]
{\noindent
\rule{\linewidth}{.5pt}\\
\textbf{Question\ifthenelse{#1>1}{s}{} \multido{}{#1}{%
\refstepcounter{num_rep}\ref{q\the\value{num_rep}} }:} ~\ \\
\ifdef{\public}{#3 ~\ \\ \feuilleDR{#2}}{#4}
}

\newcommand{\cor}
{\refstepcounter{num_cor}
\noindent
\rule{\linewidth}{.5pt}
\textbf{Question \arabic{num_cor}:} \\
}


\usepackage{enumitem}

\setenumerate[1]{align=left,label=\arabic*}
\setenumerate[2]{before=\stepcounter{enumi},label*=.\arabic*,leftmargin=1.2em,align=left}


\ifdef{\public}{\excludecomment{solution}}


\begin{document}

\begin{center}
{\Large\bf {\type} \no {\numero} -- \descrip}
\end{center}

\SetKw{KwFrom}{de} 

\begin{boxedminipage}{.9\textwidth} 
\begin{itemize}
 \item Faire tous les exercices dans un même fichier {NomPrenom.py} à sauvegarder,
 \item mettre en commentaire l'exercice et la question traités (ex: \# Exercice 1),
 \item ne pas oublier pas de commenter ce qui est fait dans votre code (ex: \# Je crée une fonction pour calculer la racine d'un nombre),
\item il est possible de demander un déblocage pour une question avec une (*), mais celle-ci sera notée 0,
 \item il faut vérifier avant de partir que le code peut s'exécuter et qu'il affiche les résultats que vous attendez.
\end{itemize}
\end{boxedminipage}





\section*{Exercice 1 : import de fichier et recherche par dichotomie}
\noindent Le fichier \verb?liste_nombres.csv? contient une liste de nombres réels, classés par ordre croissant. \\
L'emplacement de ce fichier est le suivant : /home/eleve/Ressources/PTSI/
\begin{enumerate}
\item (*) Importer le fichier. Récupérer le contenu du fichier sous forme d'une liste de \textbf{flottants}.
\begin{solution}~ \\\vspace{-0.7cm}
\begin{minted}{python}
fichier = open("liste_nombres.csv", "r")

contenu=fichier.read()
fichier.close()

lignes=contenu.split('\n')

L=[]
for i in lignes:
    L.append(float(i)) 
\end{minted}
\end{solution}
\item Afficher les 30 premières valeurs de la liste.
\begin{solution}
\verb?L[0:30]?
\end{solution}
\item Ecrire une fonction \verb?dicho? qui prend comme entrée une liste \verb?L? et un nombre \verb?a? et renvoie la position de \verb?a? si $a$ est dans le tableau et \verb?False? sinon.
\begin{solution}~ \\\vspace{-0.7cm}
\begin{minted}{python}
def dicho(L, a):
    debut = 0
    fin = len(L) - 1
    while debut <= fin:
        m = (debut+fin) // 2
        if L[m] == a:
            return m
        elif L[m] < a:
            debut = m + 1
        else:
            fin = m - 1
    return False
\end{minted}
\end{solution}
\item Afficher le résultat de la fonction \verb?dicho? pour la liste précédemment importée pour le nombre $a=1.56$ puis pour le nombre $a=2$.
\begin{solution}
\verb?print(dicho(L,1.56))? \verb?print(dicho(L,2)) ?
\end{solution}
\item Calculer et afficher la moyenne des nombres de la liste importée.
\begin{solution}
~ \\\vspace{-0.7cm}
\begin{minted}{python}
somme=0
for i in L:
	somme=somme+i
print(somme/len(L))	
\end{minted}
\end{solution}
\item Afficher la médiane des nombres de la liste importée.
\begin{solution}
print(L[len(L)//2])
\end{solution}
\end{enumerate}


\section*{Exercice 2 : recherche du maximum}
\noindent Dans cet exercice, vous n'avez pas le droit d'utiliser les fonctions Python suivantes : \verb?max? et \verb?index?.
\begin{enumerate}
\setcounter{enumi}{6}
\item Ecrire une fonction \verb?maxi? qui prend comme entrée une liste de nombres \verb?L? et renvoie la valeur ainsi qu'une position du maximum.
\begin{solution}~ \\\vspace{-0.7cm}
\begin{minted}{python}
def maxi(L):
    maxi=L[0]
    position=0
    for i in range(1,len(L)):
        if L[i]>maxi:
            maxi=L[i]
            position=i
    return(maxi,position)
\end{minted}
\end{solution}
\item Ecrire une fonction \verb?maxi_liste? qui prend comme entrée une liste \verb?L? et renvoie la valeur ainsi que la liste de toutes les positions de cette valeur maximale. 
\begin{solution}~ \\\vspace{-0.7cm}
\begin{minted}{python}
def maxi2(L):
    maxi=L[0]
    position=[0]
    for i in range(1,len(L)):
        if L[i]>maxi:
            maxi=L[i]
            position=[i]
        elif L[i]==maxi:
            position.append(i)
    return(maxi,position)
\end{minted}
\end{solution}
\item Afficher le résultat de la fonction suivante pour la liste : $L=[9,1,3,9,2,9]$.
\begin{solution}~ \\\vspace{-0.7cm}
\verb? print(maxi2([9,8,9,0,9]))?
\end{solution}
\end{enumerate}

\section*{Exercice 3 : algorithme Glouton}
Vous avez un budget maximal $B=11$ euros et des bonbons à acheter. Pour chaque type de bonbon, vous avez deux informations : son prix et votre indice de satisfaction. L'objectif est de maximiser l'indice de satisfaction total.\\
Pour cela deux stratégies gloutonnes :
\begin{itemize}
\item stratégie 1 : choisir toujours le bonbon le plus satisfaisant compatible avec le budget restant ;
\item stratégie 2 : choisir toujours le bonbon au plus fort rapport $\dfrac{\text{satisfaction}}{\text{prix}}$, compatible avec le budget restant.
\end{itemize}
La donnée est la suivante : une liste qui contient des couples [prix,satisfaction]. 
\begin{center}
bonbons=[[3,2.5],[1.5,1.5],[1,0.9]]
\end{center}
Par exemple, le premier bonbon coûte 3 euros et son indice de satisfaction est $2,5$.\\
La liste est triée par ordre décroissant de satisfaction.
\begin{enumerate}
\setcounter{enumi}{9}
\item Recopier la ligne ci-dessus.
\begin{solution}
\verb?bonbons=[[3,2.5],[1.5,1.5],[1,0.9]]?
\end{solution}
\item Afficher \verb?bonbons[2][0]? et \verb?bonbons[2][0]?. A quoi correspondent ces deux nombres ? (réponse en commentaire).
\begin{solution}
Le premier nombre correspond au prix du 3ème bonbon et le second à son indice de satisfaction.
\end{solution}
\item Ecrire un programme qui donne la liste des couples [prix,satisfaction] pour maximiser la satisfaction en suivant la stratégie 1.
\begin{solution}~ \\\vspace{-0.7cm}
\begin{minted}{python}
i=0
achat=[]
while i<len(bonbons) and B>0:
    if bonbons[i][0]<B:
        achat.append(bonbons[i])
        B=B-bonbons[i][0]
    else:
        i=i+1
\end{minted}
\end{solution}
\item Calculer (pas à la main) et afficher la satisfaction totale pour cette stratégie 1.
\begin{solution}~ \\\vspace{-0.7cm}
\begin{minted}{python}
Satisfaction_totale=0
for element in achat:
    Satisfaction_totale=Satisfaction_totale+element[1]
print(Satisfaction_totale)
\end{minted}
\end{solution}
\item (*) On appelle ratio le quotient $\dfrac{\text{satisfaction}}{\text{prix}}$.
Construire (pas à la main) une liste \verb?bonbons2? constituée de sous-listes : [ratio, prix, satisfaction].\\
Ainsi, la liste \verb?bonbons2? devra commencer de cette façon : [[0.8333334, 3, 2.5], ...]
\begin{solution}~ \\\vspace{-0.7cm}
\begin{minted}{python}
bonbons2=[]
for element in bonbons:
    bonbons2.append([element[1]/element[0],element[0],element[1]])
\end{minted}
\end{solution}
\item Recopier la commande suivante : \verb?bonbons2.sort(reverse=True)?. \\
La liste \verb?bonbons2? est alors triée par valeur de ratio décroissante.
\item Ecrire un programme qui donne la liste des triplets [ratio,prix,satisfaction] pour maximiser la satisfaction en suivant la stratégie 2.
\begin{solution}~ \\\vspace{-0.7cm}
\begin{minted}{python}
B=12
i=0
achat2=[]
while i<len(bonbons2) and B>0:
    if bonbons2[i][1]<B:
        achat2.append(bonbons2[i])
        B=B-bonbons2[i][1]
    else:
        i=i+1
\end{minted}
\end{solution}
\item Afficher la satisfaction totale pour cette stratégie 2.
\begin{solution}~ \\\vspace{-0.7cm}
\begin{minted}{python}
Satisfaction_totale2=0
for element in achat2:
    Satisfaction_totale2=Satisfaction_totale2+element[2]
print(Satisfaction_totale2)
\end{minted}
\end{solution}
\item Conclure quant à la meilleure stratégie. (réponse en commentaire)
\begin{solution}
La satisfaction dans la stratégie 2 est plus grande. Donc, c'est la meilleure sur cet exemple.
\end{solution}
\end{enumerate}
\end{document}









\end{enumerate}
\noindent On cherche à sélectionner cinq nombres de la liste suivante en cherchant à avoir leur somme la plus grande possible et en s'interdisant de choisir deux nombres voisins :
\[L=[15,4,20,17,11,8,16,7,14,2,7,5,17,19,18]\]
La stratégie gloutonne consiste à choisir à chaque étape la valeur la plus grande possible dans les choix restants.
\begin{enumerate}
\setcounter{enumi}{10}
\item Ecrire une fonction \verb?somme_glouton? qui prend comme entrée une liste de nombres \verb?L? et qui renvoie la liste des nombres choisis par la stratégie gloutonne. GNA : autorise fonction max ?
\item Afficher le résultat \verb?somme_glouton? pour la liste $L$ ainsi que la somme des valeurs obtenues.
\begin{solution}
\verb?[0, 19, 16, 15, 14]?
\end{solution}
\item Que pensez-vous du résultat précédent ? (réponse en commentaire)
